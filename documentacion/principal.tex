%%%%%%%%%%%%%%%%%%%%%%%%%%%%%%%%%%%%%%%%%%%%%%%%%%%%%%%%%%%%%%%%%%%%%%%%%%%%%%%%%%%%%%%%
%
% Tipo de documento. ¿¿¿Capítulos siempre empiezan a la izquierda????
%%%%%%%%%%%%%%%%%%%%%%%%%%%%%%%%%%%%%%%%%%%%%%%%%%%%%%%%%%%%%%%%%%%%%%%%%%%%%%%%%%%%%%%%%%
\documentclass[11pt,b5paper,openany]{book}
% ---------------- B5, 176x250 mm --------------------------------------- %


%%%%%%%%%%%%%%%%%%%%%%%%%%%%%%%%%%%%%%%%%%%%%%%%%%%%%%%%%%%%%%%%%%%%%%%%%%%%%%%%%%%%%%%%%%
%%%% Paquetes incluidos
%%%%%%%%%%%%%%%%%%%%%%%%%%%%%%%%%%%%%%%%%%%%%%%%%%%%%%%%%%%%%%%%%%%%%%%%%%%%%%%%%%%%%%%%%%
% Codificación del fichero
\usepackage[utf8]{inputenc}
% Escritura e interpretación de comandos en Español
\usepackage[spanish]{babel}
% Uso de palabras acentuadas y simbolos compuestos como un unico caracter
\usepackage[T1]{fontenc}
% Para el uso de colores al conventir a pdf por parte de PDFLaTeX.
\usepackage[pdftex,usenames,dvipsnames]{color}
% Introduce la bibliografía en el índice
\usepackage[nottoc]{tocbibind}
% Modificar el nivel de numeración de secciones, subsecciones, etc
% Ojo!!!, este paquete tiene que ir antes que el paque subfig!!!
\usepackage[titles]{tocloft}
% Paquete para permitir el manejo de imágenes y gráficos. Se compila con PDFLaTeX
% Se permiten imagenes en jpg y tambien en pdf
\usepackage[pdftex]{graphicx}
% Permite colocar varias figuras y tablas en una figura y hacer referencia a ellas
% como fig 3(a) y fig 3(b)
\usepackage{subfig}
% Paquete para cambiar el formato de los capítulos, de las secciones, partes, ...
\usepackage{titlesec}
% Paquete que permite el control sobre los apéndices
\usepackage{appendix}
% Modificar encabezamientos y pie de páginas
\usepackage{fancyhdr}
% Paquete para el manejo de cajas y marcos
\usepackage{fancybox}
% Paquete para  la inclusión de código fuente
\usepackage{listings}
% Paquete para poder utilizar el comando "\multirow" en las tablas.
\usepackage{multirow}
% Paquete para poder escribir en vertical
\usepackage{rotating}
% Paquete para colorear las tablas
\usepackage{colortbl}
% Paquete para tablas largas de más de una página
\usepackage{longtable}
% Página apaisada
\usepackage{lscape}
% AMS Math. Para poder hacer fórmulas matemáticas
\usepackage{amscd,amsmath,amssymb,verbatim}
% Paquete para añadir letras matemáticas, por ejemplo, numero reales.
\usepackage{dsfont,textcomp}
% Para el control de enlaces hipertexto en el documento. Este paquete se cargar el último.
%\usepackage[pdftex]{hyperref}
% Paquete para cambiar el espacio entre items de las citas bibliográficas
\usepackage{natbib}
% Establece el espacio entre items de las citas
\setlength{\bibsep}{2pt}
%Listas personalizables
\usepackage[ampersand]{easylist}
%Justificación
\usepackage[document]{ragged2e}

%%%%%%%%%%%%%%%%%%%%%%%%%%%%%%%%%%%%%%%%%%%%%%%%%%%%%%%%%%%%%%%%%%%%%%%%%%%%%%%%%%%%%%%%%%
%%%% Nuevos comandos, adaptaciones, espaciado entre párrafos
%%%%%%%%%%%%%%%%%%%%%%%%%%%%%%%%%%%%%%%%%%%%%%%%%%%%%%%%%%%%%%%%%%%%%%%%%%%%%%%%%%%%%%%%%%
% -------------------------------------------------------------------------------------
% Numeración hasta las subsubsecciones
\setcounter{secnumdepth}{3}
% Incluye en el índice hasta las subsubsecciones
\setcounter{tocdepth}{3}
% Para cuando quede muy ajustado el número de sección y el texto en el índice.
%\setlength{\cftsubsubsecnumwidth}{5em}
% Lo anterior hecho de otra forma sin usar el paquete ``tocloft``
%\makeatletter
%	\renewcommand*\l@subsubsection{\@dottedtocline{3}{7em}{5em}}
%\makeatother
% -------------------------------------------------------------------------------------


% -------------------------------------------------------------------------------------
% Para cambiar el nombre que pondrá en el Indice de tablas, figuras y apéndice.
\AtBeginDocument
{
	\renewcommand\listtablename{Índice de Tablas}
	\renewcommand\tablename{Tabla}
	\renewcommand\listfigurename{Índice de Figuras}
	\renewcommand{\appendixtocname}{Apéndices}
	\renewcommand{\appendixpagename}{Apéndices}
}
% -------------------------------------------------------------------------------------


% Desactiva métodos taquigráficos en Español que pueden molestarnos
\renewcommand{\shorthandsspanish}{}


% -------------------------------------------------------------------------------------
% Este comando mete una pagina en blanco sin cabeceras ni pie de pagina
\newcommand{\paginavaciacompleta}{\newpage{\thispagestyle{empty}\cleardoublepage}}
\newcommand{\paginavaciasincuerpo}{\cleardoublepage}
% -------------------------------------------------------------------------------------


% -------------------------------------------------------------------------------------
% Refefinición de los comandos \chaptermark y \sectionmark que se encargan
% de lo que va a aparecer en la cabecera de las paginas a izd y dcha
% Esto solo vale y se redefine asi si se utiliza el estilo headings
% \renewcommand{\chaptermark}[1]{%
% 	\markboth{\small\upshape\scfamily\thechapter. \ #1}{}}
% \renewcommand*{\sectionmark}[1]{
% 	\markright{\small\upshape\scfamily\thesection. \ #1}}
% -------------------------------------------------------------------------------------


% -------------------------------------------------------------------------------------
% Sangría para párrafo, tabulaciones de 1 cm
\parindent=1.0cm
% Espacio o separación entre párrafos
\parskip 1.0ex
%\parskip=1mm
% -------------------------------------------------------------------------------------


% -------------------------------------------------------------------------------------
% Margen vertical de salida o ``borde superior de impresion''.
% Colocado 1.35cm menos de lo que trae el tipo de documento por defecto
\voffset -1.35cm
% Aumentamos la altura del texto y la anchura para aprovechar más papel. A partir del
% ``borde superior  de impresion'' empieza a estirar la altura hacia abajo 2.7cm a partir
% de la altura  por texto por defecto (lo impone el tipo de documento). Lo mismo para la
% anchura
\addtolength{\textheight}{2.7cm}
\addtolength{\textwidth}{0.5cm}
% Aumentamos la distancia entre el ``borde izdo de impresion'' y el comienzo del texto en
% anchura para las impares o a derecha y disminuimos para las pares o a izquierdas
\addtolength{\oddsidemargin}{0.6cm} %Pares que estan a izquierda
\addtolength{\evensidemargin}{-1.1cm} %Impares que esta a derecha
% -------------------------------------------------------------------------------------


% -------------------------------------------------------------------------------------
% Impiden que una página acabe en una línea huérfana o empiece con una línea viuda
\clubpenalty=10000
\widowpenalty=1000
% División correcta de algunas palabras
\hyphenation{di-fe-ren-cia-les}
% -------------------------------------------------------------------------------------

%%%%%%%%%%%%%%%%%%%%%%%%%%%%%%%%%%%%%%%%%%%%%%%%%%%%%%%%%%%%%%%%%%%%%%%%%%%%%%%%%%%%%%%%%%
%%%% Formato del título de los capítulos
%%%%%%%%%%%%%%%%%%%%%%%%%%%%%%%%%%%%%%%%%%%%%%%%%%%%%%%%%%%%%%%%%%%%%%%%%%%%%%%%%%%%%%%%%%
% FORMATO 1
\newcommand{\bigrule}{\titlerule[0.5mm]}
% cambiamos el formato de los capítulos
\titleformat{\chapter}[display]
% por defecto se usarán caracteres de tamaño \Huge en negrita
{\bfseries\Huge}
{% contenido de la etiqueta
% línea horizontal
\titlerule[0.5mm]
% texto alineado a la izquierda
\filright
% "Capítulo" o "Apéndice" en tamaño \Large en lugar de \Huge
\Large\chaptertitlename\
% número de capítulo en tamaño \Large
\Large\thechapter}
% espacio mínimo entre etiqueta y cuerpo
{0mm}
% texto del cuerpo alineado a la izquierda
{\filright}
% después del cuerpo, dejar espacio vertical y trazar línea horizontal gruesa
[\vspace{0.5mm} \bigrule]

%%%%%%%%%%%%%%%%%%%%%%%%%%%%%%%%%%%%%%%%%%%%%%%%%%%%%%%%%%%%%%%%%%%%%%%%%%%%%%%%%%%%%%%%%%
%%%%  Unidades frontales del book
%%%%%%%%%%%%%%%%%%%%%%%%%%%%%%%%%%%%%%%%%%%%%%%%%%%%%%%%%%%%%%%%%%%%%%%%%%%%%%%%%%%%%%%%%%
\begin{document}
	\frontmatter
		% Portada interior
		%%%%%%%%%%%%%%%%%%%%%%%%%%%%%%%%%%%%%%%%%%%%%%%%%%%%%%%%%%%%%%%%%%%%%%%%%%%%%%%%%%%%%%%%%%
%%%%  Portada interior
%%%%%%%%%%%%%%%%%%%%%%%%%%%%%%%%%%%%%%%%%%%%%%%%%%%%%%%%%%%%%%%%%%%%%%%%%%%%%%%%%%%%%%%%%%
\thispagestyle{empty}
	\begin{center}
		{\huge\scshape\rmfamily\bf UNIVERSIDAD DE CÓRDOBA} \\[.5cm]
		{\large\scshape Escuela Politécnica Superior} \\[1cm]
		{\huge \large \scshape\bf  INTRODUCCIÓN AL APRENDIZAJE AUTOMÁTICO} \\[0.5cm]
		{\large\scshape Departamento de Informática y Análisis Numérico} \\[2cm]
	\end{center}
	\begin{center}
		\includegraphics[scale=.5]{./figuras/logo_uco_eps.png}\\[2cm]
	\end{center}
	\begin{center}
		\textbf{\begin{huge}Documentación de Prácticas\end{huge}} \\
		\rule{\textwidth}{0.01cm} \\[1cm]
		{\large \textbf{Juan Jesús Carmona Tejero}}\\[.2cm]
		{\large \textbf{Gregorio Corpas Prieto}}\\[1.5cm]
		%{Córdoba \hspace{8cm} 2017}\\[5.0cm]
	\end{center}
		%\paginavaciacompleta
		%%%%%%%%%%%%%%%%%%%%%%%%%%%%%%%%%%%%%%%%%%%%%%%%%%%%%%%%%%%%%%%%%%%%%%%%%%%%%%%%%%%%%%%%%%%
%%%%  Portada interior
%%%%%%%%%%%%%%%%%%%%%%%%%%%%%%%%%%%%%%%%%%%%%%%%%%%%%%%%%%%%%%%%%%%%%%%%%%%%%%%%%%%%%%%%%%
\thispagestyle{empty}
	\begin{center}
		{\huge\scshape\rmfamily\bf UNIVERSIDAD DE GRANADA} \\[0.5cm]
		{\large\scshape Departamento de Ciencias de la Computación e
		Inteligencia Artificial} \\[0.25cm]
		{\large \scshape Programa Oficial de Postgrado ``Diseño, Análisis y
		Aplicaciones de Sistemas Inteligentes''}
		\vspace{-0.22cm}
	\end{center}
	\begin{figure}[h!]
	\centering
	\includegraphics[keepaspectratio,width=9cm]{./figuras/ugrdecsai.jpg}
	\end{figure}
	\vspace{-0.35cm}
% 	\begin{center}
% 		\includegraphics[scale=.4]{./figuras/ugrdecsai.jpg} \\
% 	\end{center}
	\begin{center}
		\rule{\textwidth}{0.03cm} \\[0.3cm]
		{\huge \sffamily Algoritmos de aprendizaje bioinspirados multi-objetivo para el
		diseño de modelos	de	redes neuronales artificiales en clasificación} \\
		\rule{\textwidth}{0.03cm} \\[0.3cm]
		{\large\scshape Memoria de Tesis que prensenta} \\[0.2cm]
		{\large \textbf{Juan Carlos Fernández Caballero}} \\[0.2cm]
		{\large\scshape como requisito para optar al grado de \\ Doctor en Informática}
\\[0.8cm]
		{\large\scshape Tutor de Tesis} \\[0.1cm]
		{\large \textbf{Francisco Herrera Triguero}} \\[0.3cm]
		{\large\scshape Directores de Tesis} \\[0.1cm]
		{\large \textbf{César Hervás Martínez}} \\
      {\large \textbf{Francisco José Martínez Estudillo}} \\[0.5cm]
		{Granada \hspace{7cm} 2010}
	\end{center}


		%\paginavaciacompleta
		% Firmas
		%\thispagestyle{empty}
\noindent La memoria titulada “Algoritmos de apredizaje bioinspirados multiobjetivo para
el	diseño de modelos de redes neuronales artificiales en clasificación``, que presenta
	D. Juan Carlos Fernández Caballero para optar al grado de doctor, se ha realizado
	dentro del programa de doctorado “Diseño, Análisis y Aplicaciones de Sistemas
	Inteligentes” del Departamento de Ciencias de la Computación e Inteligencia
	Artificial de la Universidad de Granada, bajo la dirección del Catedrático de
	Universidad César Hervás Martínez, del Departamento de Informática y Análisis Numérico de la
	Universidad de Córdoba, y del Doctor Francisco Jośe Martínez Estudillo, del Departamento de
	Gestión y Métodos Cuantitativos de ETEA, Córdoba.
	\begin{flushright}
		Granada, Septiembre de 2010
	\end{flushright}
	\vspace*{2cm}
	\begin{tabular}{cc}
		El doctorando&El director\\
		& \\
		& \\
		& \\
		& \\
		& \\
		Fdo: Juan Carlos Fernández Caballero&    Fdo: César Hervás Martínez\\
	\end{tabular}
	\vspace*{2cm}
	\begin{center}
		El director
	\end{center}
	\vspace*{2.5cm}
	\begin{center}
		Fdo: Francisco Jośe Martínez Estudillo
	\end{center}


	  % \paginavaciacompleta
		% Subvenciones
		%\thispagestyle{empty}

\vspace*{2cm}

\begin{center}
Esta Tesis Doctoral ha sido subvencionada parcialmente con cargo a los Proyectos
\textbf{TIN 2005-08386-C05-02} y \textbf{TIN2008-06681-C06-03} de la Comisión
Interministerial de Ciencia y Tecnología (CICYT) y con fondos FEDER.\end{center}

\begin{center}
También ha sido subvencionada parcialmente por el Proyecto de Excelencia
\textbf{P08-TIC-3745} de la Junta de Andalucía.\end{center}
\begin{center}

La investigación de Juan Carlos Fernández Caballero ha sido subvencionada hasta
marzo del 2009 por el programa predoctoral de Formación de Personal Investigador (FPI,
referencia de beca \textbf{BES-2006-12543}) del Ministerio de Educación y
Ciencia.\end{center}

\begin{figure*}[htb]
\centering
\subfloat{\includegraphics[width=.35\textwidth]{figuras/escudo-ministerio.jpg}}
\hspace{0.1\linewidth}
\subfloat{\includegraphics[width=.25\textwidth]{figuras/escudo-junta.jpg}}
\end{figure*}
	   %\paginavaciacompleta
	   % Frase moraleja
		%\thispagestyle{empty}
\begin{flushright}
\textit{Si vivimos el presente pensando en el futuro,\\
	y cuando llega el futuro rápidamente lo sentimos\\
   como pasado, volveremos a no vivir el presente.}
\end{flushright}
\rightline{María Jesús Álava Reyes.}

	  % \paginavaciacompleta
		% Dedicatoria
		%\newpage
\thispagestyle{empty}
\vspace*{2cm}
\begin{flushright}
\begin{large}\textbf{Aquí irían las dedicatorias...}\end{large}\end{flushright}

	   %\paginavaciacompleta
		% Agradecimientos
		%\newpage
\thispagestyle{empty}
\Huge \bf
\begin{flushleft}
Agradecimientos\end{flushleft} \rm \normalsize
\vspace{1cm}

Aquí irían los agradecimientos....

	   %\paginavaciacompleta
		
		% Tabla de contenidos o Índice
		\tableofcontents
		%\paginavaciacompleta
		% Lista de figuras DESCOMENTAR !!!!
		\listoffigures
		%\paginavaciacompleta
		% Lista de tablas DESCOMENTAR !!!!
		\listoftables
		%\paginavaciacompleta

%%%%%%%%%%%%%%%%%%%%%%%%%%%%%%%%%%%%%%%%%%%%%%%%%%%%%%%%%%%%%%%%%%%%%%%%%%%%%%%%%%%%%%%%%%
%%%%  Unidad principal del book
%%%%%%%%%%%%%%%%%%%%%%%%%%%%%%%%%%%%%%%%%%%%%%%%%%%%%%%%%%%%%%%%%%%%%%%%%%%%%%%%%%%%%%%%%%
	\mainmatter
		%------------------------------------------------------------------------------
		% Reestablecimiento del estilo con cabecera y pie de pagina
		% Los cambios se deben hacer despues del establecimiento del estilo a fancy
		\pagestyle{fancy}
		% Reinicialización y limpieza de los estilos
		\fancyhf{}
		% Grosor de línea para la cabecera de las páginas
      \renewcommand{\headrulewidth}{0.8pt}
		% Cabecera de las paginas pares (even, E) o a izquierdas
		%\fancyhead[LE]{\thesection \ Nombre	de la seccion}
		\fancyhead[LE]{\nouppercase\rightmark\small\upshape}
		% Cabecera de las paginas impares (odd, O) o a derechas
		%\fancyhead[RO]{\chaptername \ \thechapter. Nombre del capitulo}
		\fancyhead[RO]{\nouppercase\leftmark\small\upshape}
		% Grosor de linea para los pie de pagina
		\renewcommand{\footrulewidth}{0.6pt}
		% Pie pagina Pares
		%\fancyfoot[LE]{%
		  %\textcolor[rgb]{0.75,0.75,0.75}{\textbf{\textit{Prácticas}}}}
		%\fancyfoot[RE]{%
		  %\textcolor[rgb]{0.75,0.75,0.75}{\textbf{\textit{IAA}}}}
		% Pie pagina Impares
		%\fancyfoot[LO]{%
		  %\textcolor[rgb]{0.75,0.75,0.75}{\textbf{\textit{Universidad de Córdoba}}}}
		%\fancyfoot[RO]{%
		  %\textcolor[rgb]{0.75,0.75,0.75}{\textbf{\textit{DIAN}}}}
		% Tanto para pares como para impares
		\fancyfoot[C]{\thepage}
		%------------------------------------------------------------------------------


		% --------------------------------------------------------------------------------
		% Tipo de Numeración
		\pagenumbering{arabic}
		% Empiezo de conteo a partir de aquí y con el número 1
		\setcounter{page}{1}
		% --------------------------------------------------------------------------------


		% --------------------------------------------------------------------------------
		%\linespread{1.3}
		% A partir de aqui el interlineado será de 1.5. No es aplicable al titulo de una
      %seccion, subsección, etc, solo al cuerpo que hay dentro de cada una de ellas
		%\renewcommand*{\baselinestretch}{2} <---- NO FUNCIONA
		%\linespread{2} <------ NO FUNCIONA
		% Otras referencias web: Paquete ``doublespace'' y paquete ``setspace.sty''
		\setlength{\baselineskip}{1.2\baselineskip}
		%Ahora mismo esta puesta a 1.0, por lo que no hay aumento de espacio en cuanto al
		%que viene por defecto. Con 1.2 se consigue algo parecido a 1.5 en documentos word
		% --------------------------------------------------------------------------------


		%  -------------------------------------------------------------------------------
		% Capítulos
		\chapter{Introducción a WEKA}
% \begin{quotation}
% 	\begin{small}
%    \textit{Siempre que enseñes, enseña a la vez a dudar lo que enseñes.}
% 	\end{small}
% 	\begin{flushright}José Ortega y Gasset.\end{flushright}
% \end{quotation}



\newpage
\section{Filtros no supervisados}
%Son filtros que no tienen en cuenta la variable marca de clase.
	

	\subsection{Normalize}

	\begin{justify}
	Realiza una normalización de todos los valores numéricos en el conjunto de datos. 
	\end{justify}

	\begin{justify}
	\textbf{Uso}

	Los valores son modificados al rango [0,1], tomando el valor 0 el dato más pequeño del conjunto y tomando el valor 1 el dato mayor del mismo, quedando el resto de valores en valores continuos dentro del rango.
	\end{justify}
	
	\begin{justify}

	\textbf{Ejemplo}

	Como se observa en la figura \ref{fig:normalize1}, se tiene un conjunto de atributos originales a los cuales se les aplicará el filtro. Una vez filtrados, se puede observar en la figura \ref{fig:normalize2} cómo estos han quedado en un rango 0-1.
	\end{justify}

	\begin{figure}[!htp]
	\centering
	\includegraphics[scale=.32]{./figuras/image22.png}
	\caption{Base de datos 'wine'}
	\label{fig:normalize1}
	\end{figure}
	
	\begin{figure}[!htp]
	\centering
	\includegraphics[scale=.32]{./figuras/image21.png}
	\caption{Base de datos 'wine' tras aplicar Normalize}
	\label{fig:normalize2}
	\end{figure}

	\subsection{ReplaceMissingValues}
	Reemplaza todo valor perdido, los cuales son representados con el signo '?', de los atributos nominales y numéricos.

	\begin{justify}
	\textbf{Uso} 
	\end{justify}

	Busca aquellos valores perdidos y los reemplaza con los valores de las modas y medias para dicho atributo.

	Para ilustrar el uso de este filtro se ha usado una pequeña base de datos que contiene notas de 3 asignaturas para varias instancias que son los alumnos, y se observa como los datos perdidos son reemplazados.

	
	\begin{justify}
	\textbf{Ejemplo}
	\end{justify}

	\begin{figure}[!htp]

	\centering
	  \subfloat[Base de datos 'notas' con valores perdidos]{
		\includegraphics[scale=.32]{./figuras/image19.png}}
	  \subfloat[Media tras aplicar el filtro]{
		\includegraphics[scale=.32]{./figuras/image10.png}}
	\caption{Base de datos y media calculada tras aplicar ReplaceMissingValues}
	\end{figure}
  
	\begin{figure}[!htp]
	\centering
	\includegraphics[scale=.32]{./figuras/image31.png}
	\caption{Fichero final con el valor perdido reemplazado}
	\end{figure}

\newpage
\subsection{RemoveUseless}
	\begin{justify}
		Elimina atributos nominales donde la varianza es muy grande o muy pequeña y que por tanto, no tienen utilidad.
	\end{justify}



	\begin{justify}
	\textbf{Uso} 
	\end{justify}
	Realiza el análisis y transformación de los componentes principales de los datos. Se usa junto con una búsqueda de Ranker. La reducción de la dimensionalidad se logra eligiendo suficientes vectores propios para tener en cuenta algún porcentaje de la varianza en los datos originales, por defecto 0.95.

	\begin{justify}
	\textbf{Ejemplo}
	\end{justify}
	Parar ilustrar este filtro se ha usado una pequeña base de datos que contiene un atributo nominal que representa un color, y 2 atributos numéricos que representan otros datos asociados. Se puede observar como el atributo nominal tiene un valor distinto para cada instancia, y por tanto, la varianza es la máxima, así pues el filtro actúa descartando dicho atributo nominal.
  

	\begin{figure}[!htp]
	\centering
	  \subfloat[Base de datos 'colores' con atributo nominal]{
		\includegraphics[scale=.45]{./figuras/image2.png}}
	  \subfloat[Fichero tras aplicar filtro]{
		\includegraphics[scale=.5]{./figuras/image40.png}}
	\caption{Antes y después de aplicar RemoveUseless}
	\end{figure}




\newpage

	\subsection{PrincipalComponents}

	Realiza el análisis y transformación de las componentes principales de los datos.


	\begin{justify}
	\textbf{Uso} 
	\end{justify}

	La reducción de la dimensionalidad se logra eligiendo suficientes vectores propios para tener en cuenta algún porcentaje de la varianza en los datos originales (por defecto 0.95). El ruido de los atributos puede filtrarse transformándolo en el espacio de la Componente Principal, eliminando algunos de los vectores propios peores, y luego transformando de nuevo al espacio original.

	\begin{justify}
	\textbf{Ejemplo}
	\end{justify}

	
	\begin{figure}[!htp]
	\centering
	\includegraphics[scale=.42]{./figuras/image2.png}
	\caption{Base de datos 'colores'}
	\end{figure}
	
	\begin{figure}[!htp]
	\centering
	\includegraphics[scale=.42]{./figuras/image12.png}
	\caption{Fichero final tras aplicar PrincipalComponents}
	\end{figure}


\newpage

	\subsection{RandomProjection}
		Reduce la dimensionalidad de los datos proyectándolos en un subespacio de menor dimensión usando una matriz aleatoria con columnas de longitud unitaria.
	

	\begin{justify}
	\textbf{Uso} 
	\end{justify}
	
		Primero aplica el filtro NominalToBinary para convertir todos los atributos a numérico antes de reducir la dimensión. Conserva el atributo de marca de clase.

	\begin{justify}
	\textbf{Ejemplo}
	\end{justify}

		Como se muestra en la Figura 1.8 partimos de una base de datos con un atributo nominal de 3 posibles opciones 'Blanco', 'Rojo' y 'Azul' junto con un atributo numérico x1.
		Una vez aplicado el filtro, el primer atributo queda establecido en una matriz numérica de un subespacio menor. 

	\begin{figure}[!htp]
	\centering
	\includegraphics[scale=.42]{./figuras/image11.png}
	\caption{Base de datos 'colores' con un atributo numérico}
	\end{figure}
	
	\begin{figure}[!htp]
	\centering
	\includegraphics[scale=.42]{./figuras/image44.png}
	\caption{Fichero final tras aplicar RandomProjection}
	\end{figure}


\newpage
	\subsection{NominalToBinary}
	Convierte todos los atributos nominales en atributos binarios numéricos. 	

	\begin{justify}
	\textbf{Uso} 
	\end{justify}
	
	Un atributo con k posibles valores se transforma en k atributos binarios (0-1) si la clase es nominal. Los atributos binarios se dejan binarios. Si la clase es numérica, es posible que desee utilizar la versión supervisada de este filtro.

	\begin{justify}
	\textbf{Ejemplo}
	\end{justify}






\newpage
	\subsection{RemoveMissclassified}
	\begin{justify}
	Elimina aquellas instancias que han sido incorrectamente clasificadas, de modo que no existan valores atípicos.
	\end{justify}
	\begin{itemize}
		\item \textbf{Uso}
	\begin{justify}
	Permite escoger la marca de clase en la que se basan las clasificaciones erróneas, el clasificador 	sobre el que se basarán las clasificaciones erróneas, si el resultado será descartartado o aceptado, número de iteraciones, pliegues y umbral de error permisible.
	\end{justify}
		\item \textbf{Ejemplo}
	\end{itemize}

	\subsection{RemovePercentage}
	\begin{justify}
	Permite eliminar un porcentaje de la información de la base de datos.
	\end{justify}
	\begin{itemize}
		\item \textbf{Uso}
	\begin{justify}

	\end{justify}
		\item \textbf{Ejemplo}
	\end{itemize}

	\subsection{Resample}
	\begin{justify}
	Produce una submuestra aleatoria de un conjunto de datos utilizando el muestreo con reemplazo o sin reemplazo. 
	\end{justify}
	\begin{itemize}
		\item \textbf{Uso}
	\begin{justify}
	Se puede especificar el número de instancias en el conjunto de datos generado. Cuando se utilizan en modo por lotes, los lotes posteriores no son remuestrados.
	\end{justify}
		\item \textbf{Ejemplo}
	\end{itemize}

\newpage
\section{Filtros supervisados}
	%Son filtros que tienen en cuenta la variable marca de clase.	
	\subsection{AttributeSelection}
	\begin{justify}

	\end{justify}
	\begin{itemize}
		\item \textbf{Uso}
	\begin{justify}

	\end{justify}
		\item \textbf{Ejemplo}
	\end{itemize}
	
	\subsection{Discretize}
	\begin{justify}

	\end{justify}
	\begin{itemize}
		\item \textbf{Uso} 
	\begin{justify}

	\end{justify}
		\item \textbf{Ejemplo}
	\end{itemize}
 	
	\subsection{NominalToBinary}
	\begin{justify}

	\end{justify}
	\begin{itemize}
		\item \textbf{Uso}
	\begin{justify}

	\end{justify}
		\item \textbf{Ejemplo}
	\end{itemize}
	
	\subsection{SpreadToBinary}
	\begin{justify}

	\end{justify}
	\begin{itemize}
		\item \textbf{Uso}
	\begin{justify}

	\end{justify}
		\item \textbf{Ejemplo}
	\end{itemize}
 	
	
	\subsection{ClassBalancer}
	\begin{justify}

	\end{justify}
	\begin{itemize}
		\item \textbf{Uso}
	\begin{justify}

	\end{justify}
		\item \textbf{Ejemplo}
	\end{itemize}

	\subsection{Resampler}
	\begin{justify}

	\end{justify}
	\begin{itemize}
		\item \textbf{Uso}
	\begin{justify}

	\end{justify}
		\item \textbf{Ejemplo}
	\end{itemize}


\newpage
\section{Base de datos Wine}
	%\noindent Los objetivos principales que se persiguen en esta tesis doctoral son los siguientes:
 	
	\subsection{Detalles de la base de datos}
	
	\subsection{Modificación del fichero con nombres de atributo descriptivos}

	\subsection{Descripción de atributos y clases}

	\subsection{Tratamiento de elementos perdidos}

	\subsection{Diferencia entre Distinct y Unique}

	\subsection{Eliminar atributos identificadores}
	
	\subsection{Relaciones visualmente significativas en entorno Visualice}
	

\newpage
\section{Aplicación de 6 filtros a la base de datos}
	%\noindent Los objetivos principales que se persiguen en esta tesis doctoral son los siguientes:
 	
	\subsection{Filtro de selección de características}
	\begin{itemize}
		\item \textbf{Uso}
		\item Resultados:
		\item \textbf{Ejemplo}
	\end{itemize}

	\subsection{Filtro de selección de patrones}
	\begin{itemize}
		\item \textbf{Uso}
		\item Resultados:
		\item \textbf{Ejemplo}
	\end{itemize}

	\subsection{Filtro de filters/supervised/attribute/*}
	\begin{itemize}
		\item \textbf{Uso}
		\item Resultados:
		\item \textbf{Ejemplo}
	\end{itemize}

	\subsection{Filtro de filters/supervised/instance/*}
	\begin{itemize}
		\item \textbf{Uso}
		\item Resultados:
		\item \textbf{Ejemplo}
	\end{itemize}

	\subsection{Filtro de filters/unsupervised/attribute/*}
	\begin{itemize}
		\item \textbf{Uso}
		\item Resultados:
		\item \textbf{Ejemplo}
	\end{itemize}

	\subsection{Filtro de filters/unsupervised/instance/*}
	\begin{itemize}
		\item \textbf{Uso}
		\item Resultados:
		\item \textbf{Ejemplo}
	\end{itemize}



\newpage
\section{Conversión de todo atributo nominal a codificación binaria}


\newpage
\section{División de la base de datos en particiones}

	\subsection{División en 10-Holdout 75-25}
	\begin{itemize}
		\item \textbf{Uso}
		\item Resultados:
		\item \textbf{Ejemplo}
	\end{itemize}


	\subsection{División en 10-Fold}
	\begin{itemize}
		\item \textbf{Uso}
		\item Resultados:
		\item \textbf{Ejemplo}
	\end{itemize}

% 	A continuación y una vez descritos los objetivos de esta tesis doctoral, se exponen
% 	las aportaciones fundamentales incluidas en esta memoria:
% 	\begin{enumerate}
% 	\item Análisis y descripción de los aspectos más importantes en el uso de algoritmos
% 	evolutivos mono-objetivo y multi-objetivo híbridos, funciones de aptitud, DE y
% ensembles para el diseño de modelos de red para clasificación de
% 	patrones.
% 	\item Implementación de varias versiones de un AE memético mono-objetivo usando
% 	modelos de red puros e híbridos en capa oculta. Los resultados obtenidos están en
% 	consonancia con los	de otras	técnicas	frecuentemente citadas en Aprendizaje
% Automático (\textit{Machine Learning}, ML).
% 	\item Implementación de un MOEA	memético basado en dominancia de Pareto para diseñar
% 	modelos de	red para clasificación	de patrones.
% 	\item Implementación de varias versiones de un MOEA memético basado
% 	en dominancia de	Pareto y en la DE.
% 	\item Futura obtención de	padres virtuales en DE basados en
% 	los valores extremos y mejores individuos de la 	población.
% 	\item Estudio y análisis del uso de la mínima sensibilidad en la	clasificación de
% 	patrones	para problemas multiclase, como alternativa al uso de	curvas ROC y otras
% 	medidas
% 	de	rendimiento. Utilización de dicha función de aptitud en los algoritmos
% 	desarrollados obteniéndose resultados satisfactorios.
% 	\item Nueva representación en dos dimensiones de la bondad de un clasificador
% 	mediante la optimización simultánea de la precisión y la mínima sensibilidad.
% 	\item Aplicación de los algoritmos implementados en la resolución de problemas de
% 	clasificación reales de microbiología predictiva, teledetección y predicción de viento.
% % 	\item Inclusión los algoritmos implementados en KEEL,
% % 	ampliando de esta manera el abanico de técnicas de las que dispone la herramienta,
% % 	proporcionándole una mayor riqueza para labores científicas.
% 	\end{enumerate}

		\paginavaciacompleta
		\chapter{Clasificación y Regresión en WEKA}
\
\section{Algoritmo IB1 con 10-fold crossvalidation}
	\subsection{Visualización de la clasificación}
	\subsection{Interpretación de los resultados}

\newpage
\section{Algoritmo IBK (k=1, k=3, k=5) con 10-fold crossvalidation a 1}
	\subsection{Cálculo de media y desviación típica de las medidas}

	\begin{itemize}
		\item Accuracy:
		\item Kappa:
		\item RMSE:
		\item F-Measure:
		\item Media ponderada AUC:
	\end{itemize}

	\subsection{Visualización de la clasificación}
	\subsection{Interpretación de los resultados}
 
\newpage
\section{Base de datos house.arff}
	\subsection{Carga de la base de datos}
	
	\subsection{Variable Granite}
	\begin{itemize}
		\item Influencia en el modelo
		\item Conclusiones
	\end{itemize}
	
	\subsection{Variable binaria Bathroom}
	\begin{itemize}
		\item Influencia en el modelo
		\item Aplicación de algoritmo de regresión lineal
		\item Conclusiones
	\end{itemize}

	\subsection{Variable Bedrooms}
	\begin{itemize}
		\item Influencia en el modelo
		\item Conclusiones
	\end{itemize}

	\subsection{Variable HouseSize}
	\begin{itemize}
		\item Influencia en el modelo
		\item Aportación al modelo de regresión lineal
		\item Conclusiones
	\end{itemize}

\newpage
\section{Base de datos autoMpg.arff}
	
	\subsection{Aplicación del algoritmo LinearRegression con un 80-20}
	\subsection{Resultados obtenidos}
	\begin{itemize}
		\item Conclusiones
		\item Tablas comparativas
	\end{itemize}

	\subsection{Atributo que aporta más información de la variable dependiente}


	\subsection{Atributo que no aporta información al modelo}


	\subsection{Errores cometidos}
	\begin{itemize}
		\item Visualización
		\item Representación de las diferentes cruces
	\end{itemize}


	\subsection{Modificación del parametro attributteSelectionMethod}
	\begin{itemize}
		\item Conclusión en el nuevo modelo
		\item Análisis de menor peso de "weight" frente "acceleration"
	\end{itemize}

\newpage
\section{Método SimpleLinearRegression en base de datos autoMpg.arff}

	\begin{itemize}
		\item Respuesta
		\item Solución
		\item Visualización
		\item Comentario de resultados
	\end{itemize}

\newpage
\section{Algoritmos Logistic y SimpleLogistic}

	
	\subsection{Análisis}
		\begin{itemize}
			\item Conclusión
			\item Métricas
			\item Variables más influyentes (beta)
			\item Variables no usadas
			\item Visualización
			\item Asociación de fórmulas con modelos obtenidos según el algoritmo
		\end{itemize}

	\subsection{Visualización gráfica de errores cometidos}

	\subsection{Análisis de parámetros}

		\begin{itemize}
			\item maxBoostingIterations
				\begin{itemize}
					\item Análisis
					\item Modificación
				\end{itemize}

			\item heuristicStop
				\begin{itemize}
					\item Análisis
					\item Modificación
				\end{itemize}
		\end{itemize}
		%\paginavaciacompleta
		%\paginavaciasincuerpo
		%\chapter{Rendimiento en problemas de clasificación}\label{medidasRendimiento}
% \begin{quotation}
% 	\begin{small}
% 	\textit{El afán de perfección hace a algunas personas totalmente insoportables.}
% 	\end{small}
% 	\begin{flushright}Pearl S. Buck.\end{flushright}
% \end{quotation}

\section{Métricas de rendimiento}\label{2.1}
\noindent Uno de los problemas fundamentales en aprendizaje automático, (\textit{Machine
Learning},
ML), es la
clasificación en dos o más clases de un conjunto de ejemplos no conocido (conjunto de
generalización), en base al aprendizaje de un número de ejemplos
cuya	clase	si es conocida (conjunto de entrenamiento).

La evaluación del rendimiento es algo decisivo para obtener una medida del
rendimiento de un clasificador, en cuanto a los conjuntos de entrenamiento y
generalización. En ocasiones el proceso de diseño u obtención de un clasificador conlleva
una serie de etapas que implican un proceso iterativo, donde cada	iteración, puede alterar
considerablemente el clasificador que se está diseñando. Se requiere, por tanto, una
re-evaluación del clasificador en cada iteración para determinar cuál ha sido el	impacto
producido en su rendimiento, premiando, generalmente, aquellos cambios que lo han hecho
mejorar.

En la literatura existen diversas medidas para determinar el
rendimiento de un clasificador \cite{Duda2000,Caruana2004,Sokolova2009}, pero
antes de pasar a describir algunas de las más utilizadas, es conveniente definir lo que se
denomina matriz de contingencia o matriz de confusión de un clasificador.

Dado un problema de
clasificación multiclase con $Q$ clases, siendo $Q\geq2$, y $N$ patrones de entrenamiento
o generalización, la matriz de contingencia $M(g)$, de dimensión $Q\text{x}Q$, de un
clasificador $g$, está dada por:
\begin{displaymath}
\begin{tabular}{ccc}
&Clase Predicha & \\
$\mathbf{M(g)}=$ &$\left( \begin{array}{cccc}
n_{11} & n_{12} & \ldots & n_{1Q} \\
n_{21} & n_{22} & \ldots & n_{2Q} \\
\ldots & \ldots & n_{ii} & \ldots \\
n_{Q1} & n_{Q2} & \ldots & n_{QQ} \\
\end{array} \right)$ & \begin{sideways}\hspace{-0.8cm}Clase
Real\end{sideways} \\
\end{tabular}
\end{displaymath}
donde las filas indican la clase real de pertenencia y las columnas la pronosticada
por el clasificador.

Formalmente,  la matriz de confusión, $M(g)$, se puede definir como:
\begin{displaymath}
	 \mathbf{M(g)}=\left\{n_{ij};\sum_{i,j=1}^Q n_{ij}=N\right\}
\end{displaymath}
donde $n_{ij}$ representa el número de patrones asignados a la clase $i$, cuando
realmente pertenecen a la clase $j$. La diagonal corresponde a los patrones correctamente
clasificados para las $Q$ clases del problema, y los   $Q(Q-1)$	elementos fuera de la
diagonal principal corresponden a los errores de clasificación. En consecuencia, los
totales por fila indican
el número de patrones pertenecientes a cada una de las clases, y la suma de estos totales
es el tamaño de la muestra.

Si analizamos la matriz de confusión:
\begin{displaymath}
\left( \begin{array}{cccc}
n_{11} & \ldots & \ldots & n_{1Q} \\
\ldots & \ldots & n_{ij} & \ldots \\
\ldots & \ldots & \ldots & \ldots \\
n_{Q1} & \ldots & \ldots & n_{QQ} \\
\end{array}\right)
\end{displaymath}
,tenemos que:
\begin{eqnarray}
n_{i\circ} & = & \sum_{j=1}^Q n_{ij} \quad i=1,...,Q \nonumber \\
n_{\circ j} & = & \sum_{i=1}^Q n_{ij} \quad j=1,...,Q \nonumber
\end{eqnarray}
siendo $n_{i\circ}$ el total de patrones asociados la clase $i$ (suma por filas), y
$n_{\circ j}$ el número
de patrones predichos por el clasificador que se han clasificado como pertenecientes a la
clase $j$ (suma por columnas).

A partir de las expresiones anteriores, se puede deducir que $\displaystyle
\sum_{i=1}^Q n_{i\circ}=N$ y que $\displaystyle \sum_{j=1}^Q n_{\circ j}=N$. Por tanto,
las frecuencias relativas correspondientes al número de patrones de la clase $i$ sobre el
total de patrones viene dada por:
\begin{displaymath}
f_{i \circ}= \frac{n_{i \circ}}{N}
\end{displaymath}
Y la frecuencia relativa correspondiente al número de patrones que el clasificador ha
clasificado como clase $j$, con respecto al total de patrones viene dada por:
\begin{displaymath}
f_{\circ j}= \frac{n_{\circ j}}{N}
\end{displaymath}

En el caso de clasificación binaria (una clase positiva y una clase negativa), es decir,
con un valor de $Q=2$, la matriz de confusión está dada por:
\begin{displaymath}
\mathbf{M(g)} =
\left( \begin{array}{cc}
tp & fn\\
fp & tn\\
\end{array} \right)
\end{displaymath}
donde:
\begin{itemize}
\item $tp$ significa ``verdaderos positivos'', o número de elementos que son de
la clase positiva y que el clasificador ha clasificado como positivos.
\item $fn$ significa ``falsos negativos'', o número de elementos que son de la clase
positiva y que el clasificador a clasificado como negativos.
\item $fp$ significa ``falsos positivos'', o número de elementos de la clase negativa
que son clasificados como positivos.
\item $tn$ significa ``verdaderos negativos'', o número de elementos de la clase
negativa que son clasificados como negativos.
\end{itemize}

% Independientemente de que las medidas de rendimiento sean escalares, basadas en umbral,
%en
% ranking o en probabilidades, se podría diferenciar entre aquellas se utilizan solo para
% problemas binarios y aquellas que se pueden utilizar tanto en binarios como en
% multiclase.
%
% Por norma general, ninguna métrica es siempre mejor que otra a la hora de medir la
% bondad de un clasificador, ya que en muchas ocasiones depende del conjunto de
% patrones a clasificar el que una métrica nos proporcione un valor de precisión más o
% menos real sobre la capacidad que tiene un determinado clasificador. Además hay
%problemas
% en los que interesa maximizar o minimizar una medida dependiendo de nuestros
% intereses. Por ejemplo, no es lo mismo diagnosticar cáncer sobre una base de datos
% médica a un 20\% de pacientes que en realizadas no lo tienen que diagnosticar
% una carencia de cáncer a un 20\% de pacientes que en realidad si lo tienen. La
%siguientes
% secciones muestran una breve descripción de las medidas más comúnmente utilizadas.
% (¿¿¿NOS PUEDEN CRITICAR QUE NO DIGAMOS EN QUÉ SITUACIONES UTILIZAR UNA MÉTRICA Y EN
%CUÁLES
% OTRA Y PORQUÉ????)

\subsection{Métricas para problemas binarios}\label{metricasbinarios}
\noindent A continuación exponemos las métricas para problemas de clasificación binarios
que más se utilizan a la hora de obtener el rendimiento de un clasificador:
\begin{description}
\item[Precisión, C:] Efectividad global de un
clasificador o	 porcentaje	de patrones totales correctamente
clasificados. Es una medida que suele proporcionarse en tanto por
ciento.
\begin{displaymath}
C=\frac{tp+tn}{tp+fn+fp+tn}=\frac{tp+tn}{N}
\end{displaymath}
\item[Precisión positiva, P:] Porcentaje de patrones correctamente clasificados
de la clase positiva con respecto a todos los elementos que el clasificador predijo como
positivos. Dicho de otra forma sería el porcentaje de ejemplos que el clasificador ha
predicho como positivos y que realmente son positivos.
\begin{displaymath}
P=\frac{tp}{tp+fp}
\end{displaymath}
\item[Sensibilidad, TPR:] También se nombra por \textit{Recall} o
\textit{True Positive Rate} (TPR). Es el porcentaje de patrones correctamente
clasificados de la clase positiva con respecto al número total	de	elementos existentes de
esa clase, o también
se puede decir que es la efectividad del clasificador para identificar los
elementos de la clase positiva.
\begin{displaymath}
TPR=\frac{tp}{tp+fn}
\end{displaymath}
\item[Especificidad, Sp:] Porcentaje de patrones correctamente
clasificados de la
clase negativa con respecto al número total de elementos existentes de esa clase, o
también se puede decir que es la	efectividad del clasificador para identificar los
elementos de la clase negativa.
\begin{displaymath}
Sp=\frac{tn}{fp+tn}=1-FPR
\end{displaymath}
donde $FPR$ se define a continuación.
\item[Porcentaje de falsos positivos, FPR:] También se nombra por \textit{False Positive Rate}
(FPR), y se define como el porcentaje de
patrones incorrectamente clasificados de la clase negativa con respecto al número
total de elementos existentes de esa clase.
\begin{displaymath}
FPR=\frac{fp}{fp+tn}=1-Sp
\end{displaymath}
\item[\textit{Fscore}:] Es una medida que se basa en $P$ y $TPR$, y se puede
interpretar como una media ponderada de ambas. Es la relación entre los elementos
positivos y aquellos dados por el clasificador.
\begin{eqnarray}
Fscore&=&\frac{2}{\frac{1}{P}+\frac{1}{TPR}} =
2\cdot \frac{P\cdot TPR}{
	P+TPR}
% 	= \nonumber \\
% &=& \frac{(\beta^2+1)tp}{(\beta^2+1)tp+\beta^2fn+fp}
\nonumber
\end{eqnarray}
donde $\beta$ es un número real no negativo, en el caso de esta igualdad $\beta=1$.
%\item[\textit{F-measure:}] La medida F es (BUSCAR DESCRIPCION...)
%\begin{displaymath}
%F-measure=\frac{2}{\frac{1}{Precisión}+\frac{1}{Sensitivity}}
%\end{displaymath}
\item[Area bajo la curva, AUC:]
Capacidad del
clasificador para evitar una clasificación falsa, teniendo en cuenta tanto a la clase
positiva como a la negativa. Esta medida es una aproximación del área bajo una curva
ROC, y se puede ver como una transformación lineal del índice de Youden
\cite{Youden1950}. Se puede calcular de manera más precisa mediante el ``Algoritmo 2``
mostrado en \cite{Fawcett2006}. Concretamente el $AUC$ es una porción del área de un
cuadrado de lado la unidad,
estando su valor entre 0 y 1. Una aproximación al $AUC$ es la siguiente:
% Se puede demostrar que el área bajo la curva ROC es
% equivalente a la prueba de Mann-Whitney, una prueba no paramétrica aplicada a dos
%muestras
% independientes, cuyos datos han sido medidos al menos en una escala de nivel ordinal;
% también equivalente a la prueba de los signos de Wilcoxon.
% ESTA MEDIDA ESTA SACADA DEL PAPER ''a SYSTEMATIC ANALISYS OF PERFOMANCE
% FOR CLASSIFICATIONS TASK``, DICE QUE ES BINARIA Y YO HE SUPUESTO QUE SE REFIERE AL AUC
%DE
% LA CURVA ROC. POR FAVOR REVISADMELO. pONE QUE A VECES A ESTA MEDIDA SE LE LLAMA Balanced
% Accuracy (MIRAR FINAL DE LA PAGINA 429 DEL ARTICULO)
\begin{displaymath}
AUC=\frac{1}{2}\left(TPR+Sp\right)
\end{displaymath}
\item[Media geométrica, GM:] La media geométrica intenta maximizar
la precisión en las dos clases que componen un determinado problema de la forma más
balanceada posible. Aunque esta medida se puede extender para problemas multiclase su
utilización no es usual.
\begin{displaymath}
GM=\sqrt{TPR\cdot Sp}
\end{displaymath}
\end{description}

\subsection{Métricas para problemas multiclase}\label{metricasmulticlase}
\noindent A partir de la matriz de confusión multiclase mostrada en la sección \ref{2.1},
la sensibilidad de una clase $i$ en problemas multiclase, se define como, el número
de patrones correctamente predichos en esa clase con respecto al número total de patrones
de dicha clase (probabilidad de predecir correctamente un ejemplo de la clase $i$), y la
denotaremos por:
\begin{displaymath}
S_{i}=\frac{n_{ij}}{n_{i\circ}}, \qquad i=1,...,Q.
\end{displaymath}

La especificidad, $Sp$, de una clase $i$ en problemas multiclase, se define como, el número
de patrones correctamente predichos en esa clase con respecto al número total de patrones
predichos por el clasificador para esa clase. Hay que hacer notar que para problemas
binarios la
especificidad se refiere a la clase negativa, según está definida en la sección
\ref{metricasbinarios}:
\begin{displaymath}
Sp_{i}=\frac{n_{ii}}{n_{\circ j}}, \qquad j=1,...,Q.
\end{displaymath}

Teniendo en cuenta dicha matriz y las definiciones anteriores, las métricas más utilizadas
para problemas multiclase son las siguientes:
\begin{description}
	\item[Precisión, C:]  Al igual que en el caso de problemas binarios
	es el porcentaje de patrones correctamente clasificados. Es una medida que suele
	proporcionarse en tanto por ciento:
	\begin{displaymath}
	C=\left( \frac{1}{N}\right) \sum_{j=1}^Q n_{jj}
	\end{displaymath}
	Equivalentemente, $C$ puede expresarse como media ponderada de las
	sensibilidades:
	\begin{equation}\label{CenbaseS}
	C=\sum_{i=1}^Q \frac{n_{i\circ}}{N} S_{i}, \qquad i=1,...,Q\text{,}
	\end{equation}
	donde los pesos corresponden a las probabilidades "a priori" de cada clase.
	\item[Error cuadrático medio, MSE:] El error cuadrático
	medio es una medida que
	proporciona un promedio del error cometido por un clasificador entre los valores
	predichos, y los reales u observados. Los valores que puede tomar esta medida están
	entre 0 e $\infty$, y normalmente se utiliza en problemas de regresión, aunque hay
	autores que también la usan para clasificación \cite{Abbass2001,Abbass2002a}.
	\begin{displaymath}
	MSE=\left(\frac{1}{N}\right)\sum_{i=1}^N\left(\hat{Y}_{i}-Y_{i}\right)^2
	\end{displaymath}
	siendo $\hat{Y}_{i}$ la etiqueta estimada para el patrón $i$, e $Y_{i}$ la etiqueta real
	del	patrón $i$.

	En cuanto a los valores que puede tomar la variable $\hat{Y}_{i}$ y
	la variable $Y_{i}$, tanto en esta métrica como en las siguientes que se
	exponen dentro de esta sección, dependen de la interpretación que se haga de la/s
	salida/s que proporciona el clasificador que	estemos evaluando, y de los valores
	tomados como	verdaderos. Algunos ejemplos posibles de interpretación, aplicados en
	este	caso a ANNs, podrían ser los	siguientes: Una red
	neuronal	con varias salidas interpretadas de manera probabilística, donde el valor de
	cada salida indica la probabilidad de pertenencia a una clase determinada. Otra
	interpretación posible sería tener una red neuronal con una sola salida, comprendida
	entre 0 y 1, donde la pertenencia a una determinada clase se encuentra en un cierto
	rango	de esa salida. O incluso una red con varias salidas, donde cada una de ellas
	indica 0 o 1, dependiendo de si un patrón pertenece o no a una clase determinada. Por tanto,
	habrá
	que adaptar los valores $\hat{Y}_{i}$ y los valores $\hat{Y}_{i}$, de forma que se
	pueda aplicar la correspondiente métrica.
	\item[Raíz del error cuadrático medio, RMSE:] La raíz
	cuadrada	del error cuadrático
	medio, también proporciona un promedio del error cometido por un clasificador. Sus
	valores también están entre 0 e $\infty$.
	\begin{displaymath}
	RMSE=\sqrt{\left(\frac{1}{N}\right)\sum_{i=1}^N\left(\hat{Y}_{i}
	-Y_{i}\right)^2}
	\end{displaymath}
	\item[Entropía cruzada, E:] La entropía cruzada, es una función
	de error basada en
	las probabilidades de pertenencia de cada uno de los patrones de un conjunto de datos a
	cada una de las clases que componen un determinado problema. Los valores que puede
	tomar esta medida están	entre 0 e $\infty$.
	\begin{displaymath}
	E=-\left(\frac{1}{N}\right)\sum_{n=1}^N\sum_{l=1}^Qy_{n}^{(l)}\log
	g_{l}^{}(\mathbf{x}_{n})
	\end{displaymath}
	donde $y_{n}^{(l)}$ es igual a 1 si el patrón $n$ pertenece a la clase $l$, y 0 en caso
	contrario, y donde  $\displaystyle g_{l}^{}(\mathbf{x}_{n})$ es la probabilidad de que
	el patrón $n$ 	pertenezca a la clase $l$.
% 	\item[\textit{SEP (Standar Error Prediction):}] El error estándar de predicción es
% 	es una medida adimensional y se define como la	desviación típica de las diferencias
% 	entre los valores predichos y los valores reales. Al igual que el \textit{MSE} suele
% 	utilizarse en problemas de regresión.
% 	\begin{displaymath}
% 	SEP=\left( \frac{100}{\overline{Predicho}}\right)
% 	\sqrt{\left(\frac{1}{N}\right)\sum_{i=1}^N\left(Predicho(g)-Verdadero(g)\right)^2}
% 	\end{displaymath}
% 	donde $\overline{Predicho}$ es el valor medio de todas las salidas predichas por el
% 	clasificador.
% 	\item[Coeficiente de información mutua, IC:]
% 	Define la información aportada
% 	por una variable aleatoria sobre otra, o lo que es lo mismo, cuánto reduce la
% 	incertidumbre sobre una variable el conocimiento de la otra \cite{Baldi2000}. La
% 	expresión que se muestra a continuación se obtiene asumiendo que $\displaystyle
% 	H(X)=-\sum_{i}^{Q}X_{i}\log X_{i}$,
% 	y que $\displaystyle f=\frac{n_{i\circ}}{N}$, $\displaystyle k=\frac{n_{\circ j}}{N}$ y
% 	$\displaystyle n=\frac{n_{ij}}{N}$:
% 	\begin{eqnarray}
% 	IC&=&\frac{H(f)+H(k)-H(N)}{H(f)}=\nonumber \\
% 	&=&\frac{-\sum_{i=1}^Q\left(\frac{n_{i\circ}}{N}\log
% 	\frac{n_{i\circ}}{N}\right) -\sum_{i=1}^Q \left( \frac{n_{\circ j}}{N}\log
% 	\frac{n_{\circ j}}{N}\right) +\sum_{i,j=1}^Q \left( \frac{n_{ij}}{N}\log
% 	\frac{n_{ij}}{N}\right)}{-\sum_{i=1}^Q\left( \frac{n_{i\circ}}{N}\log
% 	\frac{n_{i\circ}}{N}\right)} \nonumber
% 	\end{eqnarray}
	\item[\textit{Generalized Squared Correlation}, GC$^{2}$:] La correlación
	cuadrática generalizada se puede considerar como una generalización para problemas
	multiclase del coeficiente de correlación de Matthews \cite{Matthew1975} para dos
	clases.
	\begin{displaymath}
	GC^2=\frac{1}{N\left(Q-1\right)}\sum_{i,j=1}^Q\frac{\left(
	n_{ij}-e_{ij}\right)^2}{e_{ij}}
	\end{displaymath}
	donde $\displaystyle e_{ij}=\frac{n_{i\circ}n_{\circ j}}{N}$	es el número esperado de
patrones en
	la posición	$ij$ de la matriz de confusión, teniendo como hipótesis que las asignaciones
	y	las	predicciones son independientes.
	\item[\textit{Macro-Average}, MAVG:] La macro-media se define como la media de las
	sensibilidades de cada clase, sin considerar la desviación típica. Da la misma
	importancia o peso a cada una de las clases de un problema.
	\begin{displaymath}
	MAVG=\frac{1}{Q}\sum_{i=1}^Q S_{i}
	\end{displaymath}
\end{description}
\newpage
\section{Curvas ROC}\label{curvasROC}
\noindent Cabe comentar por separado una de las técnicas más usadas habitualmente para
la comparación del rendimiento mostrado por dos o más clasificadores binarios, las
llamadas curvas
ROC \cite{Fawcett2006}. Dichas curvas son	una alternativa al
uso de la		precisión	y sus		problemas derivados \cite{Provost1997,Provost1998}, y
es	una buena	técnica para		comprobar si un
clasificador es mejor que	otro	en	términos de	la clase	minoritaria. Las curvas ROC son
gráficas en dos dimensiones, en la que se representan los errores de clasificación de la
clase negativa o $FPR$ en el eje horizontal, y la precisión de la clase positiva o
$TPR$ en el eje vertical. Las curvas ROC muestran
toda la información relacionada con el rendimiento de un clasificador, y permiten una
rápida visualización del tipo de relación entre los rendimientos de varios
clasificadores. El análisis de la curva ROC, o simplemente análisis ROC, proporciona
información para seleccionar los modelos posiblemente óptimos, y es también independiente
de la distribución de las clases en la población. El análisis ROC se relaciona de forma
directa con el análisis de coste-beneficio en toma de decisiones en diagnóstico médico.

Dentro de una gráfica ROC, un clasificador domina a otro mientras más arriba y más a la
izquierda se encuentre (ver figura \ref{ejemploROC}). El mejor método posible de
predicción se situaría en un punto en la esquina superior izquierda, o coordenada $(0,1)$
del plano ROC, representando un 100\% de sensibilidad (ningún falso negativo) y un 100\%  de
especificidad (ningún falso positivo). Este punto $(0,1)$ se denomina
clasificación perfecta. Por el contrario, una clasificación totalmente aleatoria daría
un punto a lo largo de la línea diagonal (línea de no-discriminación),
desde el extremo inferior izquierdo hasta el extremo superior derecho.

\begin{figure*}[htb]
\centering
\subfloat[Curva ROC]{\label{ejemploROC1}\includegraphics[width=.50
\textwidth]{figuras/curvaROC.jpg}}
\subfloat[Clasificador aleatorio]{\label{ejemploROC4}\includegraphics[width=.50
\textwidth]{figuras/curvaROCrealidad.jpg}} \\
\subfloat[Buen clasificador (alto
\textit{TPR} y bajo \textit{FPR})]{\label{ejemploROC2}\includegraphics[width=.50
\textwidth]{figuras/curvaROCbueno.jpg}}
\subfloat[Mal clasificador, (bajo
\textit{TPR} y alto \textit{FPR})]{\label{ejemploROC3}\includegraphics[width=.50
\textwidth]{figuras/curvaROCmalo.jpg}}

\caption{Ejemplos de curvas ROC}
\label{ejemploROC}
\end{figure*}

La extensión de las curvas ROC para dos clases a problemas multiclase
es interesante, ya que conferiría los beneficios del análisis ROC a más
problemas en reconocimiento de patrones. Se han realizado multitud de aproximaciones,
aunque actualmente no hay ningún análisis sobre
este tema que esté completamente consolidado \cite{Lachiche2003}. En \cite{Everson2006}, se
considera un problema de optimización	multi-objetivo donde el objetivo es minimizar
simultáneamente los $Q(Q-1)$ errores de	clasificación dados por los valores no
pertenecientes a la diagonal principal de la matriz de	confusión, donde el error se
define por $\displaystyle \frac{n_{ij}}{n_{i\circ}}$, para $i=1,2,...,Q$ y $j\neq i$. De
esta forma, el coste computacional crece en función del número de	clases. En
\cite{Langrebe2008}, se propone un algoritmo que a partir de la matriz de confusión
de un clasificador identifica las clases independientes y grupos de clases que
interactúan entre sí, permitiendo la descomposición de la matriz en un curva ROC con un
número bajo de grupos dimensionales. Esto reduce la complejidad computacional
considerablemente. La hiper-superficie ROC descompuesta se puede tratar como en el
caso ideal (dos clases), permitiendo realizar aproximaciones coste-beneficio y la
aproximación de Neyman-Pearson \cite{Edwards2004}, así como el volumen bajo la curva,
AUC. Otra manera de trabajar con problemas multi-clase es mediante el uso del
volumen sobre la superficie ROC (\textit{Volume under Surface}, VUS) \cite{Dreiseitl2007}
con curvas en 3-D \cite{Sahiner2008,He2008}. Así por ejemplo, en \cite{Mossman1999}, el
concepto de curvas ROC se extiende a cuestiones de diagnóstico médico con tres
posibles alternativas; se dibuja una superficie ROC en tres dimensiones para una
tarea de decisión tridimensional, añadiendo el VUS sobre la superficie ROC. De esta
manera, el VUS resume la precisión global del modelo,
similar al AUC de una curva ROC hecho sobre una tarea de
clasificación con dos	alternativas. La obtención de información en los puntos de la
superficie se puede	calcular de la misma forma que para curvas ROC bidimensionales, así,
se pueden comparar tres tipos de curvas ROC bidimensionales, en función de la información
de cada superficie.
% Otra aproximación en problemas multiclase consiste en considerar los
% errores de clasificación de cada una de las clases, es decir, el porcentaje de falsos
% positivos para
% cada clase, definidos por los elementos exteriores a la diagonal principal de la matriz
% de
% confusión, en lugar de los errores de clasificación de cada una de las otras clases.
% Así,
% el problema queda reducido a la minimización de los $Q$ objetivos definidos por
% \begin{displaymath}
% \frac {\sum_{i\neq j}^{Q}n_{ij}}{n_{i\circ}}, \quad i=1,,2,...,Q.
% \end{displaymath}

De forma general, en el caso de utilizar MOEAs para la resolución de problemas multiclase
mediante aproximaciones ROC, se deben tener presentes algunos inconvenientes:
\begin{enumerate}
	\item La dimensión de los frentes de Pareto crece junto con el número de clases del
	problema, haciendo extremadamente difícil encontrar soluciones que dominen al resto.
	Esto es debido a que mientras más crezca el espacio
	de objetivos, más complicado es que una solución sea mejor que otra en al menos uno
	de los objetivos e igual en los demás, según la definición de no-dominancia
	\cite{Coello2007}.
	\item El coste computacional a la hora de obtener modelos de red también aumenta
	considerablemente, ya que el número de comparaciones y operaciones a realizar para
	obtener individuos no dominados crece exponencialmente, cuadráticamente o
	linealmente con el número de clases.
	\item Los	sucesivos	frentes de Pareto que conforman la población de
	soluciones tendrán	cada vez menos individuos, a medida que aumente el número de
	objetivos ($Q(Q-1)$ errores de	clasificación) a optimizar en el problema. Esto
	hace que una de las principales	ventajas de los MOEAs, que es el proporcionar al
	experto un amplio abanico de	soluciones igualmente válidas, no se aproveche.
	\item Trabajar con más de dos objetivos tiene la desventaja de que el número
	de dimensiones de las representaciones gráficas aumenta, y por tanto sea más
	difícil su análisis e interpretación.
	\end{enumerate}

% Para reducir estos inconvenientes, normalmente lo que se intenta es hacer una
% aproximación
% trabajando solo con los errores de clasificación
% para cada clase, es decir, los falsos positivos en lugar del error de clasificación en
% cada una de	las otras clases (definido por los elementos que hay fuera de la diagonal
% principal	de la matriz de confusión). Esta propuesta reduce la dimensión del
% problema desde el	punto de vista del número de objetivos, sin embargo, esta
% proyección solo suele ser efectiva para problemas de tres clases.

\section{Mínima Sensibilidad-Precisión ($MS,C$)}\label{ms-c}
\noindent Teniendo en cuenta las definiciones dadas en la sección
\ref{metricasmulticlase}, se puede definir la Mínima Sensibilidad, $MS$, de un
clasificador $g$, como el mínimo valor de las sensibilidades de cada una de las clases
de un problema:
\begin{displaymath}
MS=min\left\lbrace S_{i};i=1,...,Q\right\rbrace
\end{displaymath}

La principal ventaja de las medidas de rendimiento mencionadas en la sección \ref{2.1}
es su simplicidad, ya que resumen o recogen de manera más o menos precisa, en un solo
valor numérico, el rendimiento de un clasificador. Sin embargo, esta simplicidad es a la
vez un punto débil, ya que un mismo valor proporcionado por uno de esos indicadores puede
representar diferentes situaciones. Si se asume que todos los errores de clasificación son
igualmente costosos, y que no hay preferencia ni penalización por un determinado conjunto
de patrones, un buen clasificador debería obtener un alto nivel de precisión global, así
como un aceptable nivel de precisión para cada clase. La precisión no puede capturar todos
los aspectos de
comportamiento	de dos clasificadores diferentes \cite{Provost1997,Provost1998}, y no es
suficiente, en algunos casos, para	determinar la calidad de un clasificador.

En esta tesis proponemos una medida de rendimiento bidimensional, de manera que pueda estar
en un punto intermedio entre las medidas unidimensionales como $C$, y las
multidimensionales dadas por los errores de clasificación, definidos por los elementos
exteriores a la diagonal principal de la matriz de confusión. Así, se intentan evitar los
problemas y limitaciones de las medidas unidimensionales, sin sufrir los
problemas relacionados con las medidas en $Q$ dimensiones (inconvenientes comentados
anteriormente de las aproximaciones
basadas en los $Q(Q-1)$ porcentajes de mala clasificación). Consideremos por tanto, como
compromiso entre ambas posibilidades, las medidas $(MS,C)$ como medidas que expresan dos
características asociadas con un clasificador: el rendimiento global, $C$, y el porcentaje
de aciertos de la clase peor clasificada, $MS$.

La selección de $MS$ como una medida complementaria de $C$ se puede justificar
considerando que la ecuación (\ref{CenbaseS}) es la media ponderada de las
sensibilidades de cada una de las $Q$ clases. Desde un punto de vista estadístico, $C$ es
una medida buena y representativa del conjunto de las sensibilidades, si éstas son lo
suficientemente homogéneas, aunque no será una medida representativa si las sensibilidades
están muy dispersas. Teniendo en cuenta este hecho, una medida complementaria de $C$ se
podría obtener considerando alguna medida que minimizase dicha dispersión, por ejemplo, el
rango $R=max\{S_{i}\}-min\{S_{i}\}$ podría ser una posible elección. Esta minimización
se puede	alcanzar si se minimiza $max\{S_{i}\}$ o se maximiza $min\{S_{i}\}$. La primera
opción no es apropiada, ya que $C$ aumenta si las sensibilidades aumentan,
por tanto la segunda alternativa es la más apropiada. De esta forma $MS=min\{S_{i}\}$, se
puede	considerar como la medida complementaria de $C$, cuyo valor hay que maximizar.

\subsection{Ejemplos}
La consideración simultanea de $MS$ y $C$ puede ayudar a clarificar los
errores y confusión que generan por si solas las medidas unidimensionales, veamos algunos
ejemplos y contra-ejemplos sobre ello:

\textbf{Ejemplo 1 (inadecuación de $C$):} Consideremos un problema de
clasificación con tres clases, y las matrices de confusión, $A$ y $B$, asociadas a dos
clasificadores, $f$ y $g$.
\begin{equation} \nonumber
A=\left( \begin{array}[c]{ccc}
60 & 0 & 0\\
0 & 30 & 0\\
5 & 4 & 1
\end{array}\right) \quad
B=\left( \begin{array}[c]{ccc}
57 & 3 & 0\\
0 & 30 & 0\\
5 & 1 & 4
\end{array}\right)
\end{equation}
Ambos clasificadores tienen la misma precisión $\displaystyle C=\frac{91}{100}=0.91$. Sin
embargo, el rendimiento en las clases es muy diferente. La $MS$ del clasificador
$f$ es $\displaystyle S_{f}= min\left\lbrace 1,1,0.1\right\rbrace=0.1 $, mientras que la
$MS$ de $g$ es $\displaystyle S_{G}= min\left\lbrace 0.95,1,0.4\right\rbrace=0.4 $. Este
ejemplo muestra claramente que la precisión no es una medida óptima para evaluar la
calidad de un clasificador, ya que no tiene en cuenta la clasificación individual de las
clases, sino un resultado general.

En problemas altamente desbalanceados, por norma general, hay un error no uniforme a
favor de la clase minoritaria (a menudo la clase de mayor interés). De esta manera, los
clasificadores que optimizan la precisión son cuestionables, en cuanto a que raramente
predicen de manera adecuada la clase minoritaria.

\textbf{Contraejemplo 1:} La $MS$ de un clasificador $f$ es $\displaystyle
MS_{f}=Min\left\lbrace 1,1,0.1\right\rbrace=0.1$, mientras que la de un clasificador
$g$ es $\displaystyle MS_{g}=Min\left\lbrace 0.95,1,0.4\right\rbrace=0.4$. En la figura
\ref{marianoejemplo1} se pueden ver los clasificadores $f$ y $g$ dentro del plano
$(MS,C)$. Claramente se puede considerar el clasificador $g$ mejor que el $f$.

\begin{figure*}[!htb]
\centering
\subfloat[]{\label{marianoejemplo1}\includegraphics[width=.50
\textwidth]{figuras/articuloMarianoFigure2.jpg}}
\subfloat[]{\label{marianoejemplo3}\includegraphics[width=.50
\textwidth]{figuras/articuloMarianoFigure4.jpg}} \\
\subfloat[]{\label{marianoejemplo2}\includegraphics[width=.50
\textwidth]{figuras/articuloMarianoFigure3.jpg}}
\subfloat[]{\label{marianoejemplo4}\includegraphics[width=.50
\textwidth]{figuras/articuloMarianoFigure5.jpg}}
\caption{Clasificadores $f$ y $g$ en el plano $(MS,C)$.}
\label{marianofigures}
\end{figure*}

\textbf{Ejemplo 2 (Inadecuación de la macro-media):} Permítanos considerar un problema de
clasificación con cinco clases de tamaño 250, 150, 100, 100 y 10 respectivamente, y las
correspondientes matrices de confusión, $A$ y $B$,  asociadas a dos clasificadores $f$ y
$g$, con los siguientes elementos de la diagonal principal:
\begin{displaymath}
diag(A)=\left( 155,150,99,99,3\right); \quad
diag(B)=\left( 200,150,83,67,6\right),
\end{displaymath}
pudiendo tomar los valores exteriores a la diagonal principal cualquier valor.

Ambos clasificadores tienen la misma $MAVG$. Las sensibilidades para cada clase en el
clasificador $f$ son $\displaystyle S_{1}=0.62, S_{2}=1, S_{3}=0.99, S_{4}=0.99,
S_{5}=0.3$ y las del clasificador $g$ son $\displaystyle S_{1}=0.8, S_{2}=1, S_{3}=0.83,
S_{4}=0.67, S_{5}=0.6$. Así, la macro media de $f$ y $g$ es $\displaystyle \left( 1/5
\cdot \sum_{i=1}^5 S_{i}\right) =0.78$, con lo que no distingue a los clasificadores.

\textbf{Contraejemplo 2:} Los clasificadores del ejemplo 2 tienen la misma $MAVG$
y las precisiones de ambos son $\displaystyle \frac{506}{610}=0.829$, la $MS$ de
$f$ es $MS_{f}=0.3$, mientras que la $MS$ del clasificador $g$ es $\displaystyle
MS_{g}=0.6$, por tanto el clasificador $g$ es mejor que el clasificador $f$. En la
representación gráfica mostrada en la figura \ref{marianoejemplo3} se puede ver como $g$
domina a $f$.

\textbf{Ejemplo 3 (Inadecuación de la media geométrica):} Permítanos considerar un
problema de clasificación con 300 patrones y tres clases de 100, 100 y 100 patrones
respectivamente. Sean $A$ y $B$ las matrices de confusión correspondientes a los
clasificadores $f$ y $g$, de forma que sus diagonales principales vienen dadas por:
\begin{displaymath}
diag(A)= (50,80,90); \quad diag(B)= (60,60,100)
\end{displaymath}
Las sensibilidades para cada clase en el clasificador $f$ son $S_{1}=0.5, S_{2}=0.8,
S_{3}=0.9$ y para el clasificador $g$ son $S_{1}=S_{2}=0.6, S_{3}=1$, y la media
geométrica de los dos es $GM(f)=GM(g)=0.711$.

\textbf{Contraejemplo 3:} Los clasificadores del ejemplo 3 poseen la misma precisión
$C=0.733$. La $MS$ del clasificador $f$ es $MS_{f}=min\left\lbrace
0.5,0.8,0.9\right\rbrace = 0.5$, mientras que la de $g$ es $MS_{g}=min\left\lbrace
0.6,0.6,1\right\rbrace = 0.6$, por tanto podríamos decir que el clasificador $g$ es mejor
que el clasificador $f$ (ver figura \ref{marianoejemplo2}).

\textbf{Ejemplo 4:} Para este ejemplo hemos considerado los resultados obtenidos por
diferentes metodologías en \cite{Ou2007}. Los autores utilizan cuatro sistemas de ANNs con distinto
número de redes para la clasificación de patrones:
un sistema basado en un esquema uno contra uno (\textit{One Against One} OAO), un sistema
uno contra todos (\textit{One Against All}, OAA) para redes binarias, una red neuronal
simple entrenada con un esquema OAA, y un sistema de modelado uno contra el mejor orden
(\textit{One Against Higher Order}, OAHO). Estas metodologías se aplican a dos conjuntos
de datos desbalanceados, \textit{Glass} y \textit{Shuttle}. A partir de la tabla
\ref{tablaEjemplo4}, que se corresponde con la tabla 4 del citado artículo, podemos
representar los clasificadores obtenidos en el plano $(MS,C)$ para el conjunto de datos
\textit{Shuttle} (figura \ref{marianoejemplo4}). Dichos resultados muestran que la
metodología OAA con una red obtiene un clasificador dentro del plano $(MS,C)$ que domina a
los demás y que la precisión no distingue prácticamente entre las 6 metodologías.

\begin{table}[!htb]
\caption{Mínima sensibilidad, $MS$, y precisión, $C$, para los clasificadores en el
conjunto de datos Shuttle.}
\label{tablaEjemplo4}
\centering
\begin{tabular}{p{4cm}p{3.2cm}p{2.8cm}} \hline
\rowcolor[rgb]{0.70,0.85,1}\textbf{Metodología} & $\mathbf{MS}$ & $\mathbf{C}$\\ \hline
\rowcolor[rgb]{0.86,0.94,1}OAA, 7 redes & 0 & 0.9963\\
\rowcolor[rgb]{0.86,0.94,1}OAO, 21 redes & 0 & 0.9969\\
\rowcolor[rgb]{0.86,0.94,1}OAHO & 0.076 & 0.9969\\
\rowcolor[rgb]{0.86,0.94,1}OAA, 1 red & 0 & 0.9963\\
\rowcolor[rgb]{0.86,0.94,1}OAA, 1 red * & 0.307 & 0.9977\\
\rowcolor[rgb]{0.86,0.94,1}OAA, 1red ** & 0.923 & 0.9977 \\ \hline
\multicolumn{3}{l}{* Con duplicación previa de los datos de la clase minoritaria.} \\
\multicolumn{3}{l}{** Entrenamiento con datos de la clase minoritaria e incremento}
\\
\multicolumn{3}{l}{de la capacidad de la red para entrenamiento posterior con datos} \\
\multicolumn{3}{l}{dinámicos.}
\end{tabular}
\end{table}

\section{Propiedades de las medidas $(MS,C)$}\label{propiedades}
Previo a definir las propiedades de las medidas $(MS,C)$, es necesario establecer cuál
es la relación entre ambas:

Consideremos un problema de clasificación con $Q$ clases, siendo $C$ y $MS$ las
medidas asociadas a un clasificador $g$, y $p^*$ la mínima de las frecuencias relativas
de cada clase calculadas ''a priori``. De esta forma, podemos representar los valores
de $MS$ y $C$ en un eje de coordenadas, concretamente $MS$ en el eje $X$ y $C$ el eje $Y$,
de manera que se puede visualizar fácilmente el rendimiento de un clasificador, sin tener
en cuenta el número de clases	que tenga el problema.

Un punto en el plano $\left(MS,C\right)$ domina a otro si tiene un mayor valor de $C$ e
igual o mayor valor de $MS$, o si tiene mayor valor de $MS$ y mayor o igual de $C$.
Idealmente, a partir de la desigualdad \ref{desigualdad3}, cada clasificador se puede
representar como
un punto en la región blanca de la figura \ref{regionFactible}, de forma que el área
exterior al triángulo se marca como región no factible. El área interior al triángulo
puede ser factible a priori, dependiendo del clasificador y de la dificultad del problema;
así el clasificador óptimo $\left(MS=1,C=1\right)$ no es posible o factible para todos los
problemas y clasificadores, especialmente para problemas con elementos estocásticos. Por
esta razón es mejor decir que un clasificador no puede estar localizado en el área
marcada como región no factible. Además, puntos situados en el eje vertical
corresponderían a clasificadores que no son capaces de predecir correctamente ningún
elemento de una clase determinada. Hay que hacer notar que es posible encontrar algunos
clasificadores con un valor de $C$ alto, particularmente en problemas con valores pequeños
de $p^*$ o incluso $0$ (problemas no balanceados), y sin embargo con un valor de $MS$
bajo. Para valores de $C$ menores que $(1-p^*)$, el rango de posibles valores de $MS$
aumenta con respecto al valor de $C$, sin embargo cuando $C$ es mayor que $(1-p^*)$ el
rango de posibles valores de $MS$ disminuye cuando $C$ aumenta.

\begin{figure}[htb]
\centering
\includegraphics[keepaspectratio,width=11cm]{figuras/regionFactible.jpg}
\caption{Región no factible en el plano $\left(MS,C\right)$ para un
problema de clasificación con $p^*=0.2$.}
\label{regionFactible}
\end{figure}

Obsérvese que un incremento en $C$, no implica necesariamente un incremento
en $MS$. Recíprocamente, un incremento en la $MS$ no significa un
incremento en $C$. Por otro lado, es necesario decir que para un determinado
valor de $C$, un clasificador será mejor cuando se acerque a un punto lo más cercano
posible a la diagonal de la figura \ref{regionFactible} (tanto $C$ como $MS$ toman valores en el
intervalo
$\left[0,1\right]$). Si se utilizan MOEAs como metodología para
la obtención de clasificadores, al comienzo del proceso evolutivo, $MS$ y $C$ pueden ser
cooperativas, e incluso estar lineal y correladas positivamente, pero después de un
determinado número de generaciones, estos objetivos
llegan a ser competitivos, es decir, una mejora en uno de ellos, en general supone un
empeoramiento en el otro. En la Figura \ref{CvsS} se puede ver un ejemplo en el que $MS$ y
$C$ son, en general, objetivos en conflicto, ya que para aumentar el valor de $MS$ en el
clasificador $g_{1}$ es necesario ''perder`` la clasificación de algunos patrones, de
manera que $C$ disminuye en $g_{2}$.
\begin{figure}[htb]
\centering
\includegraphics[keepaspectratio,width=12.5cm]{figuras/CvsS.jpg}
\caption{Ejemplo gráfico para describir $MS$ y $C$ como objetivos en
conflicto.}
\label{CvsS}
\end{figure}

A continuación se muestran las propiedades de las medidas $(MS,C)$ como medidas de rendimiento de
un clasificador:
\begin{description}
\item[\textbf{Proposición 1.}] Consideremos de nuevo un problema de clasificación con $Q$
clases, y $J$ la clase con la mínima probabilidad de pertenencia o $p^{*}$, es
decir, $p^{*}=\frac{n_{J\circ}}{N}$, siendo $n_{J\circ}$ el tamaño de la clase
más pequeña y $N$ el tamaño del conjunto de datos. Esta consideración da lugar a la
siguiente desigualdad $MS\leq C\leq 1- \left(1-MS \right) p^{*}$.
\item[\textbf{Demostración.}] Comenzamos demostrando el límite superior de la
inecuación. Teniendo en cuenta
las definiciones de $C$ y $MS$, y que $\displaystyle \sum_{i=1}^Q n_{i\circ}=N$ se puede
comprobar que:
\begin{eqnarray}\label{desigualdad1}
C=\sum_{i=1}^Q \frac{n_{ii}}{n_{i\circ}}\frac{n_{i\circ}}{N}=\sum_{i=1}^Q
S_{i}\frac{n_{i\circ}}{N}=;MS\frac{n_{J\circ}}{N}+\sum_{i\neq
J}S_{i}\frac{n_{i\circ}}{N}\leq\\
\leq MS\frac{n_{J\circ}}{N}+\sum_{i\neq
J}\frac{n_{i\circ}}{N}=MS\frac{n_{J\circ}}{N}+1-\frac{n_{J\circ}}{N}
=1-\left(1-MS\right)\frac{n_{J\circ}}{N}
\nonumber
\end{eqnarray}
Por otro lado, el límite inferior se obtiene de la siguiente forma:
\begin{equation}\label{desigualdad2}
C=\frac{1}{N}\sum_{i=1}^Q n_{ii}=\sum_{i=1}^Q
\frac{n_{ii}}{n_{i\circ}}\frac{n_{i\circ}}{N}=\sum_{i=1}^Q S_{i}\frac{n_{i\circ}}{N}\geq
MS\sum_{i=1}^Q\frac{n_{i\circ}}{N}=MS
\end{equation}
uniendo (\ref{desigualdad1}) y (\ref{desigualdad2}) se puede concluir que
\begin{equation}\label{desigualdad3}
MS\leq C\leq	1-\left(1-MS\right) p^{*}
\end{equation}
% La relación entre $C$ y $MS$ se puede ver en la figura \ref{regionFactible}, de la
% cual se hablará más detalladamente en la sección \ref{seccionfactible}.
\item[\textbf{Proposición 2.}] Sea un problema de clasificación con $Q$ clases y $p^*$ la
mínima de las frecuencias relativas de clase calculadas ''a priori``. Cada punto $(MS,C)$
perteneciente al subconjunto
\begin{displaymath}
		F=\left\lbrace 	(MS,C):0\leq MS\leq 1, MS\leq C\leq 1-p^*+(MS)p^*\right\rbrace
\end{displaymath}
corresponde al menos a un clasificador $g$. Así, el subconjunto completo $F$ es
factible.
\item[\textbf{Demostración.}] Para probar que el subconjunto $F$ es factible,
probaremos que cada punto de dicho subconjunto se puede obtener a partir de una matriz de
confusión. Sea $\left( MS_{0},C_{0}\right)$ un punto en $F$, la demostración se basa en
la construcción de una matriz de confusión con $MS=MS_{0}$ y $C=C_{0}$ para ese punto.
Definamos la función continua y lineal $f:\left[ MS_{0},1\right]
\rightarrow\mathbb{R}$ como:
\begin{displaymath}
f(MS)=(MS_{0})p^*+MS(1-p^*)
\end{displaymath}
la cual satisface
\begin{displaymath}
f(MS_{0})=MS_{0}\leq C_{0}, f(1)=1-p^*+(MS_{0})p^*\geq C_{0}
\end{displaymath}

La continuidad de la función $f$ implica que existe un valor $MS^*\in \left[
MS_{0},1\right]$ de manera que $f(S^*)=C_{0}$. Definamos la siguiente matriz de
confusión:
\setlength{\arraycolsep}{0.75pt}
\renewcommand{\arraystretch}{1.75}
\scriptsize
\[\begin{array}{ccccccccc}
Clase &  & 1 & 2 & \cdots & Q-1 & Q &  &  \\
1 & \multirow{5}{*}{$\left(
\begin{array}{c}
\\
\\
\\
\\
\\
\end{array}\right.$} & (MS_{0})p^* &
(1-MS_{0})p^* & \cdots & \cdots & 0 & \multirow{5}{*}{$\left.
\begin{array}{c}
\\
\\
\\
\\
\\
\end{array}\right)$} & p^* \\
2 &  & \frac{(1-S^*)(1-p^*)}{Q-1} & \frac{S^*(1-p^*)}{Q-1} & \cdots & \cdots & 0 &  &
\frac{1-p^*}{Q-1}\\
\cdots &  & \cdots & \cdots & \cdots & \cdots & \cdots &  & \cdots\\
Q-1 &  & 0 & 0 & \frac{(1-S^*)(1-p^*)}{Q-1} & \frac{S^*(1-p^*)}{Q-1} & 0 &  & \cdots\\
Q &  & 0 & 0 & \cdots & \frac{(1-S^*)(1-p^*)}{Q-1} & \frac{S^*(1-p^*)}{Q-1} &  &
\frac{1-p^*}{Q-1}
\end{array}\]
% Reestrabelcemos de nuevo los tamaños por defecto
\normalsize
\renewcommand{\arraystretch}{1}
\setlength{\arraycolsep}{1pt}
\newline

Se puede observar que las sensibilidades cumplen lo siguiente:
\begin{displaymath}
S_{1}=S_{0}, S_{2}=...=S_{Q}=\frac{S^*(1-p^*)}{Q-1}
\frac{Q-1}{1-p^*}=S^*
\end{displaymath}

Por tanto $\displaystyle
MS=min\left\lbrace MS_{0},S^*\right\rbrace=MS_{0}$, ya que $\displaystyle S^*\in\left[
MS_{0},1\right]$. Por otra parte, se verifica que la precisión, $C$, asociada a la matriz
es:
\begin{displaymath}
C=MS_{0}p^*+(Q-1)S^* \frac{1-p^*}{Q-1}=f(S^*)=C_{0}
\end{displaymath}
Finalmente, a partir de la condición $\displaystyle p^*\leq \frac{1}{Q}$ obtenemos que
\begin{displaymath}
p^*Q\leq1 \Longrightarrow p^*Q-p^*\leq 1-p^*\Longrightarrow p^*\leq \frac{1-p^*}{Q-1}
\end{displaymath}
y podemos concluir que la mínima probabilidad de las columnas es $p^*$.
\item[\textbf{Proposición 3.}] El área $A$ del conjunto factible $F$ (región dibujada en
blanco en la figura \ref{regionFactible}) verifica que $\displaystyle A= \frac{1-p^*}{2}$,
donde $p^*$
es la mínima de las frecuencias relativas de cada clase.
\item[\textbf{Demostración.}] La demostración es inmediata a partir de la Proposición 2.
Cuando el número de clases aumenta o el problema es altamente desbalanceado, el valor de
$\displaystyle p^*\leq \frac{1}{Q}$ disminuye en ambos casos. Por tanto una consecuencia
de la Proposición 3 es que, el área de $F$ aumenta cuando el nivel de desbalanceo o el
número de clases aumenta. Un análisis similar prueba que el valor mínimo del área se
alcanza cuando el problema es totalmente balanceado, es decir, $\displaystyle p^*=
\frac{1}{Q}$. En consecuencia, en problemas no balanceados, la línea que define el borde
superior del conjunto en el plano $(MS,C)$
tiende a ser horizontal, y el rango de valores de $MS$ será mayor incluso para valores
altos de $C$. En este caso, el valor de $C$ como medida única de un clasificador
multiclase es insuficiente, ya que esconde muchas posibilidades diferentes para $MS$.
Estos comentarios muestran que las medidas $(MS,C)$ pueden ser especialmente ventajosas
en problemas con cierto desbalanceo o cuando el número de clases es alto, confirmando
que, por si sola, la medida $C$ es inadecuada en estas situaciones.

Finalmente, analizamos la relación existente entre las medidas $(MS,C)$ y las curvas
ROC en problemas binarios. Los siguientes resultados establecen la relación
entre los dos espacios en un problema de clasificación binario, probando que no hay
equivalencia entre ellos. En este sentido las medidas $(MS,C)$ constituyen un punto de vista
complementario a las curvas ROC en problemas binarios.
\item[\textbf{Proposición 4.}] Sea $Q=2$ el número de clases y $q$ la probabilidad ''a
priori`` de que un patrón pertenezca a la primera clase. Entonces cada punto
$(TPR_{0},FPR_{0})$ en el plano \textit{ROC} determina un único punto factible
$(MS_{0},C_{0})$ en el plano $(MS,C)$ definido por:
\begin{displaymath}
MS_{0}	=min\left\lbrace TPR_{0}, 1-FPR_{0}\right\rbrace \qquad
C_{0}=TPR_{0}q+(1-FPR_{0})(1-q)
\end{displaymath}
Inversamente, cada punto $(MS_{0},C_{0})$ de $F$, determina dos puntos en el plano
\textit{ROC} dado por:
\begin{align}
	FPR_{1}&=\frac{1-C_{0}-(1-MS_{0})q}{1-q},\qquad TPR_{1}=MS_{0} \nonumber \\
	FPR_{2}&=1-MS_{0}, \qquad TPR_{2}=\frac{C_{0}-MS_{0}(1-q)}{q} \nonumber
\end{align}
\item[\textbf{Demostración.}] Sea $\displaystyle (TPR_{0},FPR_{0})$ un punto en el
plano ROC, supongamos que la matriz de confusión está dada por $\left( \begin{array}{cc}
n_{11} & n_{12}\\
n_{21} & n_{22}
\end{array}\right) $, y sea $N$ el número de patrones. Recordando la definición de
MS, tenemos que
\begin{equation}\label{p4-1}
S_{0}=min\left\lbrace TPR_{0}, 1-FPR_{0}\right\rbrace
\end{equation}
Por otro lado, la precisión $C_{0}$ es $\displaystyle \frac{n_{11}}{N}+\frac{n_{22}}{N}$.
Teniendo
en cuenta que $\displaystyle TPR_{0}=\frac{n_{11}}{n_{1\circ}}$ y
$FPR_{0}=\frac{n_{21}}{n_{2\circ}}$, donde $n_{1\circ}=n_{11}+n_{12}$ y
$n_{2\circ}=n_{21}+n_{22}$ son
el número de patrones en la primera y segunda clase respectivamente, y que $q$ es la
probabilidad "a priori" de la primera clase, tenemos
\begin{equation}\label{p4-2}
\begin{split}
C_{0}&=\frac{n_{11}}{N}+\frac{n_{22}}{N}=\frac{n_{11}}{n_{1\circ}}\frac{n_{1\circ}}{N}
+\frac{n_ {22}}
{n_{2\circ}}\frac{n_{2\circ}}{N}= \\
&=(TPR_{0})q+(TPR_{0})(1-q)=(TPR_{0})q+(1-FPR_{0})(1-q)
\end{split}
\end{equation}
Inversamente, si $(MS,C)$ es un punto factible, a partir de las ecuaciones \ref{p4-1} y
\ref{p4-2} previamente establecidas se tiene que
\begin{align}
FPR_{1}&=\frac{1-C_{0}-(1-S_{0})q}{1-q},\qquad TPR_{1}=S_{0} \nonumber \\
FPR_{2}&=1-S_{0}, \qquad TPR_{2}=\frac{C_{0}-S_{0}(1-q)}{q} \nonumber
\end{align}
dependen estos puntos de la clase en la que la sensibilidad alcance el valor más bajo.
Obsérvese que $0\leq TPR_{0},FPR_{0}\leq1$.
A partir de estos resultados podemos hacer las siguientes observaciones:
\begin{enumerate}
	\item Es fácil probar que
	\begin{displaymath}
	FPR_{2}=FPR_{1}+\frac{C-SM}{1-q}, \qquad TPR_{2}=TPR_{1}+\frac{C-MS}{q}
	\end{displaymath}
	siendo la pendiente del segmento que une los puntos ROC asociados el
	$\displaystyle \left(\frac {1-q}{1}\right) $ por ciento entre las probabilidades ''a
priori``,
	independientemente del punto $(MS_{0},C_{0})$ considerado. Como $\displaystyle
	\left( \frac{1-q}{1}>0\right)$, ningún punto domina a los demás.
	\item Los puntos obtenidos en el plano ROC son iguales, si y solo si, $MS=C$.
	Es decir, los puntos en la diagonal del plano $(MS,C)$ determinan solamente un punto en
el
	plano ROC.
\end{enumerate}
\end{description}

\section{Problemas no balanceados y/o con gran número de clases}\label{balanceados}
\noindent Como ya hemos comentado en la sección anterior, cuando el número de clases
aumenta, o cuando existe un problema altamente
desbalanceado, el valor de $\displaystyle p^*=\frac{n_{\circ j}}{N}\leq \frac{1}{Q}$ disminuye, y la
línea
del borde superior de la región factible, dentro del plano $\left(MS,C\right)$, tiende a
ser horizontal, es decir, a tomar el valor $C=1$, de manera que el rango de valores de
$MS$ será más amplio cuando $C$ tome valores altos. En este tipo de problemas
\cite{Ho2002,Murphey2004} es difícil
mantener un nivel de precisión alto en cada clase,	debido a la dificultad en el
aprendizaje a partir de clases con pocos patrones, que puede conllevar una disminución
del valor de $C$ en el conjunto de generalización.

De forma general, la solución aportada al tratamiento de problemas no balanceados se
puede dividir en aproximaciones a nivel de datos y aproximaciones a nivel de algoritmo
\cite{He2009,Sun2009}.

\begin{itemize}
\item A nivel de datos, la solución es rebalancear la distribución por clase mediante
técnicas
de remuestreo, ya sea quitando patrones de la clase mayoritaria o incluyéndolos en la
minoritaria, o incluso una combinación de las dos \cite{Kubat97,Chawla2002}. Estos métodos
conllevan desventajas:
\begin{enumerate}
\item La distribución óptima por clase de un conjunto de entrenamiento es usualmente
desconocida.
\item Una estrategia de remuestreo poco efectiva
puede ocasionar riesgos de pérdida de información sobre las clases mayoritarias en caso de
hacer un muestreo eliminando patrones, o sobre las clases minoritarias, en caso de hacer
un muestreo aumentando el numero de patrones, pudiendo en este último caso ocasionar un
sobreajuste o sobreaprendizaje de los datos.
\item En la mayoría de los casos existe un inevitable coste extra relacionado con el
análisis y procesamiento de los datos.
\end{enumerate}
\item A nivel algorítmico, lo que se hace es adaptar el aprendizaje del algoritmo
para que influya sobre la clase minoritaria \cite{Jankowski2001b}. En \cite{Fernandez2009}
y \cite{Fernandez2009a} se pretende que dos EAs para ANNs vayan
nivelando el porcentaje de buena clasificación en cada una de las clases de un
problema determinado, o al menos dejarlas de la manera más equilibrada posible, usando
respectivamente $MS$ y el coeficiente de variación de las sensibilidades de
todas las clases. En \cite{Martinez-Estudillo2008} este objetivo se aborda utilizando un
algoritmo en dos etapas, optimizando $C$ en la primera etapa, y en la segunda optimizando
una función definida a partir de $C$ y $MS$, que geométricamente corresponde al área de un
recinto definido por el clasificador en la región factible de la figura
\ref{regionFactibleAreaEA}, donde dicha área viene definida por:
\begin{displaymath}
A(MS,C)=(1-MS)(1-C)-\frac{1}{2}\left[ (1-C)^2-p^*(1-MS)^2\right]
\end{displaymath}
\end{itemize}

La mayoría de nuevos métodos que se describen en esta tesis doctoral para la obtención de
clasificadores usando las medidas $(MS,C)$ son métodos a nivel de algoritmo, y bajo este
punto
de vista, $MS$ y $C$ pueden ser ventajosas como medidas de precisión en problemas no
balanceados y/o con un número de clases elevado.
\newpage

\begin{figure}[!htp]
\centering
\includegraphics[keepaspectratio,width=9cm]{figuras/EA.jpg}
\caption{Área sobre un clasificador evaluado, función $A(MS,C)$.}
\label{regionFactibleAreaEA}
\end{figure}
\paginavaciacompleta
		%\paginavaciacompleta
		% \chapter{Redes neuronales artificiales}\label{redesneuronales}
% \begin{quotation}
% \begin{small}
% \textit{La vergüenza de confesar el primer error hace cometer otros muchos.}
% \end{small}
% \begin{flushright}Jean de la Fontaine.\end{flushright}
% \end{quotation}
\section{Modelos de redes neuronales artificiales}\label{modelos}
\noindent
% Las ANNs \cite{Bishop1995,Haykin2008} se pueden definir como estructuras que se
% diseñan para resolver cierto tipo de problemas asociados a emular la forma en la que el
% cerebro humano puede resolverlos, de esta manera  su  interpretación  biológica  es
% similar a la activación  de  las neuronas cerebrales.  A diferencia  de  los
% algoritmos procedurales convencionales que  siguen  un  flujo secuencial,  las
% computaciones neuronales se realizan a través de un gran número de nodos.
Una  ANN \cite{Bishop1995,Haykin2008} es un  modelo matemático que simula un sistema
neuronal biológico, donde las neuronas se modelan mediante unidades
de  procesamiento  o  nodos interconectados.  Cada  unidad  de  procesamiento  se
compone de un conjunto de conexiones  de  entrada,  una  función  de  propagación
(encargada  de  computar  la entrada  total combinada de todas las conexiones), un núcleo
central de proceso (encargado de aplicar la función de activación) y una salida (por donde
se transmite el valor de activación a otras unidades). De esta forma, los elementos que
caracterizan un nodo de una
ANN son:
\begin{description}
	\item[Conexiones ponderadas:] Hacen  el  papel  de  las conexiones  sinápticas.  La
	existencia  de conexiones determina si es posible que un nodo influya sobre otro,
	mientras que el valor de los pesos y el signo de los mismos definen el tipo
	(excitativo/inhibitorio) y la intensidad de la influencia.
	\item[Función de activación:] Calcula el valor de base o entrada total al nodo,
	generalmente como una  suma  ponderada  de  todas  las  entradas  recibidas,  es
	decir,  de  las entradas multiplicadas por  el  peso  o  valor  de  las  conexiones.
	\item[Función de salida o de transferencia:] Calcula la salida del nodo en función de
	la activación de la misma. Se  usan diferentes	tipos  de  funciones,  desde  simples
	funciones de umbral a funciones no lineales.
 \end{description}

La  arquitectura  de  una  ANN (véase figura \ref{redSigmoideHaciaDelante}) es  la manera
en que se disponen sus nodos y conexiones. Los nodos según  su situación  dentro  de  la
red pueden
ser  de  tres  tipos:
\begin{description}
	\item[Nodos  de entrada:] Reciben  las  señales  de entrada de la red, es decir, los
	valores de las variables independientes del modelo, y forman la capa	de entrada.
	\item[Nodos de salida:] Envían las señales de salida al exterior, es decir, los
	valores de las variables dependientes del modelo, y forman la capa de salida.
	\item[Nodos ocultos:] Son el resto de nodos,  y  se  agrupan  en  una  o  más  capas
	ocultas. Normalmente se suelen utilizar ANNs con una sola capa oculta, aunque puede
	haber varias.
\end{description}

\begin{figure}[htb]
\centering
\includegraphics[keepaspectratio,width=12.5cm]{figuras/redSigmoideHaciaDelante.jpg}
\caption{Red neuronal hacia delante con unidades de base sigmoide y una capa oculta.}
\label{redSigmoideHaciaDelante}
\end{figure}

Existen  distintas  formas  de transmisión  de  la información entre los nodos de una ANN,
que determinan la naturaleza de la misma. De este modo, los tipos de transmisión son los
siguientes:
\begin{description}
	\item[Transmisión hacia delante:] Este  tipo  de transmisión  se
	produce desde los nodos de la capa de entrada	hasta los	nodos de la capa de salida.
	Las ANNs que presentan este tipo de transmisión	se
	denominan redes de transmisión hacia	delante o \textit{feedforward}.
	\item[Transmisión lateral:] Este tipo de transmisión se produce	entre nodos
	de la 	misma capa. Las	ANNs que presentan este tipo de transmisión se denominan
	redes con	transmisión lateral.
	\item[Transmisión hacia atrás:] Este tipo de transmisión	(también llamada transmisión
	\textit{feedback}) se	produce desde los nodos de la capa de salida hasta los nodos de
	la capa de entrada. Las	ANNs	que  presentan  este  tipo  de  transmisión  se
	denominan  redes  con  realimentación o redes recurrentes.
 \end{description}
En este trabajo trabajo se utilizan redes de transmisión hacia delante, con una sola
capa oculta y con unidades de base sigmoide, por su buena adaptación y resultados en
problemas de clasificación de patrones.

El modelo funcional o función de salida asociada a cada
una de las neuronas de la salida de este tipo de redes es el siguiente:
\begin{equation}\label{modelof}
	f_{q}\left(\mathbf{x},\mathbf{\Theta}_{q} \right)= \beta_{0}^q + \sum_{j=1}^M
	\beta_{j}^q B_{j} \left( \mathbf{x},\mathbf{w}_{j}\right), \quad \text{para } q=1...Q
\end{equation}
donde $Q$ es la neurona de salida $q$ de la
red (en el caso de que haya varias salidas), $\displaystyle
\mathbf{\Theta}_{q}=\left(\beta_{0}^q,\beta_{1}^q,...,\beta_{M}^q,
\mathbf{w}_{1},...,\mathbf{w}_{M}\right)$ es el vector de pesos de esa neurona
$q$,
$\displaystyle \mathbf{w}_{j}=\left( w_{0}^j,w_{1}^j,...,w_{k}^j\right) $ es el vector de
pesos de las conexiones entre la capa de entrada y la neurona de la capa oculta $j$, $M$
es el número de neuronas de la capa oculta (ver figura \ref{redSigmoideHaciaDelante}), $K$ el número
de neuronas o características en la capa de entrada (ver figura \ref{redSigmoideHaciaDelante}),
$\mathbf{x}$ el valor de las entradas de la ANN, $\displaystyle B{j}\left(
\mathbf{x},\mathbf{w}_{j}\right)$ es la función de base de la neurona $j$
de la capa oculta y $\displaystyle \beta_{0}^q$ y $\displaystyle w_{0}^j$ son los sesgos
del modelo asociado a la neurona $q$ en la capa de salida, y a la neurona $j$ de la capa
oculta respectivamente.

\section{Taxonomía}\label{taxonomia}
\noindent Se pueden considerar dos tipos fundamentales de funciones de
base o transferencia en una ANN:
\begin{description}
\item[Funciones de ventana:] Son funciones de entorno local, como pueden ser las
funciones de base radial (\textit{Radial  Basis  Functions}, RBFs).  Poseen  una  mayor
capacidad de aproximar  datos anómalos aislados, pero una mayor dificultad en entornos
globales o cuando el número de entradas es demasiado alto.
\item[Funciones de proyección:] Son funciones de entorno global, como las unidades
de base sigmoide (\textit{Sigmoidal Units}, SUs) o las  unidades de base Producto
(\textit{Product	Units}, PUs). Al ser globales, presentan dificultades en 	la
aproximación de  datos	aislados pero suelen actuar mejor en problemas donde el	número
de variables es alto.
\end{description}

En  general,  consideraremos  dos  tipos  de  modelos  de  red, en función de
la activación de las variables independientes: Modelos aditivos  y
modelos multiplicativos:
\begin{description}
	\item[Modelo aditivo:] Este modelo es el más utilizado dentro de las ANNs, y los nodos
	que componen la estructura de la red proporcionarían la siguiente salida:
	\begin{displaymath}
	B{j}\left(
	\mathbf{x},\mathbf{w}_{j}\right)=h\left(w_{0}^j+w_{1}^jx_{1}+w_{2}^jx_{2}+...+w_{n}^jx_{
	 k} \right)
	\end{displaymath}
	siendo $n$ el número de características o neuronas de entrada de la red, $w_{0}^j$ el
	sesgo del modelo asociado a la neurona $j$ y $h(.)$ la función de salida o
transferencia
	de la	neurona $j$.
	\item[Modelo multiplicativo:] Este modelo es más reciente, e   intenta  afrontar
	aquellos  casos  en	los  que  existe  una  interacción  entre  las  variables  y  las
	regiones  de decisión	que no  pueden 	separarse por hiperplanos:
	\begin{displaymath}
	B_{j}\left(\mathbf{x},\mathbf{w}_{j}\right)=
	x_{1}^{w_{1}^j}x_{2}^{w_{2}^j}...x_{n}^{w_{k}^j}
	\end{displaymath}
	Como se puede observar no existe un sesgo, pues carece de sentido para este tipo
	de modelos. La función de salida o transferencia suele ser la función identidad.
\end{description}

\subsection{Redes con unidades de base sigmoides}\label{unidadesSigmoide}
\noindent Las neuronas con SUs tienen la siguiente expresión como
función de salida o transferencia:
\begin{displaymath}
	B_{j}\left(\mathbf{x},\mathbf{w}_{j}\right)=\frac{1}{1+exp\left(-\left(w_{0}
^j+\sum_{i=1}^{K} w_{i}^{j}x_{i} \right) \right)}
\end{displaymath}

En la figura \ref{ejemploSigmoides} se muestran algunos ejemplos de funciones de base
sigmoide.

\begin{figure}[htb]
\centering
\includegraphics[keepaspectratio,width=12.5cm]{figuras/ejemploSigmoides.jpg}
\caption{Ejemplos de funciones de base sigmoide, SUs.}
\label{ejemploSigmoides}
\end{figure}

Las redes con SUs, también conocidas como perceptrón multicapa (\textit{Multi Layer
Perceptron}, MLP), poseen las siguientes propiedades:
\begin{enumerate}
	\item La familia de funciones reales que representan puede aproximar cualquier función
	continua	con  suficiente  precisión,  con  tal  de  que  el  número  de  nodos  de  la
	capa	oculta  no  esté	acotado. Se dice por tanto que son aproximadores universales.
	\item Son fáciles de entrenar, aunque se obtienen con frecuencia óptimos locales.
	\item Se obtienen modelos eficientes en entornos alisados.
	\item Son funciones acotadas.
	\end{enumerate}
En la figura \ref{redSigmoideHaciaDelante} se representa una ANN hacia delante
con unidades de base sigmoides, con una capa de entrada, una capa oculta y una capa de
salida.

\subsection{Redes con unidades de base producto}
\noindent     Las ANNs con PUs introducidas por Durbin y Rumelhart \cite{Durbin1989} en
1989, son aquellas que están formadas por nodos que siguen un modelo multiplicativo de
proyección, con la siguiente función de salida o transferencia:
\begin{displaymath}
B_{j}\left(\mathbf{x},\mathbf{w}_{j}\right)=x_{1}^{w_{1}^j}x_{2}^{w_{2}^j}...x_{n}^{
	w_{n}^j}=\prod_{i=1}^n x_{i}^{w_{i}^j}
\end{displaymath}

Entre las ventajas de las redes PU se encuentran las siguientes:
\begin{enumerate}
	\item Como consecuencia del Teorema de Stone-Weierstrass \cite{Bishop1961}, se ha
	demostrado que las
	redes PU son aproximadores universales, (obsérvese que las funciones
	polinómicas de varias variables, son un subconjunto de los modelos basados en unidades
	producto).
	\item La  posibilidad  de  usar  exponentes  negativos  permite  expresar  cocientes  y
	entre  las variables.
	\item Durbin  y  Rumelhart \cite{Durbin1989} demostraron  que  la  capacidad  de
	información  de  una
	única unidad  de tipo  producto  (medida  como  la  capacidad  para  el  aprendizaje
	de	patrones booleanos aleatorios) es aproximadamente igual a $3N$, comparado con el
	valor de	$2N$ que corresponde a una unidad de tipo aditivo, siendo $N$ el número de
	entradas de la unidad.
	\item Los  exponentes  del  modelo  son  números  reales.  Esta  característica  tiene
	especial importancia,  sobre  todo  si  se  tiene  en  cuenta  que  son  frecuentes
	las	situaciones  en  el modelado de datos reales en las que, la relación entre las
	variables	responde a una estructura de  tipo  potencial,  donde  las  potencias  no
	están	restringidas  a  los  números naturales  o enteros.
	\item Permiten  implementar  funciones  polinómicas  de  orden  superior.  Han
	demostrado	buenos resultados en el modelado de datos en los que existen interacciones
	de diferentes	órdenes entre  las  variables  independientes  del  problema.  De  esta
	forma,	cuando existen interacciones  entre  las  variables  que  intervienen  en  un
	determinado  fenómeno, las  ANNs basadas en PUs permiten obtener modelos
	más simplificados que las ANNs de tipo sigmoide.
	\item Junto a esto, es posible obtener cotas superiores de la dimensión
	de \textit{Vapnik-Chervonensky} (VC), para redes basadas en PUs similares a las
	conocidas para las	redes SUs, lo que supone que poseen una capacidad de
	generalización similar.
	\item A diferencia de lo que ocurre con las redes SUs o RBFs,
	las	funciones derivadas parciales del modelo obtenido a partir de una red PU
	son funciones	del   mismo tipo. Este hecho ayuda con frecuencia al estudio de las
	tasas de cambio de la   variable dependiente  del  modelo  con  respecto  a  cada  una
	de  las  variables independientes.
\end{enumerate}

Como  contrapartida,  las  ANNs  con  PUs  presentan  un  inconveniente
importante:

La  superficie  de error  asociada  es  especialmente  compleja,  con  numerosos  óptimos
locales  y regiones  planas,  y  por tanto  con  mayor  probabilidad  de  quedar  atrapado
en  alguno \cite{Ismail2000}.  La  estructura potencial  del  modelo provoca  que
pequeños cambios en los exponentes  tengan  como  consecuencia  un cambio significativo
en los  valores  de la  función  y  en  la  función  de  error.  Así, los  algoritmos  de
entrenamiento de redes basados en el gradiente quedan con frecuencia, y de forma especial
en este tipo de ANNs, atrapados en óptimos locales. Esta dificultad
relacionada con
el entrenamiento, es una de las razones por las que, a pesar de las ventajas anteriormente
señaladas, la teoría de ANNs basadas  en  PUs  ha  tenido  poco
desarrollo,  y  han  sido  menos utilizadas  como modelos para problemas de
clasificación y de regresión que otros tipos de ANNs.

En la figura \ref{ejemploProducto} se muestran algunos ejemplos de funciones de base
producto.

\begin{figure}[htb]
\centering
\includegraphics[keepaspectratio,width=12.5cm]{figuras/ejemploProducto.jpg}
\caption{Ejemplos de funciones de base producto, PUs.}
\label{ejemploProducto}
\end{figure}

\subsection{Redes con unidades de base radial}
\noindent     Los nodos ocultos de las ANNs con RBFs presentan funciones de activación de
tipo ventana.  Cada uno  de  los  nodos  RBF  realiza  una
aproximación local  independiente del  espacio  de búsqueda,  normalmente  mediante  una
campana  de Gauss.  La  capa  de  salida  de  tipo lineal  auna  el efecto  independiente
de  cada nodo,  sumando  cada  valor  obtenido.  La  idea  es  que cada  nodo  esté
situado en un entorno del espacio de búsqueda formado por las variables de entrada y
además con un radio determinado. El proceso de aprendizaje de la red de RBFs consistirá en
ir moviendo dichos nodos a lo largo del espacio de búsqueda, modificando los centros y los
radios de los mismos, para ajustarlos de la mejor forma a los datos de entrenamiento.

La  función  de  activación  será  equivalente  a  la  función  de  distancia
euclídea  (tomando  como centro de la RBF el vector $\mathbf{w}_{j}$), y la
función de
transferencia será una función de tipo Gaussiano. Por tanto, la función de
salida o transferencia sería la siguiente:
\begin{displaymath}
B_{j}\left(\mathbf{x},\mathbf{w}_{j}\right)=
e^{-\frac{1}{2}\left( \frac{d\left( \mathbf{x},\mathbf{w}_{j}\right) }{r_{j}}\right) }
\end{displaymath}
siendo
\begin{displaymath}
	d\left( \mathbf{x},\mathbf{w}_{j}\right) =\lVert
	\mathbf{x}-\mathbf{w}_{j}\rVert=\sqrt{\sum_{i=1}^n \left( x_{i}-w_{ij}\right)^2 }
\end{displaymath}
siendo $w_{ij}$ el peso asociado a la neurona $j-esima$ de la capa oculta y la neurona
$i-esima$ de la capa de entrada.

Algunas de las características de las redes RBF son:
\begin{enumerate}
		\item Se ha demostrado que las ANNs con unidades RBF son aproximadores
		universales \cite{Park1991}.
		\item En comparación con las redes MLP, las redes RBF presentan la
		ventaja de que	cada nodo en la capa oculta es un elemento local en el modelo, lo
		que hace que para	un	determinado patrón sólo algunas unidades ocultas se activen.
		Esta característica	facilita	el entrenamiento, disminuyendo el número de óptimos
		locales y regiones	planas de la superficie de error, al desaparecer gran parte
		de las	interacciones entre los pesos.
		\item Por último, el proceso de entrenamiento de una red MLP consta
		de una sola etapa, mientras que las redes RBF suelen entrenarse en dos etapas,
		siendo las	funciones	de base aproximadas en primer lugar mediante técnicas de
		aprendizaje no supervisado, determinando los centros y el número de funciones de
		base mediante, por ejemplo, un algoritmo $k$-medias,  y los	pesos	entre las capas
		oculta y de salida en segundo lugar,	mediante métodos supervisados.
\end{enumerate}

En la figura \ref{ejemploRadial} se muestra un ejemplo gráfico de la superficie generada
por el efecto aditivo de tres neuronas \textit{RBF}.
\begin{figure}[htb]
\centering
\includegraphics[keepaspectratio,width=12.5cm]{figuras/ejemploRadial.jpg}
\caption{Ejemplos de funciones de unidad de base radial, RBF.}
\label{ejemploRadial}
\end{figure}

\subsection{Redes híbridas}\label{redesHibridas}
\noindent Hasta este momento las ANNs que hemos definido poseían un solo tipo de
unidad de base como nodos de la red en la capa oculta, ya fuesen SUs, PUs o RBFs.

La combinación de diferentes funciones de base en capa oculta puede tener varias
ventajas si se consideran tareas de clasificación que tienen áreas generalmente
separadas, pero en las cuales es difícil situar su borde de decisión, debido a que las
mejores funciones discriminantes para cada clase del problema pueden ser muy diferentes.
Una mixtura de unidades de base puede proporcionar bordes de decisión flexibles,
cubriéndose las desventajas que puedan tener el uso de modelos con un solo tipo de unidad
de base o modelos puros. Donoho \cite{Donoho1989} demostró que cualquier función continua
se puede descomponer en dos tipos de funciones mutuamente excluyentes, una función
asociada con funciones de tipo proyección, SUs o PUs, y la otra asociada con funciones de
tipo ventana, RBFs. A pesar de
ello, en la práctica, es difícil separar las diferentes
localizaciones de una función y estimar la función residual mediante una aproximación
funcional basada en proyecciones, sin quedar atrapado en óptimos locales en la
minimización del error \cite{Friedman1991}.

Varios autores han propuesto la hibridación de redes con diferentes funciones de base, ya
sea con una sola capa oculta híbrida o interconectando varias capas puras. De acuerdo a
Duch y Jankowski \cite{Duch2001}, la hibridación de diferentes funciones de transferencia
dentro de una ANN se puede hacer de dos maneras. Una de ellas sería utilizar un método
constructivo que seleccione la mejor función de base a partir de un conjunto de funciones
RBF candidatas \cite{Duch2001a}; y un segundo método
sería partir de redes con diferentes unidades de base, y usar técnicas de poda o de
regularización para reducir el número de funciones \cite{Duch2001, Jankowski2001}.

Pao \cite{Pao1992} presentó una combinación de varias funciones (polinómicas, periódicas,
sigmoidales y gaussianas) en lo que llamó redes de enlace funcional (\textit{Functional
Link Networks}). La idea consistía en el uso de enlaces funcionales, añadiendo
transformaciones no lineales de las variables de entrada al conjunto original de variables,
y suprimiendo la capa oculta, haciendo una la primera aproximación no lineal fija, y una
segunda aproximación adaptativa.

Una aproximación más compleja considera varias capas o modelos, cada una conteniendo una
estructura de funciones de base, dando como resultado un sistema modular. Por ejemplo,
Iulian
\cite{Iulian2002} propuso una metodología incluyendo tres módulos distintos, implementando
una ANN hacia delante (llamada red neuronal RBF de tipo gaussiana), un proceso de
análisis de componentes principales, y una red SU. Otra
propuesta, de Lehtokangas and Saarinen \cite{Lehtokangas1998}, considera dos capas ocultas
en el modelo, la primera compuesta por funciones gaussianas y la segunda por SUs.

Las ANNs que utilizan diferentes funciones de base deberían contener
menos nodos, permitiendo que la función desempeñada por la red sea
más transparente. Por ejemplo, un hiperplano se puede utilizar para dividir
un espacio de entrada en dos clases y una función adicional de Gauss
que tenga en cuenta cualquier anomalía local. El análisis de la aproximación realizada
por una red MLP entrenada con los mismos datos no es tan simple. En este contexto cabe
señalar el trabajo de Cohen e Intrator \cite{Cohen2002a}, que se basa en las propiedades
de dualidad y complementariedad de las funciones de tipo proyección y de tipo ventana.

Recientemente, Wedge \cite{Wedge2006} presentó una ANN
híbrida compuesta por unidades de base RBFs y SUs. Se utiliza un algoritmo de
entrenamiento en tres pasos, en el cual se intentan identificar los aspectos de
una relación funcional que son expresados separadamente de aquellos que varían solo dentro
de las regiones particulares del espacio de entrada.

En \cite{Kordik2010}, se propone un algoritmo evolutivo llamado \textit{GAME} para la
optimización de la topología de una ANN, combinando diferentes tipos de neuronas,
entrenadas dentro de la red mediante varios algoritmos de optimización y principios de
meta-aprendizaje, para construir finalmente una ANN supervisada hacia delante con varias
capas y con varias funciones de base por capa.

Otros autores como Hervás \cite{Hervas2007,Hervas2007a}, hacen uso de
un AE que utiliza la función de entropía cruzada para evaluar a los
individuos de la población. Dicha población se compone de ANNs híbridas con una sola
capa oculta compuesta por unidades de tipo PUs y SUs. El algoritmo se encarga
de evolucionar simultáneamente la arquitectura y los pesos de cada red, utilizando un
operador de mutación adaptado a cada tipo de unidad de base. Los resultados
reflejan que la hibridación puede ser una alternativa viable con respecto a los
modelos puros en determinados problemas, y deja abiertas algunas cuestiones sobre
la posibilidad de crear procedimientos que permitan elegir
entre diferentes familias de funciones durante la evolución de las ANNs.
En \cite{Gutierrez2007,Gutierrez2009} se realiza una mejora del método anterior,
mediante la inclusión de un procedimiento de doble elitismo en el algoritmo evolutivo, y se da
 la posibilidad de utilizar tres tipos diferentes de redes, en función del tipo de
unidades de base usadas, redes Producto-Sigmoide (PSU), Producto-RBF (PRBF),
Sigmoide-RBF (SRBF). Algunas conclusiones de este trabajo
muestran que la elección adecuada del tipo de base a utilizar depende fuertemente del
conjunto de datos a clasificar, pero que, en general, los modelos compuestos por SUs y RBFs
consiguen un rendimiento mayor que los compuestos por SUs y PUs, especialmente en aquellos
conjuntos de datos con dos clases, donde solamente existe una función discriminante.
También se comprueba empíricamente que los modelos formados por SUs y PUs no consiguen una
precisión mayor que los restantes tipos de hibridación, aunque si consiguen tener un menor
número de conexiones, permitiendo que sean más interpretables.

A continuación, presentamos el modelo funcional o función de salida de una red
neuronal formada por diferentes tipos de unidades de base en su capa oculta (ver figura
\ref{ejemploHibrida}):
\begin{equation}\label{modelofhibrido}
f_{q}\left(\mathbf{x},\mathbf{\Theta}_{q} \right)= \alpha_{0}^q +\sum_{j=1}^{m1}
B_{j}^q \sigma_{j} \left( \mathbf{x},\mathbf{u}_{j}\right)+\sum_{k=1}^{m2}
\beta_{k}^q B_{k} \left( \mathbf{x},\mathbf{w}_{k}\right),
\end{equation}
donde $q$ es el número de unidades, nodos o neuronas de salida de la red,
$\displaystyle
\mathbf{\Theta}_{q}=\left(\alpha_{0}^q,\alpha_{1}^q,...,\alpha_{m1}^q,\beta_{0}^q,\beta_
{1}^q,...,\beta_{m2}^q,\mathbf{u}_{1},...,\mathbf{u}_{m1},
\mathbf{w}_{1},...,\mathbf{w}_{m2} \right)$ son los vectores de pesos de la neurona de
salida $q$, $\displaystyle \mathbf{u}_{j}=\left( u_{0}^j,u_{1}^j,...,u_{K}^j\right) $ y
$\displaystyle \mathbf{w}_{k}=\left( w_{0}^k,w_{1}^k,...,w_{K}^k\right) $ es el vector de
pesos de las conexiones entre la capa de entrada y la neurona oculta $j$ y $k$, $m1$ es el
número de neuronas ocultas del tipo 1 y $m2$ el el número de neuronas ocultas del tipo 2,
$K$ el número de neuronas o características en la capa de entrada,
$\mathbf{x}$ el valor de las entradas de la ANN, $\displaystyle B_{j}
\left( \mathbf{x},\mathbf{u}_{j}\right)$ es la función de base de la neurona $j$ de tipo 1
, $\displaystyle B_{k}
\left( \mathbf{x},\mathbf{w}_{k}\right)$ es la función de base de la neurona $k$ de tipo 2
y $\displaystyle \alpha_{0}^q$ y $\displaystyle u_{0}^j$ son los sesgos del modelo
asociado a la neurona $q$ en la capa de salida y a la neurona $j$ en la capa oculta
respectivamente.

\begin{figure}[htb]
\centering
\includegraphics[keepaspectratio,width=12.5cm]{figuras/redHibridaHaciaDelante.jpg}
\caption{Red híbrida hacia delante con dos tipos diferentes de funciones de base en capa
oculta.}
\label{ejemploHibrida}
\end{figure}

% COGER EL PAPER DE NEUROCOMPUTING PARA METER LA TEORIA y el MODELO FUNCIONAL DE SALIDA
% TAMBIEN COGER EL DEL MAEB ``2007_maebA.PDf
\newpage
\section{Métodos de entrenamiento}\label{metodosEntrena}
A continuación se expone brevemente una visión global de los
métodos clásicos de optimización utilizados para el entrenamiento de ANNs, y de los
métodos que incorporan algún tipo de heurística o técnica para mejorar el aprendizaje
\cite{Kordik2010}.

\subsection{Métodos clásicos}\label{mClasicos}
\noindent La mayoría de los métodos clásicos utilizados para el entrenamiento de
ANNs tienen el problema de que pueden llegar a estancarse en una
solución que no sea lo suficientemente idónea para el problema que se
esté intentando resolver. Los métodos más comunes son:
\begin{description}
\item[Métodos constructivos:] Los métodos constructivos comienzan con una red mínima y
van añadiendo nodos y conexiones hasta que ésta es capaz de resolver el problema planteado
con una determinada precisión. Lo que se hace es adaptar el tamaño de la red al problema
en cuestión, teniendo la ventaja de que no se necesita hacer una estimación “a priori” del
tamaño de la misma, pero se necesita un experto con suficiente experiencia y un proceso de
prueba y error \cite{Burgess1994,Setiono1995}. Este tipo de métodos son susceptibles
de caer en mínimos locales \cite{Angeline1994}, además de que solo pueden llevar a cabo la
búsqueda en subconjuntos reducidos del espacio de las posibles
topologías.
\item[Métodos destructivos o de poda:] Los métodos destructivos o de poda
\cite{Reed1993,Reed1999} parten de una red suficientemente grande (inicialmente puede
sobre-entrenar y ser demasiado compleja), y sucesivamente se van eliminando nodos y
conexiones hasta llegar a una topología en la que la eliminación de un nodo no mejore los
resultados. La ventaja de estos métodos es que las redes se que obtienen son pequeñas y por
lo tanto más fáciles de implementar y entrenar. De nuevo, es primordial la experiencia del
diseñador, y se necesita un proceso de prueba y error.
\item[Métodos basados en gradiente:] El algoritmo de retropropagación
(\textit{BackPropagation}, BP) \cite{Chauvin1995} y sus múltiples variantes
\cite{Hush1993,Moller1993,Chauvin1995}
(ver siguiente sección) es uno de los más carismáticos y utilizados dentro de los métodos
basados en gradiente. Este tipo de metodologías tienen como principal inconveniente el
poder caer en óptimos locales y quedar estancado el proceso de aprendizaje, por lo que es
necesario establecer
de antemano una serie de parámetros, como por ejemplo la tasa de aprendizaje, la
inicialización de los pesos (influye en la rapidez del aprendizaje y en que se alcance o
no el mínimo global de la función de error), número de capas ocultas y número de neuronas
en cada capa, entre otros.
\end{description}

Los algoritmos clásicos suelen presentar una serie de problemas \cite{Angeline1994,
Yao1999}:
\begin{enumerate}
\item Imposibilidad de calcular el gradiente cuando la función de activación no es
derivable.
\item Incapacidad de encontrar un mínimo global si la función de error es multimodal
y/o no diferenciable.
\item Ausencia de convergencia de los algoritmos clásicos de entrenamiento cuando el
número de dígitos utilizados para representar la
función de activación o los pesos de la red (precisión) no es suficientemente grande.
\item Tendencia, por parte de los algoritmos de entrenamiento, a obtener excesivas
soluciones no óptimas en cada ejecución.
\item En general, los métodos constructivos y destructivos limitan las arquitecturas
disponibles. En algunos de éstos métodos, una vez se ha explorado una arquitectura y se ha decidido
que es insuficiente, se adopta una nueva arquitectura,
de manera que las antiguas arquitecturas se convierte en inalcanzables. Además, los
algoritmos constructivos y destructivos basan su estudio en subconjuntos topológicos,
en lugar de basarse en la clase completa de las arquitecturas de red. En consecuencia,
estos	algoritmos tienden a optimizar una clase de arquitectura en lugar de ajustar una
arquitectura apropiada para un problema determinado.
\item Las deficiencias de los métodos constructivos y destructivos se derivan de
métodos 	inadecuados para la asignación de los componentes estructurales
de una red. Se asumen que las restricciones topológicas limitan la complejidad de las
interacciones	estructurales y paramétricas, e incrementan la probabilidad de
encontrar una red suficiente buena para resolver el problema, pero lo ideal sería que
esas restricciones proviniesen de la tarea o problema en si, y no que estuvieran
implícitas en el algoritmo.
\item Otros métodos usan una sola modificación estructural predefinida, como añadir una
unidad completamente conectada en capa oculta, para generar topologías sucesivas.
Tales métodos como la escalada en colina estructurada (\textit{Structural Hill
Climbing}), son susceptibles de quedar atrapados en óptimos locales estructurales, y
hacen recaer la carga de la tarea de inducción principalmente en la
identificación de los valores adecuados de los parámetros, en vez de distribuir
la carga uniformemente.
\end{enumerate}

\subsection{Métodos heurísticos}\label{redesheuristicas}
\noindent La utilización de métodos de naturaleza heurística
\cite{Walczak1999,Abraham2000,Glover2003} para el entrenamiento de ANNs surge
en la década de  los  90  como  solución a los  problemas  de  entrenamiento presentados
por los algoritmos clásicos. A continuación se citan algunos de los métodos heurísticos
utilizados con más frecuencia para el aprendizaje de ANNs:
\begin{itemize}
	\item Variantes heurísticas de la retropropagación del error
	\cite{Riedmiller1994}, BP con velocidad adaptativa y/o optimización de
	los parámetros de aprendizaje \cite{Zainuddin2005}, la regla \textit{delta-delta} o la
	regla	\textit{delta-bar-delta} \cite{Jacobs1988}, algoritmo \textit{Quikprop}
	\cite{Veitch1991}, algoritmo \textit{iRprop+} \cite{Igel2000}.
	\item Algoritmos de gradiente conjugado como el \textit{FRCG} de Fletcher-Reeves
	\cite{Fletcher1964}, el \textit{PRCG} de Polak-Ribiére  \cite{Polak1969}
	o	el	de Powell-Beale \cite{Powell1977}.
	\item Algoritmos de tipo \textit{Quasi-Newton} como el \textit{DFP} de
	Davidon-Fletcher-Powell	\cite{Davidon1991} y el
	\textit{BFGS} de Broyden-Fletcher-Golfarb-Shannon	\cite{Broyden1970}.
	\item Algoritmos que modifican el método \textit{Quasi-Newton} como el \textit{LM} de
\\
	Levenberg-Marquard \cite{Marquardt1963}.
	\item Métodos heurísticos no basados en gradiente como el enfriamiento simulado
	(\textit{Simulated Annealing} , SA), la búsqueda  tabú (\textit{Tabu Search}, TS), la
	búsqueda dispersa (\textit{Scatter Search}, SS),
	los enjambres de partículas (\textit{Particle Swarm}, PS) y los algoritmos evolutivos
	(\textit{Evolutionary Algorithms}, EAs), siendo éstos últimos los que más impacto han
	tenido en el diseño de ANNs.
	Este	tipo de heurísticas se han utilizado muy frecuentemente para el	aprendizaje en
	ANNs \cite{Kordik2010,Alba2006}.
\end{itemize}

\paginavaciacompleta
		%\paginavaciacompleta
		%\input{./capitulos/evolutivasmono-objetivo}
		%\paginavaciacompleta
		%\input{./capitulos/evolutivasmulti-objetivo}
		%\paginavaciacompleta
		\chapter{MOEA memético utilizando evolución diferencial}\label{diferencial}
\section{Introducción a la evolución diferencial}
\noindent A continuación se dará una breve introducción de la evolución diferencial
(\textit{Differential Evolution}, DE) \cite{Storn1997} para entender el algoritmo MPANN
(\textit{Memetic Pareto Artificial Neural Networks}, MPANN) de Abbass, el cual
describiremos en las siguientes secciones, y a partir del cual hemos desarrollado un
algoritmo propio basado en DE llamado MPDE (\textit{Memetic Pareto Differential Evolution}),
utilizando las medidas $(MS,C)$ descritas en el capítulo \ref{medidasRendimiento}.

La DE fue propuesta por Storn y Price \cite{Storn1997} como nueva
heurística para la minimización de funciones no lineales y no diferenciables en espacios
totalmente ordenados. La  DE  es  un  tipo  de técnica de optimización  global que  usa
selección y cruce como sus operadores primarios para la optimización
de problemas sobre dominios continuos, e incluso mutación, aunque este último operador
fue introducido posteriormente al trabajo inicial de Storn y Price, por ejemplo en
\cite{Abbass2002a}.

Los algoritmos tradicionales deterministas son insuficientes cuando se abordan problemas de
optimización que presentan características como: no linealidad, alta dimensionalidad, existencia de
múltiples óptimos locales (multimodal), no diferenciabilidad o ruido. La DE es un método alternativo
en la solución a problemas con estas características.

Dentro de las características fundamentales de la DE se encuentran \cite{Storn1997}:
\begin{itemize}
\item La capacidad de manejar funciones objetivo no lineales, no diferenciables y
multimodales.
\item La paralelización,  puesto  que el algoritmo es fácilmente  paralelizable y resulta
útil  cuando la evaluación de la función objetivo es computacionalmente costosa.
\item La  falta  de  predefinición  de  las  distribuciones  de  probabilidad,  como  en
el  caso  de  las estrategias evolutivas.
\item El uso de una codificación real y de una precisión determinada por el formato de
punto flotante empleado.
\item La  convergencia  a  un  valor  óptimo  (posiblemente  local)  de  manera
consistente  a  lo  largo  de una secuencia de ejecuciones independientes.
\item La  escasa  utilización  de  parámetros  de  control,  puesto  que    el  algoritmo
básico \cite{Storn1997} emplea  únicamente tres parámetros de control, además de un
criterio de terminación.
\item Así  mismo,  posee  una  amplia  gama  de  aplicaciones,  de  entre  las
que se  encuentran  el entrenamiento  de  ANNs,  el  diseño  de  filtros
digitales,  la optimización  de  procesos químicos no lineales, el diseño de redes de
transmisión de aguas, etc \cite{Chakraborty2008}.
\end{itemize}

La DE, como heurística evolutiva, tiene algunas características básicas que comparte con
los EAs \cite{Mezuza2008}:
\begin{itemize}
\item Es una aproximación basada en poblaciones.
\item El cruce y en ocasiones la mutación, se utilizan como operadores para generar nuevas
soluciones.
\item Hay un mecanismo de reemplazo para mantener un tamaño fijo en la población de
individuos.
\end{itemize}

La DE difiere \cite{Storn1997} trabaja creando una población inicial aleatoria de
soluciones, donde está garantizado, mediante reglas de reparación, que el valor de cada
variable esta dentro de unos límites. Entonces un individuo  es seleccionado
aleatoriamente  para  ser reemplazado  y  se seleccionan tres  padres
diferentes para formar un nuevo hijo. Uno de estos padres se elige como padre
principal
y cada alelo en el padre principal se cambia al azar, donde al menos una variable debe ser
cambiada. Esto
se lleva a cabo añadiendo al valor de las variables un valor ponderado de la diferencia
entre los dos valores
de esta variable en los otros dos padres. En esencia, el vector asociado al padre principal es
perturbado con el vector de  los  otros  dos  padres.  Esto representa  el  operador  de  cruce  en
la  DE. Si el  vector  resultante  es mejor  que  el  escogido  para  reemplazar,  se  reemplaza,
en  caso  contrario  el vector  escogido permanece en la población.

Detallado de manera más formal, y en la misma notación que propuso Storn
y Price \cite{Storn1997}, una solución $l$, en la generación $i$, es un vector
multidimensional $\displaystyle \mathbf{x}_{G=i}^l =
\left(x_{1}^l,\cdots,x_{N}^l\right)^T$. Una población, $P_{G=k}$, en la generación $G=k$ es un
vector de $M$ soluciones $(M>4)$. La población inicial, $P_{G=0}=\left\lbrace
\mathbf{x}_{G=0}^1,\cdots,\mathbf{x}_{G=0}^M\right\rbrace $se inicializa como:
\begin{gather}\nonumber
	x_{i,G=0}^l = inferior(x_{i})+aleatorio_{i}[0,1]\cdot(superior(x_{i})-inferior(x_{i})), \nonumber
\\
		l=1,\cdots,M, \quad i= 1,2,\cdots,N, \nonumber
\end{gather}
donde $M$ es el tamaño de la población, $N$ es la dimensión de la solución, y cada variable
$i$ en un vector de soluciones $l$ en la generación inicial $\displaystyle G=0,x_{i,G=0}^l$, se
inicializa dentro de los límites $\displaystyle (inferior(x_{i}),superior(x_{i}))$. La selección
se lleva a cabo seleccionando 4 soluciones con índices diferentes (3 padres más la solución a
reemplazar), $r^1, r^2, r^3$ y $j\in [1,M]$. Los valores de cada variable en el hijo se cambian
mediante un operador de cruce con una probabilidad $CR$, de la siguiente manera:
\begin{equation}
				\forall i \leq N,x_{i,G=k}^{'}= \left\lbrace
				\begin{array}{ll}\nonumber
			   x_{i,G=k-1}^{r^3} + F\cdot (x_{i,G=k-1}^{r^1} - x_{i,G=k-1}^{r^2}) & \\
			   \mbox{si $(aleatorio[0,1)< CR \wedge i=i_{aleatorio})$} \\
            x_{i,G=k-1}^{j} \quad \mbox{en otro caso}
            \end{array}
            \right.
\end{equation}
donde $F\in(0,1)$ es un parámetro que representa la cantidad de perturbación añadida al padre
principal ($r^3$ en este caso). La nueva solución reemplaza a la antigua seleccionada si es mejor
que ella, y al menos una de las variables se debe cambiar. Esta última está representada en el
algoritmo de forma aleatoria, seleccionando una variable, $\displaystyle i_{aleatorio}\in(1,N)$.
Después del cruce, si una o más de las variables en la nueva solución se encuentran fuera de su
límite inferior o superior, se aplica la siguiente regla de reparación:
\begin{equation}
				x_{i,G=k}= \left\lbrace
				\begin{array}{ll}\nonumber
			   \frac{x_{i,G}+inferior(x_{i})}{2}  \quad \text{si $x_{i,G+1}^j < inferior(x_{i})$} \\
            inferior(x_{i}) + \frac{x_{i,G}+superior(x_{i})}{2} \quad \text{si $x_{i,G+1}^j>
superior(x_{i})$} \\
            x_{i,G+1}^j \quad \text{en otro caso}
            \end{array}
            \right.
\end{equation}

Una vez comentados los detalles esenciales de la DE, decir que la DE está restringida a
dominios donde el espacio de búsqueda está completamente
ordenado y en especial subespacios de $\Re^n$. Un individuo se representa como una
$n$-tupla, llamada vector objetivo $\mathbf{x}=(x_{1},\cdots,x_{n})$, donde $x_{i} \in \Re
(i=1,\cdots,n)$ son los valores escalares que representan las variables de diseño del
problema. Emplear una codificación
real permite generar perturbaciones acotadas en las variables de diseño, a consecuencia
del orden determinado por $\Re^n$. La codificación real de la DE, a diferencia las
estrategias de evolución (\textit{Evolution Strategies}, ES) no utiliza una distribución
fijada (como la
distribución Gaussiana fijada en ES) para controlar el comportamiento del operador de
mutación, en lugar de ello, la distribución de las soluciones en el espacio de búsqueda
determina el tamaño de paso y la dirección de búsqueda para cada individuo.

El lector puede consultar una revisión completa y detallada sobre DE y sobre aplicaciones
en \cite{Price2005,Chakraborty2008,Mezuza2008,Iorio2006} y en un reciente trabajo de
Ferrante Neri y Ville Tirronen en \cite{Neri2010}.

\subsection{Variantes de la evolución diferencial}\label{variantes}
\noindent Hay algunas variantes \cite{Mezuza2008} del algoritmo básico de DE, que se
diferencian en:
\begin{itemize}
	\item El tipo de criterio para seleccionar uno de los individuos a usar en el operador
de cruce como padre principal.
	\item El número de diferencias computadas en la operación de cruce.
	\item El operador de cruce escogido.
\end{itemize}
La variante más popular se llama ``\textit{DE/rand/1/bin}``, donde DE se refiere al
nombre del algoritmo, \textit{rand} indica una elección aleatoria de los vectores que
conforman el vector de mutación, la cifra \textit{1} señala el número de pares de vectores
que conforman la diferencia y \textit{bin} establece un proceso de cruce binomial. Así,
por ejemplo, un algoritmo de DE que selecciona aleatoriamente a cuatro vectores (2 pares) que
componen al vector de mutación, y que son recombinados con un proceso exponencial, se
representa como ''\textit{DE/rand/2/exp}''. La figura \ref{diferencial1} muestra
gráficamente el cruce binomial y exponencial.

Además de los parámetros típicos de los EAs, la DE adopta dos nuevos parámetros: $CR$,
que representa una probabilidad de cruce, normalmente entre $\left[ 0,1\right] $ y
controla la influencia de los padres en la generación de los hijos, y $F$, un parámetro
que regula las magnitudes relativas de las diferencias del vector mutación, y que suele ser
un valor aleatorio obtenido de una distribución normal $N(0,1)$ o uniforme $U(0,1)$. Estos
parámetros
dependen en cierta medida de las características de las funciones objetivo y del tamaño
de la población.

\begin{figure}[!htp]
\centering
\includegraphics[keepaspectratio,width=11.5cm]{figuras/crucesBinonialExponencial.jpg}
\caption{Cruce discreto binomial (izquierda) y cruce discreto exponencial (derecha). Los
individuos $\mathbf{x}$ e $\mathbf{y}$ dan lugar a otro individuo $\mathbf{z}$.}
\label{diferencial1}
\end{figure}

\newpage
Definimos a continuación las variantes de la DE (la figura \ref{diferencial2} muestra un resumen de
las principales variantes):
\begin{itemize}
	\item Variantes con operador de cruce discreto (binomial o exponencial):
		\begin{itemize}
			\item \textit{DE/rand/1/bin}
			\item \textit{DE/rand/1/exp}
			\item \textit{DE/best/1/bin}
			\item \textit{DE/best/1/exp}
	   \end{itemize}
		Las variantes con \textit{rand} seleccionan el padre principal y un
		par de padres secundarios para calcular la mutación diferencial aleatoriamente. En
		contraste, las	variantes con \textit{best} utilizan la mejor solución
		de la población	como padre principal, y un par de individuos seleccionados
		aleatoriamente como padres		secundarios.
	\item Variantes con cruce aritmético:
		\begin{itemize}
			\item \textit{DE/current-to-rand/1}
			\item \textit{DE/current-to-best/1}
	   \end{itemize}
		La diferencia entre ellos es que el primero selecciona el padre principal y los
		padres secundarios de manera aleatoria en la población actual, mientras que el
		segundo utiliza la mejor solución de la población actual como padre principal, y los
		padres secundarios se eligen aleatoriamente.
			\item Variantes con cruce combinado aritmético discreto (similar a las anteriores pero
		utiliza cruce binomial):
		\begin{itemize}
			\item \textit{DE/current-to-rand/1/bin}
	   \end{itemize}
\end{itemize}
\newpage
\begin{figure}[!htp]
\centering
\includegraphics[width=13.2cm,height=9cm]{figuras/modelosDE.jpg}
\caption{Principales modelos de DE. $p$ es el número de pares de vectores que
conforman la diferencia, $j_{r}$ es un valor aleatorio generado en el
intervalo $\left[ 0,n\right]$, donde $n$ es el número de variables del problema.
$x_{r3}$ es el padre principal y $x_{r1}$ y $x_{r2}$ los padres
secundarios. $F$ y $K$ son valores de
escala, $u_{i}$ es el hijo creado y $x_{best}$ significa que se ha seleccionado como padre
principal el mejor individuo o solución de la población en una determinada generación,
\textit{bin} representa cruce binario y \textit{exp} cruce exponencial, y \textit{\textbf{dir}}
indica que se incluye información de alguna función de aptitud al cruce y a la mutación.}
\label{diferencial2}
\end{figure}
% \section{Evolución diferencial en redes neuronales artificiales mono-objetivo}
% \noindent Ning Guiying en \cite{Ning2007} propone un algoritmo basado en DE y llamado
%MDE
% (Modified Differential Evolution), en el que se optimizan los individuos iniciales con
%la
% regla $1/2$, introduciendo posteriormente una reorganización de la estrategia de
%evolución
% durante el período mutación. El MDE se utiliza para optimizar los pesos de ANNs
%multicapa
% hacia delante. MDE se compara con el algoritmo BP y con el algoritmo básico de DE,
% mostrando en los resultados finales que MDE tiene una alta calidad de convergencia
% global y mejora la precisión y la velocidad de convergencia de las ANNs.

% En \cite{Lahiri2008}, se utiliza un algoritmo de DE con ANNs, llamado ANN-DE, para
% optimizar reactores industriales catalíticos de óxido de etileno. En el proceso
% evolutivo del algoritmo se construye un modelo de ANN para correlacionar los datos del
% proceso que compone los valores de funcionamiento y de rendimiento de variables del
% reactor. Para la optimización de los pesos de la red se utiliza una algoritmo de
% gradiente y un conjunto de patrones de entrenamiento y generalización con tres
%variables, % que son obtenidas de la monitorización de una serie de reactores. Una vez se
%construye el % modelo de red se genera una serie de vectores aleatorios como población
%del proceso % evolutivo formados por posibles valores de las tres variables de entrada de
%los % reactores, es decir, se obtienen un número de conjuntos de condiciones de
%funcionamiento % que intentan maximizar la producción del reactor al aplicarlos a la ANN
%diseñada, la cual % muestra qué vectores de solución son mejores. Al final del proceso
%evolutivo, que posee % tanto cruce como mutación, se obtiene un conjunto de soluciones
%optimizadas  que se % vuelven a aplicar a la ANN para escoger finalmente la mejor.
\section{Evolución diferencial para optimización multi-objetivo utilizando el concepto de
dominacia de Pareto}
\noindent Para aplicar la estrategia de la DE a problemas multi-objetivo, hay que
modificar el esquema original \cite{Storn1997}, ya que el conjunto de soluciones de un
problema con múltiples objetivos no consiste en una sola solución (ver sección
\ref{multiobjetivo} del capítulo \ref{MOEANNs}).

Hay varios aspectos que se deben considerar para extender la DE a un problema de
optimización multi-objetivo:
\begin{itemize}
	\item ¿Cómo promover la diversidad de la población?
	\item ¿Cómo seleccionar o retener los mejores individuos, es decir, cómo realizar
	elitismo?
\end{itemize}

Para promover la diversidad hay que tener en cuenta el proceso de selección por medio de
mecanismos basados en alguna medida de calidad, la cual indique la cercanía entre los
individuos que forman la población. Las dos medidas de diversidad más usadas en
optimización multi-objetivo son:
\begin{description}
	\item[Distancia \textit{crowding} \cite{Deb2002}:] Esta medida da una idea de cómo de
agrupados están los vecinos de un determinado individuo en el espacio de la función
objetivo y hace que los frentes sean lo más uniformes posible, sin agrupar muchos
individuos en una
zona y dejar ninguno o muy pocos en otras. La distancia \textit{crowding} se estima en función
de la media de las caras de un cubo formado al tomar como vértices los vecinos más
cercanos a un
individuo $i$ (ver figura \ref{figuraCrowding} y el trabajo de K. Deb \cite{Deb2002} para más
información).
	\item[Compartición de aptitud o \textit{fitness sharing}:] Cuando un individuo
comparte valores de su función de aptitud con otros, ésta se degrada en proporción al número y
a la
proximidad de los individuos que lo rodean dentro de un determinado perímetro. La vecindad
de un individuo se define en términos de un parámetro llamado $\sigma_{share}$, que
indica el radio de la vecindad, a la cual se le llama nicho.
\cite{Golberg1989,Sareni1998}.
\end{description}

El lector puede obtener en \cite{Du2010} un estado del arte actualizado en métodos de
agrupamiento para obtener diversidad en ANNs.

\begin{figure}[!htp]
\centering
\includegraphics[keepaspectratio,width=8cm]{figuras/CrowdingDistance.jpg}
\caption{Cálculo de la distancia \textit{crowding}. Los puntos azules son soluciones de un mismo
frente.}
\label{figuraCrowding}
\end{figure}

Para promover el elitismo en optimización multi-objetivo se suele utilizar un archivo
externo, llamado población secundaria, que almacena los individuos no dominados
encontrados a lo largo de la búsqueda. Uno de los métodos más populares para seleccionar
los mejores individuos de una población formada por padres e hijos es la ordenación de
no dominados. Esta técnica se basa en el mecanismo de orden que se le da a los diferentes
individuos de una población en forma de niveles. Por ejemplo, en el nivel 1 estarán los
individuos no dominados. En el segundo nivel, estarán los individuos no dominados si no
se tienen en cuenta los del primer nivel, y así sucesivamente. Según Goldberg
\cite{Golberg1989}, para mantener una diversidad apropiada, la metodología de ordenación
de no dominados se debería usar en conjunción con alguna técnica de nichos como las
mencionadas anteriormente. El algoritmo NSGAII \cite{Deb2002} es un claro ejemplo de esto.

Existen varios trabajos interesantes que utilizan DE con MOEAs y aplicaciones en
\cite{Price2005,Chakraborty2008,Mezuza2008}.

\section{Evolución diferencial de Pareto con MOANNs}
\noindent En primer lugar vamos a considerar brevemente los artículos más interesantes
sobre el uso de DE para el diseño de ANNs mediante técnicas multi-objetivo basadas en el concepto de
de dominancia y óptimo de Pareto, y en los cuales nos hemos basado para la construcción
de un
nuevo MOEA basado en DE para el diseño de ANNs con unidades de base sigmoide.

El referente principal de este tipo de aplicaciones, usando una variación del algoritmo de
DE original \cite{Storn1997}, es H. Abbass, creador del algoritmo multi-objetivo PDE
(\textit{Pareto Differential Evolution}) \cite{Abbass2001-1,Abbass2002a}. En el algoritmo
PDE se utiliza un caso especial de la variante \textit{DE/current-to-rand/1/bin} con $K=0$
(ver penúltima fila de la figura \ref{diferencial2}), ya que el padre principal se utiliza
para la creación de un nuevo hijo y también para un tipo de cruce discreto. Los objetivos a
minimizar son el $MSE$ y la complejidad de la red.

El algoritmo trabaja como sigue: La población inicial se inicializa utilizando una distribución
Gaussiana de
media $0,5$ y de desviación típica $0,15$. Solamente las soluciones no dominadas se
retienen en la población para el cruce y las dominadas se eliminan. Se seleccionan tres
padres de manera aleatoria (uno de ellos como padre principal y también como solución
hija) para generar un nuevo hijo. Lo hijos se incluyen en la población solo si dominan al
padre principal, en caso contrario, se hace un nuevo proceso de selección. Si el número de
soluciones no dominadas excede un umbral, se adopta una métrica de distancia para eliminar
padres que están muy cercanos unos de otros (esto se puede ver como un procedimiento de
nichos en el cual la métrica de distancia es el radio del nicho). En esta aproximación, el
tamaño de paso $F$ se genera a partir de una distribución Gaussiana $N(0,1)$, y las
restricciones de frontera se preservan, ya sea mediante un cambio de signo si la variable
es $\leq 0$ o mediante restas repetitivas, restando 1 si es $\geq 0$, hasta que la
variable esté dentro de los límites permitidos. El algoritmo PDE también incorpora un operador de
mutación que se aplica con una determinada probabilidad, después del operador de cruce,
mediante la suma a cada variable de una pequeña perturbación aleatoria.

A partir de la aparición del algoritmo PDE, y casi en paralelo, Abbass desarrolla el algoritmo
MPANN (\textit{Memetic Pareto Artificial Neural Networks}) \cite{Abbass2001}, que es una
versión del PDE añadiéndole un algoritmo de LS basado en gradiente como es BP, con algunas
mejoras, para así aumentar la velocidad de convergencia. MPANN trata de obtener modelos
de ANNs que tengan buena capacidad de generalización sin aumentar demasiado el tamaño de
su arquitectura. Concretamente trata  de  minimizar  el $MSE$ y el número  de neuronas en  capa
oculta. MPANN evoluciona conjuntamente
la arquitectura y los pesos de la red y utiliza operadores cruce y mutación para la obtención de
los hijos, codificando cada ANN  en  un cromosoma  que  representa  la
estructura  y el  valor  de  los pesos. Otras  versiones  de MPANN  se
utilizan para problemas reales como la diagnosis del cáncer \cite{Abbass2002a},
diferenciándose del MPANN original en la incorporación de un operador de mutación, ya que
los algoritmos PDE y MPANN carecían de ello.

Otra variación del PDE es el algoritmo SPDE (\textit{Self-adaptive Pareto Differential
Evolution}) \cite{Abbass2002}, la cual adapta de manera automática las probabilidades de
cruce y mutación. Ambas probabilidades se heredan de los padres de la misma forma que se
realiza el cruce para las variables de decisión. Si las probabilidades de cruce y de mutación
no están entre $(0,1)$, se modifican automáticamente de acuerdo a unas reglas de
reparación. Al igual que se realizó una versión auto-adaptativa del PDE con el algoritmo
SPDE, Abbass propuso una versión auto-adaptativa del algoritmo MPANN, llamada SPANN
(\textit{Self-adaptive Pareto Artificial Neural Networks}) \cite{Abbass2003}.

Un algoritmo que se debe mencionar a pesar de que sea un algoritmo evolutivo
mono-objetivo con término de regularización es el de J. Illonen \cite{Jarno2003}, donde
se propone un estudio de la DE en el diseño de ANNs para encontrar el óptimo global de
un problema. El algoritmo utiliza el $MSE$ medio regularizado mediante la media de los pesos y los
sesgos para entrenar a las ANNs. Illonen compara su metodología con métodos basados en gradiente y
concluye que la DE puede ser más útil en el caso especial de algunas superficies de
error, pero que la inclusión de alguna metodología híbrida que utilice conjuntamente
optimización evolutiva e información basada en gradiente podría ser más beneficiosa.

En \cite{Yau2007}, se proponen una serie de experimentos utilizando el algoritmo
multi-objetivo PDE de Abbass para evolucionar ANNs aplicadas a juegos de inteligencia
artificial. La metodología propuesta contienes tres subsistemas: Un subsistema PDE canónico, un
subsistema que introduce coevolución en PDE con tres
configuraciones posibles (PCDE), y un subsistema también de coevolución con PDE usando un
archivo con tres configuraciones diferentes (PCDE-A). El primer subsistema se trata del algoritmo
PDE con operadores de cruce y mutación y sin LS. El segundo sistema introduce coevolución,
siendo la evaluación de cada individuo la principal diferencia con el algoritmo PDE. Para la
ejecución
del PCDE, cada ANN se compara con un número constante de ANNs elegidas al azar de la
población de la generación actual. Si la puntuación de la ANN es mayor o igual a la de su
oponente (elegidas al azar), recibirá una victoria. Por otra parte, la clasificación del
primer frente de Pareto (mediante el etiquetado de soluciones no dominadas) se basa en el
	número de victorias como criterio de evaluación principal. En el algoritmo PDCE-A, al
igual
que en PCDE, después de la puntuación de cada ANN se lleva a cabo un segundo computo.
Sin embargo, PCDE-A tiene un archivo extra que se utiliza para almacenar las soluciones
de Pareto cada 50 generaciones. En consecuencia, cada ANN se compara con un número
mínimo de ANNs elegidas al azar (sin repeticiones), a partir del archivo extra. Sólo si el
número de ANNs en el archivo es menor que el número mínimo exigido de oponentes elegidos al
azar, entonces la lista de oponentes se completa de ANNs elegidas al azar de la
población. Del mismo modo, una ANN recibirá una victoria si su puntación es mayor o
igual que la de su competidora. El número de victorias se utilizará como
criterio de evaluación principal para etiquetar las soluciones no dominadas para
el primer frente de Pareto. Este valor de evaluación será menor al depender del
etiquetado de las soluciones dominadas, a causa del conjunto acotado de
evaluadores. De la experimentación realizada con una serie de juegos de inteligencia
artificial, se concluye que los mejores resultados los obtiene el algoritmo PDE. El pobre
desempeño de los sistemas PCDE, incluso la versión que utilizan un archivo extra, en la
producción de una buena distribución de soluciones a lo largo del frente de Pareto, es una
prueba más de que los métodos co-evolutivos no son especialmente beneficiosos para la
síntesis de agentes inteligentes para juegos en la evolución de Pareto.

En \cite{Iorio2004} se propone un algoritmo llamado NSDE (\textit{Non-dominated Sorting
Differential Evolution}), que consiste en una modificación del algoritmo NSGAII
\cite{Deb2002}, al que se introduce DE con la variante \textit{DE/current-to-rand/1} (ver
sección \ref{variantes}) en el cruce y la mutación. NSDE se utiliza para resolver problemas
de rotación de funciones en el plano, atendiendo a dos funciones objetivo, una para cada
eje de coordenadas del plano basándose en los grados de rotación. El algoritmo NSDE se
compara con NSGAII en una serie de problemas de rotación, mostrando mejores
soluciones por el proceso de cruce y mutación realizado en este tipo de problemas.

A continuación exponemos algunos de los caminos futuros de la DE usando MOEAS.

\section{Caminos futuros en la evolución diferencial multi-objetivo}
\noindent Según se puede ver en \cite{Mezuza2008}, la DE debe tener en cuenta estas
futuras
mejoras:
\begin{description}
	\item[Diversidad:] A pesar de que la DE tiene una alta convergencia, no posee
suficiente robustez y tiene problemas para alcanzar el verdadero frente de Pareto,
pudiendo quedar atrapada en óptimos locales. Además parece que la DE tiene problemas para
crear un frente de Pareto homogéneo, con lo que se deberían aplicar alternativas de
diversidad cuando se utilice con problemas multi-objetivo.
	\item[Variantes:] A día de hoy no se sabe que variante de DE es mejor para problemas
multi-objetivo para alcanzar el verdadero frente de Pareto de menera más efectiva.
	\item[Operador de mutación:] Se deben tomar algunos nuevos criterios a la hora de
seleccionar pares de soluciones en el proceso de mutación que sean más efectivos
\cite{Iorio2006}. En este momento estamos estudiando la selección de soluciones utilizando
intervalos  de confianza asociados a las distribuciones de los mejores individuos de la población
\cite{Cruz2010}.
	\item [Adaptación de los parámetros:] Debe haber nuevas propuestas que no sean
solamente las autoadaptativas \cite{Abbass2002,Abbass2003} para optimizar los parámetros
CR y F.
	\item[Alternativas en la codificación:] DE se propuso para espacios se búsqueda
continuos, por lo que se debería buscar una alternativa de codificación que permita el
uso de la DE en problemas de optimización combinatoria.
	\item[Teoría:] Los estudios sobre la convergencia de las variantes de DE y análisis en
tiempo de ejecución, mejorarían la teoría actual.
 \end{description}

\section{El algoritmo MPANN}
\noindent A continuación exponemos el algoritmo MPANN \cite{Abbass2001} en su versión
adaptada al reconocimiento de diagnosis del cáncer \cite{Abbass2002a}. Ésta versión
se diferencia del algoritmo MPANN original en que incorpora un operador de mutación, del
cual
carecían los algoritmos PDE y MPANN originales. Además, tendremos en cuenta el algoritmo
NSGAII \cite{Deb2002}, que nos servirá de base para la creación de nuestro algoritmo MPDE para el
diseño de ANNs en multi-clasificación de patrones.

En la figura \ref{diferencial3}, se presenta el pseudocódigo del algoritmo
MPANN que comentamos a continuación:

El primer paso de MPANN es generar una población inicial al azar siguiendo una
distribución Gausiana $(0,1)$ (Paso 1).

Acto seguido, el algoritmo comienza su proceso evolutivo:

La primera acción (Paso 4) que tiene lugar al comienzo de una generación es la evaluación
de los individuos, en el caso de MPANN utilizando la minimización del $MSE$ y la
complejidad
de la red (número de neuronas en capa oculta). Una vez evaluados, se etiquetan los individuos no
dominados.

Si el número de individuos no dominados es menor que 3, se busca un individuo no dominado
de entre las soluciones que no están etiquetadas y se etiqueta como no dominado. Esto se
repite hasta que el número de no dominados sea igual a 3 (Paso 5).
\newpage
\begin{figure}[!htp]
\centering
\fbox{
	\includegraphics[keepaspectratio,width=12.2cm]{figuras/etapasMPANN.jpg}
}
\caption{Pseudocódigo del algoritmo MPANN.}
\label{diferencial3}
\end{figure}

Seguidamente (Paso 6), se eliminan todas las soluciones dominadas de la población.

A continuación (Paso 7), se marca un 20\% del conjunto de patrones de entrenamiento como
conjunto de validación para la LS.

El siguiente paso (Paso 8) es el más importante de todo el algoritmo, ya que se trata de
la generación de hijos. Este paso se repetirá hasta que la población alcance un tamaño
máximo, $M$, fijado de antemano.

La primera acción a realizar (Paso 8.1) es la selección aleatoria de tres
padres. Uno de ellos se etiqueta como padre principal $(\alpha_{1})$ y los otros dos
como secundarios $(\alpha_{2},\alpha_{3})$.

A continuación (Paso 8.2), tiene lugar la operación de cruce. En la operación de cruce se
calcula aleatoriamente una probabilidad uniforme en el intervalo $(0,1)$ para cada una de
las
neuronas de la capa oculta en conexión con la capa de entrada. Si el valor obtenido es menor que el
valor de $CR$ (\textit{Crossover Probability}), se aplican las expresiones (\ref{1}) y
(\ref{2}), donde  $w_{ih}^{hijo}$ se refiere, en el hijo, a los pesos asociados a cada una de las
neuronas de entrada, $i$, con cada una de las neuronas de la capa oculta, $h$. $\rho_{h}^{hijo}$ se
refiere a la existencia o no en la capa oculta de las neuronas del nuevo hijo. Todo este
proceso se hace para cada neurona de la capa oculta en conexión con la capa de entrada, siendo el
número máximo de neuronas ocultas prefijado al inicio del algoritmo.
\begin{equation}\label{1}
w_{ih}^{hijo} \leftarrow w_{ih}^{\alpha_{1}} +
N(0,1)(w_{ih}^{\alpha_{2}}-w_{ih}^{\alpha_{3}})
\end{equation}
\begin{equation}\label{2}
 \rho_{h}^{hijo} \leftarrow \left\lbrace
 \begin{array}{ll}
 1 & \mbox{si
$\rho_{h}^{\alpha_{1}}+N(0,1)(\rho_{h}^{\alpha_{2}}-\rho_{h}^{\alpha_{3}})\geq 0.5$} \\
 0 & \mbox{en otro caso}
 \end{array}
 \right.
\end{equation}

En el caso de que el valor aleatorio obtenido en el intervalo $(0,1)$ no sea menor que $CR$, se
aplican las expresiones (\ref{3}) y (\ref{4}).
\begin{eqnarray}
w_{ih}^{hijo} \leftarrow w_{ih}^{\alpha_{1}} \label{3} \\
\rho_{h}^{hijo} \leftarrow \rho_{h}^{\alpha_{1}} \label{4}
\end{eqnarray}

Una vez finalizada la operación de cruce para todas las neuronas de la capa oculta en
conexión
con la capa de entrada, es el turno de las neuronas de capa oculta en conexión con la capa de
salida. De nuevo se obtiene un valor aleatorio probabilístico uniforme en el intervalo $(0,1)$, y si
el valor obtenido es menor que el valor de $CR$, se aplica la expresión (\ref{5}), donde
$w_{ho}^{hijo}$ significa el peso asociado a la conexión que
va desde la neurona $h$ de la capa oculta hasta la neurona $o$ de la capa de salida  del nuevo hijo.
Esto se hace, al igual que antes, para todas las neuronas que haya en capa oculta en conexión con la
capa de salida.
\begin{equation}\label{5}
w_{ho}^{hijo} \leftarrow w_{ho}^{\alpha_{1}} +
N(0,1)(w_{ho}^{\alpha_{2}}-w_{ho}^{\alpha_{3}})
\end{equation}

En el caso de que el valor aleatorio obtenido en el intervalo $(0,1)$ no sea menor que $CR$, se
aplica la siguiente expresión:
\begin{equation}\label{6}
w_{ho}^{hijo} \leftarrow w_{ho}^{\alpha_{1}}
\end{equation}

Durante la operación de cruce, al menos una variable del hijo se debe modificar para que
sea distinto al padre principal.

A continuación se realiza la operación de mutación (Paso 8.3), calculándose
una probabilidad uniforme $(0,1)$ para cada una de las neuronas de capa oculta del hijo, tanto
las que están en conexión con la capa de entrada como las que están en conexión con la capa de
salida. Si el valor obtenido es menor que el valor de $MR$ (\textit{Crossover Probability}), se
aplican las expresiones (\ref{7}) y (\ref{9}), o la expresión (\ref{8}) si el cambio que se va a
producir es para una conexión de la  capa de oculta con la capa de entrada, o si
es para una conexión de la  capa de oculta con la capa de salida respectivamente. Si el valor
aleatorio obtenido es mayor que $MR$, no se hace la mutación. La nomenclatura que se sigue es la
misma que para la operación de cruce.
\begin{eqnarray}
w_{ih}^{hijo} \leftarrow w_{ih}^{hijo} +
N(0,\text{porcentaje mutación}) \label{7} \\
w_{ho}^{hijo} \leftarrow w_{ho}^{hijo} +
N(0,\text{porcentaje mutación}) \label{8}
\end{eqnarray}
\begin{equation}\label{9}
\rho_{h}^{hijo} \leftarrow \left\lbrace
\begin{array}{ll}
1 & \mbox{si
$\rho_{h}^{hijo}=0$} \\
0 & \mbox{en otro caso}
\end{array}
\right.
\end{equation}

Una vez realizadas las operaciones de cruce y mutación, se aplica al hijo resultante la LS (Paso
8.4), usando el algoritmo de retropropagación BP, aplicando el
conjunto de validación del paso 7.

Este hijo se evalúa y se añade a la población si presenta una relación de dominancia
con respecto al padre principal (Paso 8.5 y Paso 8.6)

Este proceso se repetirá a lo largo de las generaciones hasta que se cumpla la condición
de parada (Paso 10).

\section{El algoritmo MPDE}
\noindent A continuación exponemos nuestro algoritmo basado en DE para multiclasificación de
patrones usando el concepto de dominancia de Pareto. Dicho algoritmo se llama MPDE
(\textit{Memetic Pareto Differential Evolution}) \cite{Fernandez2009}. Nuestro procedimiento, al
igual que el algoritmo MPENSGAII descrito en el
capítulo \ref{MOEANNs}, evoluciona simultáneamente los pesos y la arquitectura de la red,
y se encarga de diseñar modelos de red para multi-clasificación de patrones.

MPDE obtiene diferentes conjuntos de clasificadores no dominados que presentan un buen
balance entre precisión y $MS$ (ver capítulo
\ref{medidasRendimiento}), que son los dos objetivos a optimizar.

La población de individuos está sujeta a operaciones de cruce y de mutación. En cuanto a la
codificación de los modelos de red, se sigue la misma codificación explicada en los algoritmos CBFEP
y MPENSGAII del capítulo \ref{evoMonoObjetivo} y \ref{MOEANNs} respectivamente.

MPDE está hibridado con un algoritmo de LS, y utiliza funciones de base sigmoides (SUs).

Para llevar a cabo un proceso de elitismo, MPDE se basa en algunos aspectos de NSGAII
\cite{Deb2002}, utilizando como característica más representativa el ordenamiento rápido
de no-dominados para la obtención del frente de Pareto.

Con respecto a la diversidad se utiliza la distancia \textit{crowding} de NSGAII, para el
caso en que haya que completar la población hasta alcanzar un número determinado de
individuos, y también una ecuación de cálculo de la distancia  de un individuo a los dos
vecinos más cercanos, la cual comentaremos en las siguientes secciones.

En cuanto a la variante de la DE, se trata de la variante \textit{DE/rand/1/bin}.
\newpage
\subsection{Funciones objetivo}
\indent Las funciones objetivo a optimizar con MPDE son las mismas que se utilizan con
MPENSGAII, $E$ y $MS$, ya que pensamos que un buen
clasificador
debería obtener un alto nivel de
precisión global, así como un aceptable nivel de clasificación para cada clase de un determinado
problema:
\begin{itemize}
\item \textbf{Objetivo 1:} La mínima sensibilidad de todas las clases de un problema, $MS$:
\begin{displaymath}
A_{1}\left( g,\mathbf{\Theta}\right) = MS(g)
\end{displaymath}
\item \textbf{Objetivo 2:} La entropía cruzada como medida de error global, $E$. Concretamente, como
función
de aptitud a la hora de evaluar un individuo se usará la siguiente expresión:
\begin{displaymath}
\label{aptitudNNEP}
A_{2}\left( g,\mathbf{\Theta}\right) =\frac{1}{1+E\left( g,\mathbf{\Theta}\right)},
\end{displaymath}
es decir, maximizar una transformación estrictamente decreciente de $E$.
\end{itemize}

A la hora de asignar una clase a una nueva observación se sigue el esquema ``1 de $Q$''
explicado en los algoritmos CBFEP y MPNESGAII.

\subsection{Operadores}
\noindent En cuanto a los operadores de cruce y mutación se utilizarán los mismos que los
del algoritmo MPANN, pero adaptados a nuestra representación.

Las expresiones (\ref{1}) a (\ref{6})
representan el operador de cruce y las expresiones (\ref{7}) a (\ref{9}) el
operador de mutación. El significado de la nomenclatura seguida es la misma que la
explicada en el algoritmo MPANN.

\subsection{Búsqueda local}
\noindent Como algoritmo de búsqueda local usamos el algoritmo iRprop+ utilizado con el
algoritmo MPENSGAII (ver sección \ref{rprop} del capítulo \ref{MOEANNs}). En la siguiente
sección se explica detalladamente cuándo se utiliza el algoritmo iRprop+ y se comenta
cada una de las etapas de nuestra metodología.

\subsection{Etapas y aspectos relevantes de MPDE}
\noindent En la figura \ref{diferencial3} se muestran las etapas del algoritmo MPDE, que
pasamos a comentar:

\begin{figure}[!htp]
\centering
\fbox{
	\includegraphics[keepaspectratio,width=12.2cm]{figuras/etapasMPDE.jpg}
}
\caption{Pseudocódigo del algoritmo MPDE.}
\label{diferencial4}
\end{figure}

El algoritmo comienza con la creación de una población de individuos tomados al azar,
siendo $M$ el tamaño de la población (Paso 1).

Empieza el proceso evolutivo hasta que se cumpla la condición de parada (Paso 3). Los
individuos se evalúan en base a las dos funciones objetivo que guían el algoritmo y se
realiza un ordenamiento rápido de no dominados equivalente al del algoritmo NSGAII
\cite{Deb2002}. Se etiquetan entonces aquellos que sean no dominados. (Paso 4)

Si el número de soluciones no dominadas es menor que 3 (Paso 5), entonces se repite el siguiente
proceso hasta que haya al menos 3 soluciones: Encontrar una solución no dominada
entre las que no están etiquetadas, en función del orden asignado en el ordenamiento
rápido de no dominados. En caso de empate en orden, se utiliza el valor de la distancia
\textit{crowding} de NSGAII (Paso 5.1) y se elige la solución con mayor distancia. A
continuación se
etiqueta el individuo escogido como no dominado (Paso 5.2).

Si el número de soluciones no dominadas ya era de al menos 3 individuos, se comprueba si
el número de soluciones es mayor que $M/2$ (Paso 6). Si es así, se calcula la distancia de
cada individuo a sus dos vecinos más cercanos (Paso 6.1), y se elimina el individuo con
menor distancia (Paso 6.2). El cálculo de la distancia viene dado por:
\begin{displaymath}
D(x)=\frac{(min\|x-x_{i}\|+min\|x-x_{j}\|)}{2}
\end{displaymath}
siendo $x$ el individuo al que se le va a calcular la distancia a sus dos vecinos más
cercanos, siendo estos $x_{i}$ y $x_{j}$. Esto nos permite mantener un mayor grado de
diversidad y que el algoritmo no quede estancado en el caso de que el número de
soluciones no dominadas sea cercano a $M$, ya que el tamaño de la población suele ser
pequeño, con lo que obtendríamos de una generación a otra muy pocos
individuos mejorados.

En el paso 7 se eliminan de la población actual todas las soluciones no etiquetadas, es
decir, las soluciones dominadas.

Se pasa ahora a completar la población actual para prepararla para la siguiente
generación hasta que el número de individuos sea igual a $M$ (Paso 8).

Primero se selecciona aleatoriamente un individuo como padre principal y otros dos como
padres secundarios, todo ello sin repetición, para que los 3 sean distintos (Paso 8.1).

A partir de esos tres individuos realizamos una serie de operaciones hasta
obtener un nuevo hijo que se añada a la población. En primer lugar aplicamos el operador
de cruce a partir del padre principal y a partir de los padres secundarios, con una
probabilidad de cruce uniforme designada como $CR$ (Paso 8.2). El hijo que se obtenga, tendrá
características de los tres padres.

El hijo obtenido se evalúa en base a las dos funciones objetivo que guían al algoritmo
(Paso 8.3), y si el individuo es igual al padre porque no se hayan producido cambios con
las operaciones realizadas anteriormente, se le fuerza a cambiar aleatoriamente un enlace,
añadiéndole un valor de una distribución Gaussiana $N(0,1)$ (Paso 8.4).

A continuación, se aplica al hijo el operador de mutación en cada una de sus
neuronas de la capa oculta (Paso 8.5). Al comenzar el algoritmo se establece un número
máximo de neuronas como en el algoritmo MPENSGAII (ver capítulo \ref{MOEANNs}). Para el
total del máximo de neuronas, si la neurona $i$ existe se elimina, y si no, se añade,
estableciendo enlaces y pesos de la misma forma que la mutación añadir neurona del
algoritmo MPENSGAII. Todo ello aplicando la mutación con una probabilidad determinada.

Cuando el nuevo hijo se ha creado, aumentamos en 1 el valor de una
variable llamada ``creados'', que nos servirá, en un momento dado de la evolución. Concretamente
cuando se hayan creado muchos hijos y ninguno de ellos por las circunstancias que se explican a
continuación se pueda añadir a la población actual (Paso 8.6).

Evaluamos al hijo en base a las dos funciones objetivo que guían a MPDE (Paso 8.7).

Si el hijo domina al padre principal se le aplica la LS con el algoritmo iRprop+ y se añade a
la población actual (Paso 8a y 8b). En caso contrario, si no hubiera relación de dominancia entre
padre e hijo, también se añade el hijo a la población actual (Paso 8c). En caso contrario, si el
padre principal domina al hijo, se comprueba si $creados=100$. En ese caso se elige el mejor de los
100 hijos almacenados, en base a la función objetivo $A_{1}$, y la variable creados se establece a
$0$ (Paso 8d y 8e). Sino se produce nada de lo anterior, el candidato es descartado (Paso
8f). El paso 8 se repite entero hasta completar el tamaño de la población que se establece en la
variable $M$.

Cuando se complete el tamaño de la población, ésta queda preparada para la siguiente
generación (Paso 9), y el proceso evolutivo continúa hasta que se cumpla la condición de
parada, que en nuestro caso es un número determinado de generaciones (Paso 10).


\subsection{Diferencias con el algoritmo MPANN}\label{diferencias}
\noindent Las diferencias fundamentales con respecto al algoritmo MPANN \cite{Abbass2002a} son las
siguientes:
\begin{itemize}
	\item El operador de cruce que nosotros utilizamos, también calcula aleatoriamente una
probabilidad uniforme en el intervalo $(0,1)$, y si el valor obtenido es menor que
el	valor de $CR$ no se aplica el operador. La diferencia está en que nuestro método no aplica el
valor de probabilidad obtenido aleatoriamente a todos los enlaces y neuronas de la capa oculta, sino
que utilizamos un nuevo valor aleatorio dentro del intervalo uniforme definido para
cada neurona y no para la capa oculta entera, como en el caso de MPANN. En este caso, nuestro
operador de cruce es menos agresivo con los cambios en las ANNs, ya que en alguna ocasión puede
que el valor aleatorio obtenido a partir de una probabilidad uniforme no sea menor que $CR$, con lo
que la neurona que se esté tratando es ese momento no cambia.
	\item La probabilidad de mutación $MR$, también se utiliza de manera independiente, al igual que
en el cruce, para cada neurona, y no para la capa oculta entera, como en el caso del algoritmo
MPANN.
	\item La manera en que  se añaden los individuos a la población en el algoritmo MPANN
	(solo los que dominan al padre principal), hace que el algoritmo pueda quedar estancado durante
	un buen número de generaciones (se ha comprobado experimentalmente) hasta que se pueda añadir
	un nuevo hijo. Nosotros añadimos individuos de una manera más ``relajada'', de
	manera que hijos que no dominen al padre principal tienen la opción de añadirse a la
	población. De esta manera también se reduce el coste computacional sin disminuir la
	calidad de los resultados.
\end{itemize}

\subsection{Diseño experimental}
\noindent Para analizar el rendimiento de MPDE hemos utilizado 6
conjuntos de datos del repositorio de la UCI \cite{UCI2007}.

En la tabla \ref{tabla1MPDE} se muestran las características de cada
conjunto: Número total de patrones por cada
conjunto de datos, número de patrones en entrenamiento y en generalización, número de
variables de entrada, número de clases, número total de patrones por clase y valor de $p^*$.

\begin{table}[htb!]
\scriptsize
\caption{Características de los conjuntos de datos de la UCI.}
\label{tabla1MPDE}
\centering
\tabcolsep 1pt
\begin{tabular}{c c c c c c p{2.5cm} c}
\hline
\rowcolor[rgb]{0.70,0.85,1}\textbf{Conjunto} & \textbf{Patrones} &
\textbf{Patrones} & \textbf{Patrones} &
\textbf{Variables} & \textbf{Clases} &
\textbf{Patrones} & $\mathbf{p^{*}}$ \\
\rowcolor[rgb]{0.70,0.85,1} & & \textbf{entrena.} & \textbf{generaliz.} & \textbf{de
entrada} & & \textbf{por clase} & \\ \hline
% \multicolumn{8}{>{\columncolor[rgb]{0.70,0.85,1}}c}{Dos clases} \\ \hline
\rowcolor[rgb]{0.86,0.94,1}Autos & 205 & 152 & 53 & 72 & 6 & 67-3-22-54-32-27 & 0.0188 \\
\rowcolor[rgb]{0.86,0.94,1}Balance & 625 & 469 & 156 & 4 & 3 & 288-49-288 & 0.0641 \\
\rowcolor[rgb]{0.86,0.94,1}BreastC & 286 & 215 & 71 & 15 & 2 & 201-85 & 0.2957 \\
\rowcolor[rgb]{0.86,0.94,1}HeartStatlog & 270 & 202 & 68 & 13 & 2 & 150-120 & 0.4411 \\
\rowcolor[rgb]{0.86,0.94,1}Newthyroid & 215 & 161 & 54 & 5 & 3 & 150-35-30 & 0.1296 \\
\rowcolor[rgb]{0.86,0.94,1}Pima & 768 & 576 & 192 & 8 & 2 & 500-268 & 0.3489 \\ \hline
\end{tabular}
\end{table}

El diseño experimental consiste en una partición estratificada del conjunto de datos con $3n/4$
patrones para
el conjunto de entrenamiento y $n/4$ patrones para el conjunto de generalización, siendo $n$ el
tamaño del conjunto.

El proceso de obtención de resultados es el mismo que el utilizado con MPENSGAII (ver
sección \ref{disenio} del capítulo \ref{MOEANNs}). Una vez se forma el frente de Pareto
se utilizan dos estrategias de selección
automática de individuos, el mejor modelo en $E$ y el mejor modelo en $MS$
(extremos del frente de Pareto). En cada ejecución del algoritmo (hacemos 30, dado que el proceso
de entrenamiento de la red es estocástico), una vez que tenemos el frente de Pareto de la última
generación del proceso evolutivo, se escogen los
extremos del frente en entrenamiento. Esto es, el mejor individuo en $E$, y el
mejor individuo en $MS$. A estos individuos se les llamamos individuo
$EI$, para el primer caso, e individuo $MSI$, para el segundo. Cuando tenemos los
individuos del paso anterior calculamos su valor de $C$ y de $MS$, sobre el conjunto
de generalización. De esta manera, tenemos para los extremos del frente dos pares de valores,
$\displaystyle EI=(C_{EI},MS_{EI})$ y $\displaystyle MSI=(C_{MSI},MS_{MSI})$ de una ejecución de las
30 realizadas.

Al repetirse el proceso anterior 30 veces obtenemos la media y la desviación
típica de los dos pares de valores para los individuos $EI$ y $MSI$, es decir,
$\displaystyle \overline{EI}=(\overline{C}_{EI},\overline{MS}_{EI})$ y
$\displaystyle \overline{MSI}=(\overline{C}_{MSI},\overline{MS}_{MSI})$, de forma que la
primera expresión muestra el rendimiento medio obtenido teniendo en cuenta solo los
mejores individuos en $E$, mientras que la segunda expresión muestra el rendimiento
medio teniendo en cuenta solo los mejores individuos en $MS$. A la manera
de obtener automáticamente el rendimiento medio teniendo en cuenta los mejores individuos
en $E$ (parte superior del frente) le hemos llamado MPEDEE, a y la forma de obtener el
rendimiento medio teniendo en cuenta los mejores individuos en $MS$ (parte inferior del
frente) la hemos llamado MPDES.

La probabilidad de cruce se estableció a $CR=0.8$ y la de mutación a $MR=0.1$, que es la
adoptada por Abbass en MPANN, y el tamaño de la población se estableció como $M=25$.

Para iRprop+, los parámetros adoptados son $\eta^{-}=0.5$ (tamaño de paso para el factor
de decremento), $\eta^{+}=1.2$ (tamaño de paso para el factor de incremento),
$\bigtriangleup_{0} =0.0125$ (valor inicial de tamaño de paso para los pesos,
$\bigtriangleup_{ij}$),  $\bigtriangleup_{min} =0$ (tamaño
mínimo de paso para los pesos), $\bigtriangleup_{max} =50$ (tamaño máximo de paso
para los pesos), $Epochs=5$ (número de épocas para la optimización local).

\subsection{Resultados}
\noindent Hemos comparado MPDE con nuestro algoritmo MPENSGAII y con la metodología SVM, a
partir del algoritmo SMO que proporciona Weka \footnote{http://www.cs.waikato.ac.nz/ml/weka/}
\cite{Witten2005}.

La tabla \ref{tabla2MPDE} presenta los valores de media y desviación típica para $C$ y
$MS$ obtenidos de los mejores modelos en $E$ en cada ejecución. Observar que en Balance y
Breast Cancer, el algoritmo MPDES obtiene los mejores valores en $MS$, encontrándose muy
cercano al algoritmo MPENSGAII en valores de $C$. En Autos, el mejor resultado en $C$ lo
obtiene MPDEE, pero el mejor resultado en $MS$ lo consigue el algoritmo MPENSGAIIS. En Newthyroid,
MPDE obtiene los mejores valores en $MS$ y $C$, y en Pima y Heart Statlog, MPDES obtiene
los mejores valores en $MS$, y muy similares en $C$ a los que obtiene MPENSGAIIE.

\begin{table}[!htb]
\tiny
\caption{Resultados estadísticos para MPDE, MPENSGAII y SVM en media y desviación típica
sobre el conjunto de generalización para $C$ y $MS$.}
\label{tabla2MPDE}
\centering
\tabcolsep 2pt
\renewcommand{\arraystretch}{1.2}
\begin{tabular}{llccllcc}
\hline
\rowcolor[rgb]{0.70,0.85,1}\textbf{Conjunto} & \textbf{Algoritmo} & \textbf{C(\%)} &
\textbf{MS(\%)} & \textbf{Conjunto} & \textbf{Algoritmo} & \textbf{C(\%)} &
\textbf{MS(\%)} \\ \hline
\rowcolor[rgb]{0.86,0.94,1}Autos & MPDEE & \textbf{68.79$\pm$5.59} & 28.75$\pm$21.40 &
Balance & MPDEE & 91.43$\pm$1.01 & 54.36$\pm$26.25 \\
\rowcolor[rgb]{0.86,0.94,1}& MPDES & 64.15\textit{$\pm$}5.63 &
12.26\textit{$\pm$}20.54\textbf{} &  & MPDES & 91.41$\pm$1.53 & \textbf{87.42$\pm$4.32}
\\
\rowcolor[rgb]{0.86,0.94,1}& MPENSGAIIE & \textit{66.67$\pm$4.07} &
\textit{39.64$\pm$14.92} &  & MPENSGAIIE & \textbf{94.01}$\pm$1.52\textbf{} &
42.66$\pm$17.00 \\
\rowcolor[rgb]{0.86,0.94,1}& MPENSGAIIS & 66.04\textit{$\pm$}4.78\textbf{\textit{}} &
\textbf{42.28\textit{$\pm$}10.98}\textit{\underbar{}} &  & MPENSGAIIS &
\textit{92.47$\pm$2.16} & \textit{83.72$\pm$8.19} \\
\rowcolor[rgb]{0.86,0.94,1}& SVM & 67.92 & 0.00 &  & SVM & 88.46 & 0.00 \\ \hline
\rowcolor[rgb]{0.86,0.94,1}BreastC & MPDEE & \textit{67.27$\pm$2.71} & 38.09$\pm$11.59 &
Newthy & MPDEE & \textit{96.66$\pm$2.02} & \textit{81.42$\pm$10.74} \\
\rowcolor[rgb]{0.86,0.94,1} & MPDES & 65.39$\pm$3.40\textbf{} & \textbf{57.04$\pm$7.01} &
& MPDES & \textbf{96.66$\pm$1.84} & \textbf{81.64$\pm$9.76} \\
\rowcolor[rgb]{0.86,0.94,1}& MPENSGAIIE & \textbf{69.34$\pm$2.30} & 28.88$\pm$9.09 &  &
MPENSGAIIE & 95.12$\pm$2.30\textbf{} & 74.81$\pm$10.07 \\
\rowcolor[rgb]{0.86,0.94,1} & MPENSGAIIS & 63.99$\pm$3.10 & \textit{53.08$\pm$6.57} &  &
MPENSGAIIS & 95.55$\pm$2.15 & 75.07$\pm$10.66\textit{} \\
\rowcolor[rgb]{0.86,0.94,1}& SVM & 64.79 & 23.81 &  & SVM & 88.89 & 55.56 \\ \hline
\rowcolor[rgb]{0.86,0.94,1}Pima & MPDEE & \textit{78.59$\pm$1.59} & 61.94$\pm$4.10 &
HeartStlg & MPDEE & 76.17$\pm$1.41 & 61.11$\pm$2.20 \\
\rowcolor[rgb]{0.86,0.94,1}& MPDES & 77.11$\pm$2.20\textbf{\textit{}} &
\textbf{73.12$\pm$2.98} &  & MPDES & \textit{76.27$\pm$1.57} & \textbf{63.66$\pm$2.37} \\
\rowcolor[rgb]{0.86,0.94,1}& MPENSGAIIE & \textbf{78.99$\pm$1.80} & 60.44$\pm$2.59 &  &
MPENSGAIIE & \textbf{78.28$\pm$1.75}\textit{} & 61.88$\pm$2.08\textit{} \\
\rowcolor[rgb]{0.86,0.94,1}\textbf{} & MPENSGAIIS & 76.96$\pm$2.08 &
\textit{72.68$\pm$3.06} &  & MPENSGAIIS & 77.5$\pm$1.73 & \textit{62.66$\pm$2.38}\textbf{}
\\
\rowcolor[rgb]{0.86,0.94,1} & SVM & 78.13 & 50.75 &  & SVM & 76.47 & 60.00 \\ \hline
\multicolumn{8}{l}{Los mejores resultados se muestran en \textbf{negrita} y los segundos
mejores resultados se muestran en \textit{cursiva}.}\\
\end{tabular}
\end{table}

Como ejemplo gráfico en la figura \ref{figuraBalance}, presentamos los resultados obtenidos por
el algoritmo MPDE en el conjunto de datos Balance, y al igual que en el caso de MPENSAII (ver
sección \ref{resultadosMPENSGAII} del capítulo \ref{MOEANNs}), están divididos en gráficos de
entrenamiento $(A_{1},A_{2})$, y en gráficos de generalización $(A_{1},C)$.

En nuestra opinión, el uso de la DE junto con métodos de LS usando $E$ y $MS$ como
objetivos
a optimizar, puede ser un nuevo punto de vista para tratar problemas multi-clase en clasificación,
con resultados muy prometedores.

En el siguiente capítulo se expone una aplicación realiza con el algoritmo MPENSGAII que
estudiamos y detallamos en el capítulo \ref{MOEANNs}, concretamente una aplicación sobre
microbiología predictiva \cite{Valero2009}.

\begin{figure}[!htb]
\centering
\includegraphics[keepaspectratio, width=8cm]{figuras/ejemploBalanceMPDE.jpg}
\caption{Frente de Pareto en entrenamiento $(A_{1},A_{2})$, y valores asociados a\\
$(A_{1},C)$ en generalización para el conjunto de datos Balance.}
\label{figuraBalance}
\end{figure}
\paginavaciacompleta
% \section{Introducción a los intervalos de confianza}
%
% \subsection{Intervalos de confianza aplicados a la evolución diferencial con redes}
% \noindent Aquí se hablará de los intervalos de confianza, las distintas normas y se
% pondrá HPDE con y sin intervalos de confianza.
% Se podría hacer también una breve comparación con resultados obtenidos sin intervalos de
% confianza para ver si se mejora.
%
% Velazquez tesis, pag 96
% \chapter{El algoritmo HPDCI}
%
% \section{Funciones objetivo}
%
% \section{Operadores}
%
% \section{iRprop+ como búsqueda local}
%
% \section{Etapas y aspectos relevantes de HPDE}
%
% \section{Resultados}
		%\paginavaciacompleta
		%\input{./capitulos/aplicaciones}
		%\paginavaciacompleta
		%\paginavaciasincuerpo
\chapter{Comentarios finales}
\noindent En este capítulo exponemos las conclusiones obtenidas en este trabajo de tesis,
las
publicaciones asociadas a la misma, y las líneas futuras en la que vamos a trabajar a corto o medio
plazo.

\section{Conclusiones}
\begin{itemize}
	\item Hemos realizado un estudio de las técnicas de multi-clasificación
		de patrones utilizando modelos de ANNs entrenados mediante EAs,  tanto a nivel mono-objetivo
	como multi-objetivo,	incidiendo en esta última propuesta. Este estudio se
	ha incluido en esta tesis con una revisión de trabajos previos realizados en este
	área, justo
	antes de explicar cada uno de los algoritmos que hemos desarrollado.

	\item Hemos estudiado las métricas que usualmente se utilizan en la clasificación de
	patrones con
	ANNs y hemos propuesto una nueva forma de evaluar la precisión de un multi-clasificador
	mediante las
	métricas de precisión, $C$ y de mínima sensibilidad, $MS$ en un marco
	multi-objetivo. La precisión no puede capturar todos los aspectos de comportamiento	de
	dos
	clasificadores diferentes \cite{Provost1997,Provost1998}, y no es
	suficiente, en algunos casos, para	determinar la calidad de un clasificador. Si se
	asume que todos los errores de clasificación son igualmente costosos y que no hay
	preferencia ni penalización para un determinado conjunto	de patrones, un buen
   clasificador	debería obtener un alto nivel de precisión global, así como un aceptable
	nivel de	precisión para cada clase, aspecto muy importante en medicina y en
	aplicaciones
	como la microbiología predictiva.

	\item Hemos desarrollado un algoritmo evolutivo mono-objetivo llamado CBFEP
(\textit{Combined
	Basis Function Evolutionary Programing}) para multi-clasificación de	patrones, que optimiza
	simultáneamente la topología y los pesos de cada ANN, usando un EA e	introduciendo en la capa
	oculta diferentes tipos de funciones de base (SUs, PUs y RBFs).

	Determinar cuál es la mejor configuración de funciones de base en la capa oculta depende, en gran
	medida, de la estructura del conjunto de	datos, pero en general, los modelos híbridos en
	cuanto a funciones de base SRBF (SUs+RBFs)	alcanzan una precisión mayor que los modelos 	PRBF
	(PUs+RBFs), especialmente en conjuntos de datos
	con dos clases, donde solamente existe	una función discriminante. La capacidad de generalización
	utilizando modelos con funciones	de base de tipo local y de tipo global (por ejemplo, modelos
	SRBF o PRBF) es similar o mayor que la capacidad de generalización de modelos puros formados
	solamente por PUs, SUs o RBFs. Por otra parte, la hibridación	de dos funciones de base de tipo
	proyección (PUs y SUs) no presenta	mejores resultados en precisión que las restantes
	hibridaciones, aunque si tienen un	número más bajo de conexiones, permitiendo modelos más
	interpretables sin disminuir excesivamente la precisión del multi-clasificador.
% 	La capacidad de generalización de los modelos PSU (PUs+SUs) es similar a la que se ha obtenido
% 	con sus correspondientes modelos puros (solo PUs o solo SUs). En general, el número de conexiones
% 	de los modelos puros	formados con PUs son siempre menores que en los demás modelos y el número de
% 	conexiones en modelos híbridos que usan PUs, como PRBF (PUs+RBF) y PSU (PUs +SUs) son
% 	menores que las que se obtienen con un modelo híbrido SRBF (SUs+RBF).

	\item Hemos desarrollado un MOEA basado en el concepto de dominancia de Pareto a partir del
	algoritmo NSGAII, al que hemos llamado MPENSGAII (\textit{Memetic Pareto NSGAII}). MPENSGAII
	se encarga de utilizar las funciones de aptitud de entropía, $E$ y de mínima
	sensibilidad, $MS$, para guiar al
   algoritmo en entrenamiento y obtener  modelos de red para multi-clasificación de
	patrones en generalización dentro del espacio $(MS,C)$, de manera que combinen
	un alto nivel  de clasificación con un buen nivel de clasificación por clase. El
	algoritmo MPENSGAII también optimiza de manera conjunta la topología y los pesos de red. Los
	resultados obtenidos son muy prometedores.

	\item Los MOEAs desarrollados incorporan el algoritmo iRprop+ como LS, el
	cual se ha adaptado a las nuevas métricas utilizadas para guiar dichos MOEAS ($E$ y
	$MS$).

	\item Se ha desarrollado un conjunto de clasificadores o \textit{ensemble}  que intenta obtener
	un buen equilibrio entre diversidad y precisión. Los resultados obtenidos como primera
	aproximación a este tipo de técnicas son buenos.

	\item Hemos desarrollado un MOEA basado en el concepto de dominancia de Pareto utilizando la DE.
	Al algoritmo desarrollado le hemos llamado MPDE (\textit{Memetic Pareto Differential
	Evolution}), y mejora sustancialmente a otros algoritmos de la literatura como el algoritmo
	MPANN. MPDE al igual que los demás MOEAs desarrollados en esta tesis está
	hibridado con un procedimiento de LS, utilizando el algoritmo iRprop+. También
	evoluciona de	manera conjunta la topología y pesos de la red, y el coste computacional
asociado a
	la	obtención del frente de Pareto final es mucho menor que en otros algoritmos de la
	literatura, ya que hemos utilizado una manera más "relajada" de incorporar y sustituir
	individuos de la población de una generación a otra. Los resultados obtenidos también son muy
	prometedores.

	\item La manera de representar gráficamente los resultados obtenidos con los MOEAs desarrollados,
	se muestran en gráficas de entrenamiento y generalización usando la región
	factible (sección \ref{propiedades} del capítulo \ref{medidasRendimiento}) y se presenta
	como una alternativa a las curvas ROC pudiéndose utilizar en problemas multi-clase. Trabajar con
	más
	de dos objetivos con curvas ROC tiene la desventaja de que el número	de dimensiones de las
	representaciones gráficas aumenta, y por tanto es más difícil su análisis e
	interpretación. Sin embargo, nuestras gráficas bidimensionales permiten una lectura más
	sencilla y rápida del rendimiento de un multi-clasificador.

	\item Hemos aplicado el algoritmo MPENSGAII desarrollado a la
	predicción de crecimiento	microbiano, debido a la demanda de los expertos en alimentación de
	contar con productos	alimenticios más	saludables. Los modelos obtenidos
	predicen, a	partir de factores medioambientales de envasado y/o conservación, si un patógeno va a
	crecer o no en función de los valores
	de esas variables. Los conjuntos de datos utilizados para testar nuestro algoritmo son los
	microorganismos que producen más trastornos gastrointestinales en nuestro organismo,
	concretamente los	patógenos \textit{Listeria Monocytogenes}, \textit{Escherichia Coli R31},
	\textit{Staphylococcus	Aureus} y \textit{Shiguella Flexneri}. Los datos de estos patógenos se
	han dividido mediante un diseño factorial fraccional, en un conjunto de datos de entrenamiento y
	otro de generalización para entrenar	y validar nuestros	modelos de red.

	Los resultados estadísticos obtenidos y los test de comparaciones múltiples analizados muestran
	que nuestro método es muy competitivo para ser considerado en microbiología predictiva, donde hay
	una alta necesidad de obtener una buena precisión en la clasificación, tanto para el
	crecimiento como para el no crecimiento. Este enfoque puede ayudar a los investigadores de
	microbiología predictiva a definir mejor los límites de crecimiento de los microorganismos, y
	también para modelar la	variabilidad microbiana asociada a las condiciones de envasado y
	conservación. En conclusión, la
	utilización de este método	constituye una valiosa alternativa para la creación de modelos
	matemáticos que determinen la probabilidad de crecimiento microbiano bajo un determinado conjunto
	de condiciones.

   \item Hemos incrementado el número de algoritmos de la herramienta software Keel (ver apéndice
	\ref{anexo}) con	la inclusión y adaptación del algoritmo iRprop+ como método de entrenamiento de
	ANNs (tanto con funciones de base SUs como con PUs), y hemos	desarrollado una parte
dedicada a la enseñanza, de forma que los alumnos puedan
	obtener visualmente	los resultados que un determinado algoritmo produce a lo largo de
la evolución
	o del tiempo,	mostrando los resultados finales de manera local, en la misma máquina donde se
	instala la herramienta. Normalmente con Keel se trabaja diseñando baterías de	pruebas y de
	algoritmos para lanzarlas en maquinas diferentes a donde está instalado el software, por
	ejemplo en un cluster de	ordenadores que nos permita obtener una solución de la manera más
	rápida posible.
\end{itemize}

\section{Publicaciones asociadas a la tesis}
\noindent A continuación se exponen las publicaciones asociadas a esta tesis doctoral.
\subsection{Artículos en revistas}
\begin{itemize}
	\item J. C. Fernández, F. J. Martínez, C. Hervás, and P. A. Gutiérrez. Sensitivity versus
accuracy in multi-class problems using memetic pareto evolutionary neural
			networks. \textit{IEEE Transactions on Neural Networks}, 21(5):750–770, 2010.
	\item J. C. Fernández, C. Hervás, F. J. Martínez-Estudillo, and P. A. Gutierrez. Memetic
			pareto evolutionary artificial neural networks to determine growth/no-growth in predictive
			microbiology. \textit{Applied Soft Computing}, doi:10.1016/j.asoc.2009.12.013, 2009.
	\item P. A. Gutiérrez, C. Hervás, J. C. Fernández, M. Jurado-Exposito, J. M. Peña-Barragán, and
			F. López-Granados. Structural simplification of hybrid neuro-logistic regression models in
			multispectral analysis of remote sensed data. \textit{ Neural Network World}, 19(1):3–20,
			2009.
	\item P. A. Gutiérrez, C. Hervás, M. Carbonero, and J. C. Fernández. Combined projection and
			kernel basis functions for classification in evolutionary neural
			networks. \textit{Neurocomputing}, 72:2731–2742, 2009.
	\item J. Alcala-Fdez, L. Sánchez, S. García, M. J. d. Jesus, S. Ventura, J. M. Garrell, J. Otero,
			C. Romero, J. Bacardit, V. M. Rivas, J. C. Fernández, and F. Herrera. Keel: A software tool
			to assess evolutionary algorithms for data mining problems. \textit{ Soft Computing},
			13:301–318,	2009.
\end{itemize}

\subsection{Artículos en congresos internacionales}
\begin{itemize}
	\item M. Cruz-Ramírez, C. Hervás-Martínez, J.C. Fernández, and J. Sánchez-Monedero: Learning
			Artificial Neural Networks Multiclassifiers by Evolutionay Multiobjective Differential
			Evolution guided by Statistical Distributions. \textit{In Proceedings of the Tenth
			International Join Conference on Neural Networks, IJCNN2010}, Accepted, Barcelona,
			Spain, July 2010.
	\item M. Cruz-Ramírez, J. Sánchez-Monedero, F. Fernández-Navarro, J.C. Fernández, and C.
			Hervás-Martínez: Hybrid Pareto Differential Evolutionary Artificial Neural Networks to
			determine growth multi-classes in Predictive Microbiology. \textit{In Proccedings of the
			The Twenty	Third International Conference on Industrial, Engineering and Other
			Applications of Applied	Intelligent Systems IEA-AIE 2010}, Accepted, Córdoba, Spain,
			June 2010.
	\item C. Hervás-Martínez, P. A. Gutiérrez, J. C. Fernández, S. Salcedo-Sanz, A.
			Portilla-Figueras, A. Pérez-Bellido, and L. Prieto. Hyperbolic tangent basis
			function neural networks training by hybrid evolutionary programming for accurate
			short-term wind speed prediction. \textit{In Proceedings of the Ninth International
			Conference on Intelligent Systems Design and	Applications, ISDA 2009}, pages 193–198,
			Pisa, Italy, November 2009.
	\item J. C. Fernández, C. Hervás, F. J. Martínez, and M. Cruz. Design of artificial neural
			networks
			using a memetic pareto evolutionary algorithm using as objectives entropy versus variation
			coefficient. \textit{In Proceedings of the Ninth International Conference on Intelligent
			Systems Design and Applications, ISDA 2009}, pages 408–413, Pisa, Italy, November
			2009.
	\item J. C. Fernández, C. Hervás, F. J. Martínez, P. A. Gutiérrez, and M. Cruz. Memetic
			pareto differential evolution for designing artificial neural networks in
			multiclassification problems using cross-entropy versus sensitivity. \textit{In Proceedings
			of the 4th International Conference, HAIS 2009}, in Lecture Notes in Artificial
			Intelligence, volume 5572, pages 433–441, Salamanca, Spain, June 2009.
	\item P. A. Gutiérrez, C. Hervás-Martínez, J. C. Fernández, and F. López-Granados. Hybrid
			multilogistic regression by means of evolutionary radial basis functions: Application to
			precision agriculture. \textit{In Proceedings of the 4th International Conference, HAIS
			2009}, in Lecture Notes in Artificial Intelligence, volume 5572, pages 244–251,
			Salamanca, Spain,	June 2009.
	\item P. A. Gutiérrez, C. Hervás-Martínez, F. J. Martínez-Estudillo, and J. C. Fernández. Mul-
			tilogistic regression using initial and radial basis function covariates. \textit{In
			Proceedings of	the 2009 International Join Conference on Neural Networks, IJCNN 09},
			pages 1067–1074, Atlanta, USA, June 2009.
	\item F. Fernández-Navarro, P. A. Gutiérrez, C. Hervás-Martínez, and J. C. Fernández. A sen-
			sitivity clustering method for hybrid evolutionary algorithms. \textit{In Proceedings of
			the Third International Work-Conference on the Interplay between Natural and
			Artificial Computation, IWINAC2009} in Lecture Notes in Computer Science, Part 1,
			volume 5601, pages	245–254, Santiago de Compostela, Spain, June 2009.
	\item P. A. Gutiérrez, J. C. Fernández, and C. Hervás. Feature selection for hybrid neuro-
			logistic regression applied to classification of remote sensed data. \textit{In Proceedings
			of the Eighth International Conference on Hybrid Intelligent Systems, HIS 2008},
			pages 625–630, Barcelona, Spain, September 2008.
	\item J. C. Fernández, P. A. Gutiérrez, C. Hervás, and F. J. Martínez. Memetic pareto
			evolutionary artificial neural networks for the determination of growth limits of listeria
			monocytogenes. \textit{In Proceedings of the Eighth International Conference on
Hybrid Intelligent			Systems, HIS 2008}, pagess 631–636, Barcelona, Spain, September 2008.
	\item F. J. Martínez-Estudillo, P. A. Gutiérrez, C. Hervás, and J. C. Fernández. Evolutionary
			learning by a sensitivity-accuracy approach for multi-class problems. \textit{In
			Proceedings of	the IEEE World Congress on Computational Intelligence, CEC 08}, pages
			1581–1588, Hong Kong, China, June 2008.
	\item A. Valero, F. Pérez-Rodriguez, E. Carrasco, C. Hervás, P. A. Gutiérrez, J. C. Fernández, R.
			M. García-Gimeno, and G. Zurera. Evolutionary combined neural networks for
			modelling the growth boundaries for a five strain staphylococcus aureus cocktail against
			temperature, ph and water activity. \textit{In Proceedings of the 5th International
			Conference	Predictive Modelling in Foods, IC PMF 2007}, pages 291–294, Athens,
			Greece, September	2007.
	\item P. A. Gutierrez, C. Hervás, M. Carbonero, and J. C. Fernández. Combined proyection and
			kernel basis functions for classification in evolutionary neural networks. \textit{In
			Proceedings	of the XII Conference of the Spanish Association for Artificial
			Intelligence (CAEPIA-TTIA 2007), II International Workshop on Hybrid Artificial
			Intelligence Systems	(HAIS 2007)} in Innovations in Hybrid Intelligent Systems,
			Advances in Soft Computing, volume 44, pages 88–95, Salamanca, Spain, September 2007.
	\item C. Hervás, F. J. Martínez, M. Carbonero, C. Romero, and J. C. Fernández. Evolutionary
			combining of basis function neural networks for classification. \textit{In Proceedings of
			the 2nd.	International Work-Conference on the Interplay between Natural and
			Artificial Computaion, IWINAC2007} in Lecture Notes in Computer Science, Part 1,
			volume 4527, pages	447–456, La Manga del Mar Menor, Spain, June 2007.
\end{itemize}

\subsection{Artículos en congresos nacionales}
\begin{itemize}
	\item P. A. Gutiérrez, C. Hervás, J. C. Fernández, J. M. Peña-Barragan, M. Jurado-Esposito, y
			F. López-Granados. Hibridación de algoritmos de aprendizaje para modelos neurologísticos
			aplicados a la clasificación de cubiertas vegetales. \textit{ En Actas del VI Congreso
			español sobre	metaheurísticas, algoritmos evolutivos y bioinspirados, MAEB 2009},
			páginas 325–332, Málaga, España, Febrero 2009.
	\item J. C. Fernández, M. Carbonero, P. A. Gutiérrez, y C. Hervás. Ensembles de redes
			neuronales construidos mediante algoritmos híbridos multiobjetivo para optimizar la
			precisión y la sensitividad. \textit{En Actas del VI Congreso español sobre
metaheurísticas,
			algoritmos evolutivos y bioinspirados, MAEB 2009}, páginas 309–316, Tenerife, España,
			Febrero 2009.
	\item C. Hervás, F. J. Martínez, A. C. Martínez, P. A. Gutiérrez, y J. C. Fernández. Aprendizaje
			mediante la hibridación de técnicas heurísticas y estadísticas de optimización en
			regresión logística binaria. \textit{En Actas del V Congreso Español sobre metaheurísticas,
			algoritmos evolutivos y bioinspirados, MAEB 2007}, páginas 61–68, Tenerife, España,
			Febrero	2007.
	\item C. Hervás, F. J. Martínez, P. A. Gutiérrez, J. C. Fernández, y A.J. Tallón. Clasificación
			mediante la evolución de modelos híbridos de redes neuronales. \textit{En Actas del V
			Congreso	Español sobre Metaheurísticas, Algoritmos Evolutivos y Bioinspirados,
			MAEB07}, páginas 77–84, Tenerife, España, Febrero 2007.
	\item P. A. Gutierrez, J. C. Fernández, y C. Hervás. Algoritmos de aprendizaje evolutivo y
			estadístico para la determinación de mapas de malas hierbas utilizando técnicas de telede-
			tección. \textit{En Actas del II congreso español de informática, CEDI 2007, dentro del
			taller I	jornadas sobre algoritmos evolutivos y metaheurísticas, JAEM 07}, páginas
			65–72, Zaragoza,	España, Septiembre 2007.
	\item C. Hervás, P. A. Gutiérrez, J. C. Fernández, y A.J. Tallón. Regresión logística multi-
			clase utilizando funciones de base evolutivas de tipo proyección. \textit{En Actas del II
			congreso	español de informática, CEDI 2007, dentro del IV taller de minería de datos
			y aprendizaje,	TAMIDA 2007}, páginas 65–72, Zaragoza, España, Septiembre 2007.
\end{itemize}
\newpage
\section{Trabajo futuro}
\begin{itemize}
% 	\item Con respecto a la hibridación de modelos con diferentes unidades en capa oculta:
% 				\begin{enumerate}
% 				   \item Actualmente estamos trabajando en la hipótesis sobre
% 					el caso en que una estructura	de los modelos que utilizan de forma combinada
% 					funciones de base PU y RBF podrían ajustarse a los datos mejor que
% 					otros modelos que utilicen	unidades	SU y RBF puras, asociado a la existencia de una
% 					gran diferencia entre los	valores de un relativamente pequeño número de
% 					datos en los bordes del dominio y en el dominio interno.
% 					\item A partir de las distribuciones de frecuencias asociadas a los patrones de los
% 				    conjuntos de entrenamiento en problemas de clasificación y de las funciones de
% 				    entropía cruzada, ¿es posible definir las características asociadas a la base
% 				    de datos que determinen la familia de funciones (reconocedores universales)
% 				    que es adecuada para mejorar la clasificación?
% 				    \item Podemos determinar algunas propiedades de las funciones de probabilidad a
% 		          priori que permitan elegir entre diferentes familias de reconocedores
% 				    universales, o bien entre la mezcla ''inteligente`` de diferentes modelos.
% 					 \item ¿Es posible incorporar al algoritmo evolutivo procedimientos que permitan, en
% 		          determinados momentos de la evolución, determinar el tipo de unidad que hay
% 					 que incluir en el modelo para mejorar la precisión del clasificador en el
% 				    conjunto de generalización?
% 				\end{enumerate}
	\item En cuanto al algoritmo MPENSGAII hay varias direcciones de trabajo que se podrían llevar a
			cabo en el futuro:
			\begin{enumerate}
					\item Desarrollar otros MOEAS basados en las medidas $(MS,C)$.
					\item Utilizar otro tipo de funciones de base en el modelo de red neuronal, ya que
					ambas medidas son	independientes del algoritmo y de la función base
					utilizada.
					\item Incorporar nuevas técnicas de \textit{ensembles} que mejoren los resultados
					obtenidos con nuestra aproximación \cite{Dietterich1997,Kuncheva2005}, y que
					trabajen con los modelos del frente de Pareto de manera más eficaz. Por
					ejemplo,	particionando el conjunto de datos, usando coevolución o
					utilizando la técnica TOPSIS (\textit{Technique for Order Pre-ference by
					Similarity to Ideal Solution}) \cite{Hwang1981}. La lógica subyacente
					de la técnica TOPSIS es definir la solución ideal positiva y la solución
					ideal negativa. La	solución ideal positiva es la solución que maximiza
					los criterios de beneficio y reduce al mínimo	los criterios de costes; la
					solución ideal negativa	es aquella que maximiza los criterios de
					costes y reduce al mínimo los criterios de beneficio. La alternativa
					óptima es la que esté más cerca de la solución ideal positiva y más
					alejada	de la solución ideal negativa. Las distintas	alternativas se
					ordenan según su valor de proximidad con la solución ideal,	siendo la
					mejor alternativa aquella con mayor	valor de proximidad.
					\item Incorporar técnicas de remuestreo para	problemas altamente desbalanceados, como
					por ejemplo SMOTE \cite{Chawla2002}.
			\end{enumerate}
	\item Utilizar otras métricas distintas a la precisión y la $MS$ que sean
			favorables para obtener clasificadores con una precisión alta y con un buen balance de
			clasificación en todas las clases de un determinado problema multi-clase.
	\item Estamos estudiando y trabajando en el algoritmo diferencial desarrollado en esta tesis
			para mejorar la			manera de			seleccionar los padres	principales, a partir de
			los cuales se obtiene un			nuevo hijo. Concretamente
			estamos usando	como padres los valores extremos y el estadístico de centralización del
			intervalo de confianza construido utilizando distribuciones normales y no
normales
			asociadas a los mejores individuos de la población, creando así padres virtuales.
	 \item En un trabajo futuro, quedaría abierta la opción de aplicar un algoritmo de LS en las
			dos direcciones u objetivos que guían a MPENSGAII y a MPDE (ver artículos
			de Moller \cite{Moller1993,Falas2005}), por ejemplo mediante el uso de otra función
			objetivo que sustituya a $MS$ y que sea derivable, mediante un procedimiento por etapas en
			las que de forma alternada o secuencial se optimicen los objetivos del problema, o
			mediante algún tipo de combinación que utilice información de los dos objetivos.
	 \item Otra intención a corto plazo es probar el rendimiento del algoritmo MPENSGAII y MPDE con
			otras funciones de base como PUs, RBFs, GRBFs y q-Gaussianas en capa oculta, en
vez de utilizar solo funciones de base SUs,
			como se ha hecho hasta	ahora.
	 \item Finalmente, siguiendo la metodología del algoritmo desarrollado para entrenar
modelos
			híbridos, CBFEP, pretendemos desarrollar este tipo de modelos en un entorno
multi-objetivo.
% 	 \item También sería interesante estudiar otras técnicas de la literatura sobre DE para poder
% 			aplicar algún concepto diferente a nuestro algoritmo MPDE que hagan mejorar su rendimiento
% 			y		velocidad de convergencia, esta última se ha mejorado considerablemente con
% 			respecto a los		algoritmo más importantes de la literatura en este ámbito, como por
% 			ejemplo el algoritmo		MPANN.
\end{itemize}
\paginavaciacompleta
		%\paginavaciacompleta
		%\input{./capitulos/anexo}
		%\paginavaciacompleta
		%  -------------------------------------------------------------------------------
		% APENDICES
		%\renewcommand{\appendixname}{Apéndice}
		%\renewcommand{\thechapter}{A}
		% DESCOMENTAR LAS 4 SIGUIENTES
		%\appendix
		%\appendixpage
		%\addappheadtotoc
		%\input{./capitulos/apendice}

%%%%%%%%%%%%%%%%%%%%%%%%%%%%%%%%%%%%%%%%%%%%%%%%%%%%%%%%%%%%%%%%%%%%%%%%%%%%%%%%%%%%%%%%%%
%%%%  Unidad posterior del book
%%%%%%%%%%%%%%%%%%%%%%%%%%%%%%%%%%%%%%%%%%%%%%%%%%%%%%%%%%%%%%%%%%%%%%%%%%%%%%%%%%%%%%%%%%
	\backmatter
		%  -------------------------------------------------------------------------------
		% Apéndices del documento
		%\appendix
		% Paginado del apéndice
		%\appendixpage \addappheadtotoc
		% Apéndice para la evolucion diferencial
		%\chapter{MOEA memético utilizando evolución diferencial}\label{diferencial}
\section{Introducción a la evolución diferencial}
\noindent A continuación se dará una breve introducción de la evolución diferencial
(\textit{Differential Evolution}, DE) \cite{Storn1997} para entender el algoritmo MPANN
(\textit{Memetic Pareto Artificial Neural Networks}, MPANN) de Abbass, el cual
describiremos en las siguientes secciones, y a partir del cual hemos desarrollado un
algoritmo propio basado en DE llamado MPDE (\textit{Memetic Pareto Differential Evolution}),
utilizando las medidas $(MS,C)$ descritas en el capítulo \ref{medidasRendimiento}.

La DE fue propuesta por Storn y Price \cite{Storn1997} como nueva
heurística para la minimización de funciones no lineales y no diferenciables en espacios
totalmente ordenados. La  DE  es  un  tipo  de técnica de optimización  global que  usa
selección y cruce como sus operadores primarios para la optimización
de problemas sobre dominios continuos, e incluso mutación, aunque este último operador
fue introducido posteriormente al trabajo inicial de Storn y Price, por ejemplo en
\cite{Abbass2002a}.

Los algoritmos tradicionales deterministas son insuficientes cuando se abordan problemas de
optimización que presentan características como: no linealidad, alta dimensionalidad, existencia de
múltiples óptimos locales (multimodal), no diferenciabilidad o ruido. La DE es un método alternativo
en la solución a problemas con estas características.

Dentro de las características fundamentales de la DE se encuentran \cite{Storn1997}:
\begin{itemize}
\item La capacidad de manejar funciones objetivo no lineales, no diferenciables y
multimodales.
\item La paralelización,  puesto  que el algoritmo es fácilmente  paralelizable y resulta
útil  cuando la evaluación de la función objetivo es computacionalmente costosa.
\item La  falta  de  predefinición  de  las  distribuciones  de  probabilidad,  como  en
el  caso  de  las estrategias evolutivas.
\item El uso de una codificación real y de una precisión determinada por el formato de
punto flotante empleado.
\item La  convergencia  a  un  valor  óptimo  (posiblemente  local)  de  manera
consistente  a  lo  largo  de una secuencia de ejecuciones independientes.
\item La  escasa  utilización  de  parámetros  de  control,  puesto  que    el  algoritmo
básico \cite{Storn1997} emplea  únicamente tres parámetros de control, además de un
criterio de terminación.
\item Así  mismo,  posee  una  amplia  gama  de  aplicaciones,  de  entre  las
que se  encuentran  el entrenamiento  de  ANNs,  el  diseño  de  filtros
digitales,  la optimización  de  procesos químicos no lineales, el diseño de redes de
transmisión de aguas, etc \cite{Chakraborty2008}.
\end{itemize}

La DE, como heurística evolutiva, tiene algunas características básicas que comparte con
los EAs \cite{Mezuza2008}:
\begin{itemize}
\item Es una aproximación basada en poblaciones.
\item El cruce y en ocasiones la mutación, se utilizan como operadores para generar nuevas
soluciones.
\item Hay un mecanismo de reemplazo para mantener un tamaño fijo en la población de
individuos.
\end{itemize}

La DE difiere \cite{Storn1997} trabaja creando una población inicial aleatoria de
soluciones, donde está garantizado, mediante reglas de reparación, que el valor de cada
variable esta dentro de unos límites. Entonces un individuo  es seleccionado
aleatoriamente  para  ser reemplazado  y  se seleccionan tres  padres
diferentes para formar un nuevo hijo. Uno de estos padres se elige como padre
principal
y cada alelo en el padre principal se cambia al azar, donde al menos una variable debe ser
cambiada. Esto
se lleva a cabo añadiendo al valor de las variables un valor ponderado de la diferencia
entre los dos valores
de esta variable en los otros dos padres. En esencia, el vector asociado al padre principal es
perturbado con el vector de  los  otros  dos  padres.  Esto representa  el  operador  de  cruce  en
la  DE. Si el  vector  resultante  es mejor  que  el  escogido  para  reemplazar,  se  reemplaza,
en  caso  contrario  el vector  escogido permanece en la población.

Detallado de manera más formal, y en la misma notación que propuso Storn
y Price \cite{Storn1997}, una solución $l$, en la generación $i$, es un vector
multidimensional $\displaystyle \mathbf{x}_{G=i}^l =
\left(x_{1}^l,\cdots,x_{N}^l\right)^T$. Una población, $P_{G=k}$, en la generación $G=k$ es un
vector de $M$ soluciones $(M>4)$. La población inicial, $P_{G=0}=\left\lbrace
\mathbf{x}_{G=0}^1,\cdots,\mathbf{x}_{G=0}^M\right\rbrace $se inicializa como:
\begin{gather}\nonumber
	x_{i,G=0}^l = inferior(x_{i})+aleatorio_{i}[0,1]\cdot(superior(x_{i})-inferior(x_{i})), \nonumber
\\
		l=1,\cdots,M, \quad i= 1,2,\cdots,N, \nonumber
\end{gather}
donde $M$ es el tamaño de la población, $N$ es la dimensión de la solución, y cada variable
$i$ en un vector de soluciones $l$ en la generación inicial $\displaystyle G=0,x_{i,G=0}^l$, se
inicializa dentro de los límites $\displaystyle (inferior(x_{i}),superior(x_{i}))$. La selección
se lleva a cabo seleccionando 4 soluciones con índices diferentes (3 padres más la solución a
reemplazar), $r^1, r^2, r^3$ y $j\in [1,M]$. Los valores de cada variable en el hijo se cambian
mediante un operador de cruce con una probabilidad $CR$, de la siguiente manera:
\begin{equation}
				\forall i \leq N,x_{i,G=k}^{'}= \left\lbrace
				\begin{array}{ll}\nonumber
			   x_{i,G=k-1}^{r^3} + F\cdot (x_{i,G=k-1}^{r^1} - x_{i,G=k-1}^{r^2}) & \\
			   \mbox{si $(aleatorio[0,1)< CR \wedge i=i_{aleatorio})$} \\
            x_{i,G=k-1}^{j} \quad \mbox{en otro caso}
            \end{array}
            \right.
\end{equation}
donde $F\in(0,1)$ es un parámetro que representa la cantidad de perturbación añadida al padre
principal ($r^3$ en este caso). La nueva solución reemplaza a la antigua seleccionada si es mejor
que ella, y al menos una de las variables se debe cambiar. Esta última está representada en el
algoritmo de forma aleatoria, seleccionando una variable, $\displaystyle i_{aleatorio}\in(1,N)$.
Después del cruce, si una o más de las variables en la nueva solución se encuentran fuera de su
límite inferior o superior, se aplica la siguiente regla de reparación:
\begin{equation}
				x_{i,G=k}= \left\lbrace
				\begin{array}{ll}\nonumber
			   \frac{x_{i,G}+inferior(x_{i})}{2}  \quad \text{si $x_{i,G+1}^j < inferior(x_{i})$} \\
            inferior(x_{i}) + \frac{x_{i,G}+superior(x_{i})}{2} \quad \text{si $x_{i,G+1}^j>
superior(x_{i})$} \\
            x_{i,G+1}^j \quad \text{en otro caso}
            \end{array}
            \right.
\end{equation}

Una vez comentados los detalles esenciales de la DE, decir que la DE está restringida a
dominios donde el espacio de búsqueda está completamente
ordenado y en especial subespacios de $\Re^n$. Un individuo se representa como una
$n$-tupla, llamada vector objetivo $\mathbf{x}=(x_{1},\cdots,x_{n})$, donde $x_{i} \in \Re
(i=1,\cdots,n)$ son los valores escalares que representan las variables de diseño del
problema. Emplear una codificación
real permite generar perturbaciones acotadas en las variables de diseño, a consecuencia
del orden determinado por $\Re^n$. La codificación real de la DE, a diferencia las
estrategias de evolución (\textit{Evolution Strategies}, ES) no utiliza una distribución
fijada (como la
distribución Gaussiana fijada en ES) para controlar el comportamiento del operador de
mutación, en lugar de ello, la distribución de las soluciones en el espacio de búsqueda
determina el tamaño de paso y la dirección de búsqueda para cada individuo.

El lector puede consultar una revisión completa y detallada sobre DE y sobre aplicaciones
en \cite{Price2005,Chakraborty2008,Mezuza2008,Iorio2006} y en un reciente trabajo de
Ferrante Neri y Ville Tirronen en \cite{Neri2010}.

\subsection{Variantes de la evolución diferencial}\label{variantes}
\noindent Hay algunas variantes \cite{Mezuza2008} del algoritmo básico de DE, que se
diferencian en:
\begin{itemize}
	\item El tipo de criterio para seleccionar uno de los individuos a usar en el operador
de cruce como padre principal.
	\item El número de diferencias computadas en la operación de cruce.
	\item El operador de cruce escogido.
\end{itemize}
La variante más popular se llama ``\textit{DE/rand/1/bin}``, donde DE se refiere al
nombre del algoritmo, \textit{rand} indica una elección aleatoria de los vectores que
conforman el vector de mutación, la cifra \textit{1} señala el número de pares de vectores
que conforman la diferencia y \textit{bin} establece un proceso de cruce binomial. Así,
por ejemplo, un algoritmo de DE que selecciona aleatoriamente a cuatro vectores (2 pares) que
componen al vector de mutación, y que son recombinados con un proceso exponencial, se
representa como ''\textit{DE/rand/2/exp}''. La figura \ref{diferencial1} muestra
gráficamente el cruce binomial y exponencial.

Además de los parámetros típicos de los EAs, la DE adopta dos nuevos parámetros: $CR$,
que representa una probabilidad de cruce, normalmente entre $\left[ 0,1\right] $ y
controla la influencia de los padres en la generación de los hijos, y $F$, un parámetro
que regula las magnitudes relativas de las diferencias del vector mutación, y que suele ser
un valor aleatorio obtenido de una distribución normal $N(0,1)$ o uniforme $U(0,1)$. Estos
parámetros
dependen en cierta medida de las características de las funciones objetivo y del tamaño
de la población.

\begin{figure}[!htp]
\centering
\includegraphics[keepaspectratio,width=11.5cm]{figuras/crucesBinonialExponencial.jpg}
\caption{Cruce discreto binomial (izquierda) y cruce discreto exponencial (derecha). Los
individuos $\mathbf{x}$ e $\mathbf{y}$ dan lugar a otro individuo $\mathbf{z}$.}
\label{diferencial1}
\end{figure}

\newpage
Definimos a continuación las variantes de la DE (la figura \ref{diferencial2} muestra un resumen de
las principales variantes):
\begin{itemize}
	\item Variantes con operador de cruce discreto (binomial o exponencial):
		\begin{itemize}
			\item \textit{DE/rand/1/bin}
			\item \textit{DE/rand/1/exp}
			\item \textit{DE/best/1/bin}
			\item \textit{DE/best/1/exp}
	   \end{itemize}
		Las variantes con \textit{rand} seleccionan el padre principal y un
		par de padres secundarios para calcular la mutación diferencial aleatoriamente. En
		contraste, las	variantes con \textit{best} utilizan la mejor solución
		de la población	como padre principal, y un par de individuos seleccionados
		aleatoriamente como padres		secundarios.
	\item Variantes con cruce aritmético:
		\begin{itemize}
			\item \textit{DE/current-to-rand/1}
			\item \textit{DE/current-to-best/1}
	   \end{itemize}
		La diferencia entre ellos es que el primero selecciona el padre principal y los
		padres secundarios de manera aleatoria en la población actual, mientras que el
		segundo utiliza la mejor solución de la población actual como padre principal, y los
		padres secundarios se eligen aleatoriamente.
			\item Variantes con cruce combinado aritmético discreto (similar a las anteriores pero
		utiliza cruce binomial):
		\begin{itemize}
			\item \textit{DE/current-to-rand/1/bin}
	   \end{itemize}
\end{itemize}
\newpage
\begin{figure}[!htp]
\centering
\includegraphics[width=13.2cm,height=9cm]{figuras/modelosDE.jpg}
\caption{Principales modelos de DE. $p$ es el número de pares de vectores que
conforman la diferencia, $j_{r}$ es un valor aleatorio generado en el
intervalo $\left[ 0,n\right]$, donde $n$ es el número de variables del problema.
$x_{r3}$ es el padre principal y $x_{r1}$ y $x_{r2}$ los padres
secundarios. $F$ y $K$ son valores de
escala, $u_{i}$ es el hijo creado y $x_{best}$ significa que se ha seleccionado como padre
principal el mejor individuo o solución de la población en una determinada generación,
\textit{bin} representa cruce binario y \textit{exp} cruce exponencial, y \textit{\textbf{dir}}
indica que se incluye información de alguna función de aptitud al cruce y a la mutación.}
\label{diferencial2}
\end{figure}
% \section{Evolución diferencial en redes neuronales artificiales mono-objetivo}
% \noindent Ning Guiying en \cite{Ning2007} propone un algoritmo basado en DE y llamado
%MDE
% (Modified Differential Evolution), en el que se optimizan los individuos iniciales con
%la
% regla $1/2$, introduciendo posteriormente una reorganización de la estrategia de
%evolución
% durante el período mutación. El MDE se utiliza para optimizar los pesos de ANNs
%multicapa
% hacia delante. MDE se compara con el algoritmo BP y con el algoritmo básico de DE,
% mostrando en los resultados finales que MDE tiene una alta calidad de convergencia
% global y mejora la precisión y la velocidad de convergencia de las ANNs.

% En \cite{Lahiri2008}, se utiliza un algoritmo de DE con ANNs, llamado ANN-DE, para
% optimizar reactores industriales catalíticos de óxido de etileno. En el proceso
% evolutivo del algoritmo se construye un modelo de ANN para correlacionar los datos del
% proceso que compone los valores de funcionamiento y de rendimiento de variables del
% reactor. Para la optimización de los pesos de la red se utiliza una algoritmo de
% gradiente y un conjunto de patrones de entrenamiento y generalización con tres
%variables, % que son obtenidas de la monitorización de una serie de reactores. Una vez se
%construye el % modelo de red se genera una serie de vectores aleatorios como población
%del proceso % evolutivo formados por posibles valores de las tres variables de entrada de
%los % reactores, es decir, se obtienen un número de conjuntos de condiciones de
%funcionamiento % que intentan maximizar la producción del reactor al aplicarlos a la ANN
%diseñada, la cual % muestra qué vectores de solución son mejores. Al final del proceso
%evolutivo, que posee % tanto cruce como mutación, se obtiene un conjunto de soluciones
%optimizadas  que se % vuelven a aplicar a la ANN para escoger finalmente la mejor.
\section{Evolución diferencial para optimización multi-objetivo utilizando el concepto de
dominacia de Pareto}
\noindent Para aplicar la estrategia de la DE a problemas multi-objetivo, hay que
modificar el esquema original \cite{Storn1997}, ya que el conjunto de soluciones de un
problema con múltiples objetivos no consiste en una sola solución (ver sección
\ref{multiobjetivo} del capítulo \ref{MOEANNs}).

Hay varios aspectos que se deben considerar para extender la DE a un problema de
optimización multi-objetivo:
\begin{itemize}
	\item ¿Cómo promover la diversidad de la población?
	\item ¿Cómo seleccionar o retener los mejores individuos, es decir, cómo realizar
	elitismo?
\end{itemize}

Para promover la diversidad hay que tener en cuenta el proceso de selección por medio de
mecanismos basados en alguna medida de calidad, la cual indique la cercanía entre los
individuos que forman la población. Las dos medidas de diversidad más usadas en
optimización multi-objetivo son:
\begin{description}
	\item[Distancia \textit{crowding} \cite{Deb2002}:] Esta medida da una idea de cómo de
agrupados están los vecinos de un determinado individuo en el espacio de la función
objetivo y hace que los frentes sean lo más uniformes posible, sin agrupar muchos
individuos en una
zona y dejar ninguno o muy pocos en otras. La distancia \textit{crowding} se estima en función
de la media de las caras de un cubo formado al tomar como vértices los vecinos más
cercanos a un
individuo $i$ (ver figura \ref{figuraCrowding} y el trabajo de K. Deb \cite{Deb2002} para más
información).
	\item[Compartición de aptitud o \textit{fitness sharing}:] Cuando un individuo
comparte valores de su función de aptitud con otros, ésta se degrada en proporción al número y
a la
proximidad de los individuos que lo rodean dentro de un determinado perímetro. La vecindad
de un individuo se define en términos de un parámetro llamado $\sigma_{share}$, que
indica el radio de la vecindad, a la cual se le llama nicho.
\cite{Golberg1989,Sareni1998}.
\end{description}

El lector puede obtener en \cite{Du2010} un estado del arte actualizado en métodos de
agrupamiento para obtener diversidad en ANNs.

\begin{figure}[!htp]
\centering
\includegraphics[keepaspectratio,width=8cm]{figuras/CrowdingDistance.jpg}
\caption{Cálculo de la distancia \textit{crowding}. Los puntos azules son soluciones de un mismo
frente.}
\label{figuraCrowding}
\end{figure}

Para promover el elitismo en optimización multi-objetivo se suele utilizar un archivo
externo, llamado población secundaria, que almacena los individuos no dominados
encontrados a lo largo de la búsqueda. Uno de los métodos más populares para seleccionar
los mejores individuos de una población formada por padres e hijos es la ordenación de
no dominados. Esta técnica se basa en el mecanismo de orden que se le da a los diferentes
individuos de una población en forma de niveles. Por ejemplo, en el nivel 1 estarán los
individuos no dominados. En el segundo nivel, estarán los individuos no dominados si no
se tienen en cuenta los del primer nivel, y así sucesivamente. Según Goldberg
\cite{Golberg1989}, para mantener una diversidad apropiada, la metodología de ordenación
de no dominados se debería usar en conjunción con alguna técnica de nichos como las
mencionadas anteriormente. El algoritmo NSGAII \cite{Deb2002} es un claro ejemplo de esto.

Existen varios trabajos interesantes que utilizan DE con MOEAs y aplicaciones en
\cite{Price2005,Chakraborty2008,Mezuza2008}.

\section{Evolución diferencial de Pareto con MOANNs}
\noindent En primer lugar vamos a considerar brevemente los artículos más interesantes
sobre el uso de DE para el diseño de ANNs mediante técnicas multi-objetivo basadas en el concepto de
de dominancia y óptimo de Pareto, y en los cuales nos hemos basado para la construcción
de un
nuevo MOEA basado en DE para el diseño de ANNs con unidades de base sigmoide.

El referente principal de este tipo de aplicaciones, usando una variación del algoritmo de
DE original \cite{Storn1997}, es H. Abbass, creador del algoritmo multi-objetivo PDE
(\textit{Pareto Differential Evolution}) \cite{Abbass2001-1,Abbass2002a}. En el algoritmo
PDE se utiliza un caso especial de la variante \textit{DE/current-to-rand/1/bin} con $K=0$
(ver penúltima fila de la figura \ref{diferencial2}), ya que el padre principal se utiliza
para la creación de un nuevo hijo y también para un tipo de cruce discreto. Los objetivos a
minimizar son el $MSE$ y la complejidad de la red.

El algoritmo trabaja como sigue: La población inicial se inicializa utilizando una distribución
Gaussiana de
media $0,5$ y de desviación típica $0,15$. Solamente las soluciones no dominadas se
retienen en la población para el cruce y las dominadas se eliminan. Se seleccionan tres
padres de manera aleatoria (uno de ellos como padre principal y también como solución
hija) para generar un nuevo hijo. Lo hijos se incluyen en la población solo si dominan al
padre principal, en caso contrario, se hace un nuevo proceso de selección. Si el número de
soluciones no dominadas excede un umbral, se adopta una métrica de distancia para eliminar
padres que están muy cercanos unos de otros (esto se puede ver como un procedimiento de
nichos en el cual la métrica de distancia es el radio del nicho). En esta aproximación, el
tamaño de paso $F$ se genera a partir de una distribución Gaussiana $N(0,1)$, y las
restricciones de frontera se preservan, ya sea mediante un cambio de signo si la variable
es $\leq 0$ o mediante restas repetitivas, restando 1 si es $\geq 0$, hasta que la
variable esté dentro de los límites permitidos. El algoritmo PDE también incorpora un operador de
mutación que se aplica con una determinada probabilidad, después del operador de cruce,
mediante la suma a cada variable de una pequeña perturbación aleatoria.

A partir de la aparición del algoritmo PDE, y casi en paralelo, Abbass desarrolla el algoritmo
MPANN (\textit{Memetic Pareto Artificial Neural Networks}) \cite{Abbass2001}, que es una
versión del PDE añadiéndole un algoritmo de LS basado en gradiente como es BP, con algunas
mejoras, para así aumentar la velocidad de convergencia. MPANN trata de obtener modelos
de ANNs que tengan buena capacidad de generalización sin aumentar demasiado el tamaño de
su arquitectura. Concretamente trata  de  minimizar  el $MSE$ y el número  de neuronas en  capa
oculta. MPANN evoluciona conjuntamente
la arquitectura y los pesos de la red y utiliza operadores cruce y mutación para la obtención de
los hijos, codificando cada ANN  en  un cromosoma  que  representa  la
estructura  y el  valor  de  los pesos. Otras  versiones  de MPANN  se
utilizan para problemas reales como la diagnosis del cáncer \cite{Abbass2002a},
diferenciándose del MPANN original en la incorporación de un operador de mutación, ya que
los algoritmos PDE y MPANN carecían de ello.

Otra variación del PDE es el algoritmo SPDE (\textit{Self-adaptive Pareto Differential
Evolution}) \cite{Abbass2002}, la cual adapta de manera automática las probabilidades de
cruce y mutación. Ambas probabilidades se heredan de los padres de la misma forma que se
realiza el cruce para las variables de decisión. Si las probabilidades de cruce y de mutación
no están entre $(0,1)$, se modifican automáticamente de acuerdo a unas reglas de
reparación. Al igual que se realizó una versión auto-adaptativa del PDE con el algoritmo
SPDE, Abbass propuso una versión auto-adaptativa del algoritmo MPANN, llamada SPANN
(\textit{Self-adaptive Pareto Artificial Neural Networks}) \cite{Abbass2003}.

Un algoritmo que se debe mencionar a pesar de que sea un algoritmo evolutivo
mono-objetivo con término de regularización es el de J. Illonen \cite{Jarno2003}, donde
se propone un estudio de la DE en el diseño de ANNs para encontrar el óptimo global de
un problema. El algoritmo utiliza el $MSE$ medio regularizado mediante la media de los pesos y los
sesgos para entrenar a las ANNs. Illonen compara su metodología con métodos basados en gradiente y
concluye que la DE puede ser más útil en el caso especial de algunas superficies de
error, pero que la inclusión de alguna metodología híbrida que utilice conjuntamente
optimización evolutiva e información basada en gradiente podría ser más beneficiosa.

En \cite{Yau2007}, se proponen una serie de experimentos utilizando el algoritmo
multi-objetivo PDE de Abbass para evolucionar ANNs aplicadas a juegos de inteligencia
artificial. La metodología propuesta contienes tres subsistemas: Un subsistema PDE canónico, un
subsistema que introduce coevolución en PDE con tres
configuraciones posibles (PCDE), y un subsistema también de coevolución con PDE usando un
archivo con tres configuraciones diferentes (PCDE-A). El primer subsistema se trata del algoritmo
PDE con operadores de cruce y mutación y sin LS. El segundo sistema introduce coevolución,
siendo la evaluación de cada individuo la principal diferencia con el algoritmo PDE. Para la
ejecución
del PCDE, cada ANN se compara con un número constante de ANNs elegidas al azar de la
población de la generación actual. Si la puntuación de la ANN es mayor o igual a la de su
oponente (elegidas al azar), recibirá una victoria. Por otra parte, la clasificación del
primer frente de Pareto (mediante el etiquetado de soluciones no dominadas) se basa en el
	número de victorias como criterio de evaluación principal. En el algoritmo PDCE-A, al
igual
que en PCDE, después de la puntuación de cada ANN se lleva a cabo un segundo computo.
Sin embargo, PCDE-A tiene un archivo extra que se utiliza para almacenar las soluciones
de Pareto cada 50 generaciones. En consecuencia, cada ANN se compara con un número
mínimo de ANNs elegidas al azar (sin repeticiones), a partir del archivo extra. Sólo si el
número de ANNs en el archivo es menor que el número mínimo exigido de oponentes elegidos al
azar, entonces la lista de oponentes se completa de ANNs elegidas al azar de la
población. Del mismo modo, una ANN recibirá una victoria si su puntación es mayor o
igual que la de su competidora. El número de victorias se utilizará como
criterio de evaluación principal para etiquetar las soluciones no dominadas para
el primer frente de Pareto. Este valor de evaluación será menor al depender del
etiquetado de las soluciones dominadas, a causa del conjunto acotado de
evaluadores. De la experimentación realizada con una serie de juegos de inteligencia
artificial, se concluye que los mejores resultados los obtiene el algoritmo PDE. El pobre
desempeño de los sistemas PCDE, incluso la versión que utilizan un archivo extra, en la
producción de una buena distribución de soluciones a lo largo del frente de Pareto, es una
prueba más de que los métodos co-evolutivos no son especialmente beneficiosos para la
síntesis de agentes inteligentes para juegos en la evolución de Pareto.

En \cite{Iorio2004} se propone un algoritmo llamado NSDE (\textit{Non-dominated Sorting
Differential Evolution}), que consiste en una modificación del algoritmo NSGAII
\cite{Deb2002}, al que se introduce DE con la variante \textit{DE/current-to-rand/1} (ver
sección \ref{variantes}) en el cruce y la mutación. NSDE se utiliza para resolver problemas
de rotación de funciones en el plano, atendiendo a dos funciones objetivo, una para cada
eje de coordenadas del plano basándose en los grados de rotación. El algoritmo NSDE se
compara con NSGAII en una serie de problemas de rotación, mostrando mejores
soluciones por el proceso de cruce y mutación realizado en este tipo de problemas.

A continuación exponemos algunos de los caminos futuros de la DE usando MOEAS.

\section{Caminos futuros en la evolución diferencial multi-objetivo}
\noindent Según se puede ver en \cite{Mezuza2008}, la DE debe tener en cuenta estas
futuras
mejoras:
\begin{description}
	\item[Diversidad:] A pesar de que la DE tiene una alta convergencia, no posee
suficiente robustez y tiene problemas para alcanzar el verdadero frente de Pareto,
pudiendo quedar atrapada en óptimos locales. Además parece que la DE tiene problemas para
crear un frente de Pareto homogéneo, con lo que se deberían aplicar alternativas de
diversidad cuando se utilice con problemas multi-objetivo.
	\item[Variantes:] A día de hoy no se sabe que variante de DE es mejor para problemas
multi-objetivo para alcanzar el verdadero frente de Pareto de menera más efectiva.
	\item[Operador de mutación:] Se deben tomar algunos nuevos criterios a la hora de
seleccionar pares de soluciones en el proceso de mutación que sean más efectivos
\cite{Iorio2006}. En este momento estamos estudiando la selección de soluciones utilizando
intervalos  de confianza asociados a las distribuciones de los mejores individuos de la población
\cite{Cruz2010}.
	\item [Adaptación de los parámetros:] Debe haber nuevas propuestas que no sean
solamente las autoadaptativas \cite{Abbass2002,Abbass2003} para optimizar los parámetros
CR y F.
	\item[Alternativas en la codificación:] DE se propuso para espacios se búsqueda
continuos, por lo que se debería buscar una alternativa de codificación que permita el
uso de la DE en problemas de optimización combinatoria.
	\item[Teoría:] Los estudios sobre la convergencia de las variantes de DE y análisis en
tiempo de ejecución, mejorarían la teoría actual.
 \end{description}

\section{El algoritmo MPANN}
\noindent A continuación exponemos el algoritmo MPANN \cite{Abbass2001} en su versión
adaptada al reconocimiento de diagnosis del cáncer \cite{Abbass2002a}. Ésta versión
se diferencia del algoritmo MPANN original en que incorpora un operador de mutación, del
cual
carecían los algoritmos PDE y MPANN originales. Además, tendremos en cuenta el algoritmo
NSGAII \cite{Deb2002}, que nos servirá de base para la creación de nuestro algoritmo MPDE para el
diseño de ANNs en multi-clasificación de patrones.

En la figura \ref{diferencial3}, se presenta el pseudocódigo del algoritmo
MPANN que comentamos a continuación:

El primer paso de MPANN es generar una población inicial al azar siguiendo una
distribución Gausiana $(0,1)$ (Paso 1).

Acto seguido, el algoritmo comienza su proceso evolutivo:

La primera acción (Paso 4) que tiene lugar al comienzo de una generación es la evaluación
de los individuos, en el caso de MPANN utilizando la minimización del $MSE$ y la
complejidad
de la red (número de neuronas en capa oculta). Una vez evaluados, se etiquetan los individuos no
dominados.

Si el número de individuos no dominados es menor que 3, se busca un individuo no dominado
de entre las soluciones que no están etiquetadas y se etiqueta como no dominado. Esto se
repite hasta que el número de no dominados sea igual a 3 (Paso 5).
\newpage
\begin{figure}[!htp]
\centering
\fbox{
	\includegraphics[keepaspectratio,width=12.2cm]{figuras/etapasMPANN.jpg}
}
\caption{Pseudocódigo del algoritmo MPANN.}
\label{diferencial3}
\end{figure}

Seguidamente (Paso 6), se eliminan todas las soluciones dominadas de la población.

A continuación (Paso 7), se marca un 20\% del conjunto de patrones de entrenamiento como
conjunto de validación para la LS.

El siguiente paso (Paso 8) es el más importante de todo el algoritmo, ya que se trata de
la generación de hijos. Este paso se repetirá hasta que la población alcance un tamaño
máximo, $M$, fijado de antemano.

La primera acción a realizar (Paso 8.1) es la selección aleatoria de tres
padres. Uno de ellos se etiqueta como padre principal $(\alpha_{1})$ y los otros dos
como secundarios $(\alpha_{2},\alpha_{3})$.

A continuación (Paso 8.2), tiene lugar la operación de cruce. En la operación de cruce se
calcula aleatoriamente una probabilidad uniforme en el intervalo $(0,1)$ para cada una de
las
neuronas de la capa oculta en conexión con la capa de entrada. Si el valor obtenido es menor que el
valor de $CR$ (\textit{Crossover Probability}), se aplican las expresiones (\ref{1}) y
(\ref{2}), donde  $w_{ih}^{hijo}$ se refiere, en el hijo, a los pesos asociados a cada una de las
neuronas de entrada, $i$, con cada una de las neuronas de la capa oculta, $h$. $\rho_{h}^{hijo}$ se
refiere a la existencia o no en la capa oculta de las neuronas del nuevo hijo. Todo este
proceso se hace para cada neurona de la capa oculta en conexión con la capa de entrada, siendo el
número máximo de neuronas ocultas prefijado al inicio del algoritmo.
\begin{equation}\label{1}
w_{ih}^{hijo} \leftarrow w_{ih}^{\alpha_{1}} +
N(0,1)(w_{ih}^{\alpha_{2}}-w_{ih}^{\alpha_{3}})
\end{equation}
\begin{equation}\label{2}
 \rho_{h}^{hijo} \leftarrow \left\lbrace
 \begin{array}{ll}
 1 & \mbox{si
$\rho_{h}^{\alpha_{1}}+N(0,1)(\rho_{h}^{\alpha_{2}}-\rho_{h}^{\alpha_{3}})\geq 0.5$} \\
 0 & \mbox{en otro caso}
 \end{array}
 \right.
\end{equation}

En el caso de que el valor aleatorio obtenido en el intervalo $(0,1)$ no sea menor que $CR$, se
aplican las expresiones (\ref{3}) y (\ref{4}).
\begin{eqnarray}
w_{ih}^{hijo} \leftarrow w_{ih}^{\alpha_{1}} \label{3} \\
\rho_{h}^{hijo} \leftarrow \rho_{h}^{\alpha_{1}} \label{4}
\end{eqnarray}

Una vez finalizada la operación de cruce para todas las neuronas de la capa oculta en
conexión
con la capa de entrada, es el turno de las neuronas de capa oculta en conexión con la capa de
salida. De nuevo se obtiene un valor aleatorio probabilístico uniforme en el intervalo $(0,1)$, y si
el valor obtenido es menor que el valor de $CR$, se aplica la expresión (\ref{5}), donde
$w_{ho}^{hijo}$ significa el peso asociado a la conexión que
va desde la neurona $h$ de la capa oculta hasta la neurona $o$ de la capa de salida  del nuevo hijo.
Esto se hace, al igual que antes, para todas las neuronas que haya en capa oculta en conexión con la
capa de salida.
\begin{equation}\label{5}
w_{ho}^{hijo} \leftarrow w_{ho}^{\alpha_{1}} +
N(0,1)(w_{ho}^{\alpha_{2}}-w_{ho}^{\alpha_{3}})
\end{equation}

En el caso de que el valor aleatorio obtenido en el intervalo $(0,1)$ no sea menor que $CR$, se
aplica la siguiente expresión:
\begin{equation}\label{6}
w_{ho}^{hijo} \leftarrow w_{ho}^{\alpha_{1}}
\end{equation}

Durante la operación de cruce, al menos una variable del hijo se debe modificar para que
sea distinto al padre principal.

A continuación se realiza la operación de mutación (Paso 8.3), calculándose
una probabilidad uniforme $(0,1)$ para cada una de las neuronas de capa oculta del hijo, tanto
las que están en conexión con la capa de entrada como las que están en conexión con la capa de
salida. Si el valor obtenido es menor que el valor de $MR$ (\textit{Crossover Probability}), se
aplican las expresiones (\ref{7}) y (\ref{9}), o la expresión (\ref{8}) si el cambio que se va a
producir es para una conexión de la  capa de oculta con la capa de entrada, o si
es para una conexión de la  capa de oculta con la capa de salida respectivamente. Si el valor
aleatorio obtenido es mayor que $MR$, no se hace la mutación. La nomenclatura que se sigue es la
misma que para la operación de cruce.
\begin{eqnarray}
w_{ih}^{hijo} \leftarrow w_{ih}^{hijo} +
N(0,\text{porcentaje mutación}) \label{7} \\
w_{ho}^{hijo} \leftarrow w_{ho}^{hijo} +
N(0,\text{porcentaje mutación}) \label{8}
\end{eqnarray}
\begin{equation}\label{9}
\rho_{h}^{hijo} \leftarrow \left\lbrace
\begin{array}{ll}
1 & \mbox{si
$\rho_{h}^{hijo}=0$} \\
0 & \mbox{en otro caso}
\end{array}
\right.
\end{equation}

Una vez realizadas las operaciones de cruce y mutación, se aplica al hijo resultante la LS (Paso
8.4), usando el algoritmo de retropropagación BP, aplicando el
conjunto de validación del paso 7.

Este hijo se evalúa y se añade a la población si presenta una relación de dominancia
con respecto al padre principal (Paso 8.5 y Paso 8.6)

Este proceso se repetirá a lo largo de las generaciones hasta que se cumpla la condición
de parada (Paso 10).

\section{El algoritmo MPDE}
\noindent A continuación exponemos nuestro algoritmo basado en DE para multiclasificación de
patrones usando el concepto de dominancia de Pareto. Dicho algoritmo se llama MPDE
(\textit{Memetic Pareto Differential Evolution}) \cite{Fernandez2009}. Nuestro procedimiento, al
igual que el algoritmo MPENSGAII descrito en el
capítulo \ref{MOEANNs}, evoluciona simultáneamente los pesos y la arquitectura de la red,
y se encarga de diseñar modelos de red para multi-clasificación de patrones.

MPDE obtiene diferentes conjuntos de clasificadores no dominados que presentan un buen
balance entre precisión y $MS$ (ver capítulo
\ref{medidasRendimiento}), que son los dos objetivos a optimizar.

La población de individuos está sujeta a operaciones de cruce y de mutación. En cuanto a la
codificación de los modelos de red, se sigue la misma codificación explicada en los algoritmos CBFEP
y MPENSGAII del capítulo \ref{evoMonoObjetivo} y \ref{MOEANNs} respectivamente.

MPDE está hibridado con un algoritmo de LS, y utiliza funciones de base sigmoides (SUs).

Para llevar a cabo un proceso de elitismo, MPDE se basa en algunos aspectos de NSGAII
\cite{Deb2002}, utilizando como característica más representativa el ordenamiento rápido
de no-dominados para la obtención del frente de Pareto.

Con respecto a la diversidad se utiliza la distancia \textit{crowding} de NSGAII, para el
caso en que haya que completar la población hasta alcanzar un número determinado de
individuos, y también una ecuación de cálculo de la distancia  de un individuo a los dos
vecinos más cercanos, la cual comentaremos en las siguientes secciones.

En cuanto a la variante de la DE, se trata de la variante \textit{DE/rand/1/bin}.
\newpage
\subsection{Funciones objetivo}
\indent Las funciones objetivo a optimizar con MPDE son las mismas que se utilizan con
MPENSGAII, $E$ y $MS$, ya que pensamos que un buen
clasificador
debería obtener un alto nivel de
precisión global, así como un aceptable nivel de clasificación para cada clase de un determinado
problema:
\begin{itemize}
\item \textbf{Objetivo 1:} La mínima sensibilidad de todas las clases de un problema, $MS$:
\begin{displaymath}
A_{1}\left( g,\mathbf{\Theta}\right) = MS(g)
\end{displaymath}
\item \textbf{Objetivo 2:} La entropía cruzada como medida de error global, $E$. Concretamente, como
función
de aptitud a la hora de evaluar un individuo se usará la siguiente expresión:
\begin{displaymath}
\label{aptitudNNEP}
A_{2}\left( g,\mathbf{\Theta}\right) =\frac{1}{1+E\left( g,\mathbf{\Theta}\right)},
\end{displaymath}
es decir, maximizar una transformación estrictamente decreciente de $E$.
\end{itemize}

A la hora de asignar una clase a una nueva observación se sigue el esquema ``1 de $Q$''
explicado en los algoritmos CBFEP y MPNESGAII.

\subsection{Operadores}
\noindent En cuanto a los operadores de cruce y mutación se utilizarán los mismos que los
del algoritmo MPANN, pero adaptados a nuestra representación.

Las expresiones (\ref{1}) a (\ref{6})
representan el operador de cruce y las expresiones (\ref{7}) a (\ref{9}) el
operador de mutación. El significado de la nomenclatura seguida es la misma que la
explicada en el algoritmo MPANN.

\subsection{Búsqueda local}
\noindent Como algoritmo de búsqueda local usamos el algoritmo iRprop+ utilizado con el
algoritmo MPENSGAII (ver sección \ref{rprop} del capítulo \ref{MOEANNs}). En la siguiente
sección se explica detalladamente cuándo se utiliza el algoritmo iRprop+ y se comenta
cada una de las etapas de nuestra metodología.

\subsection{Etapas y aspectos relevantes de MPDE}
\noindent En la figura \ref{diferencial3} se muestran las etapas del algoritmo MPDE, que
pasamos a comentar:

\begin{figure}[!htp]
\centering
\fbox{
	\includegraphics[keepaspectratio,width=12.2cm]{figuras/etapasMPDE.jpg}
}
\caption{Pseudocódigo del algoritmo MPDE.}
\label{diferencial4}
\end{figure}

El algoritmo comienza con la creación de una población de individuos tomados al azar,
siendo $M$ el tamaño de la población (Paso 1).

Empieza el proceso evolutivo hasta que se cumpla la condición de parada (Paso 3). Los
individuos se evalúan en base a las dos funciones objetivo que guían el algoritmo y se
realiza un ordenamiento rápido de no dominados equivalente al del algoritmo NSGAII
\cite{Deb2002}. Se etiquetan entonces aquellos que sean no dominados. (Paso 4)

Si el número de soluciones no dominadas es menor que 3 (Paso 5), entonces se repite el siguiente
proceso hasta que haya al menos 3 soluciones: Encontrar una solución no dominada
entre las que no están etiquetadas, en función del orden asignado en el ordenamiento
rápido de no dominados. En caso de empate en orden, se utiliza el valor de la distancia
\textit{crowding} de NSGAII (Paso 5.1) y se elige la solución con mayor distancia. A
continuación se
etiqueta el individuo escogido como no dominado (Paso 5.2).

Si el número de soluciones no dominadas ya era de al menos 3 individuos, se comprueba si
el número de soluciones es mayor que $M/2$ (Paso 6). Si es así, se calcula la distancia de
cada individuo a sus dos vecinos más cercanos (Paso 6.1), y se elimina el individuo con
menor distancia (Paso 6.2). El cálculo de la distancia viene dado por:
\begin{displaymath}
D(x)=\frac{(min\|x-x_{i}\|+min\|x-x_{j}\|)}{2}
\end{displaymath}
siendo $x$ el individuo al que se le va a calcular la distancia a sus dos vecinos más
cercanos, siendo estos $x_{i}$ y $x_{j}$. Esto nos permite mantener un mayor grado de
diversidad y que el algoritmo no quede estancado en el caso de que el número de
soluciones no dominadas sea cercano a $M$, ya que el tamaño de la población suele ser
pequeño, con lo que obtendríamos de una generación a otra muy pocos
individuos mejorados.

En el paso 7 se eliminan de la población actual todas las soluciones no etiquetadas, es
decir, las soluciones dominadas.

Se pasa ahora a completar la población actual para prepararla para la siguiente
generación hasta que el número de individuos sea igual a $M$ (Paso 8).

Primero se selecciona aleatoriamente un individuo como padre principal y otros dos como
padres secundarios, todo ello sin repetición, para que los 3 sean distintos (Paso 8.1).

A partir de esos tres individuos realizamos una serie de operaciones hasta
obtener un nuevo hijo que se añada a la población. En primer lugar aplicamos el operador
de cruce a partir del padre principal y a partir de los padres secundarios, con una
probabilidad de cruce uniforme designada como $CR$ (Paso 8.2). El hijo que se obtenga, tendrá
características de los tres padres.

El hijo obtenido se evalúa en base a las dos funciones objetivo que guían al algoritmo
(Paso 8.3), y si el individuo es igual al padre porque no se hayan producido cambios con
las operaciones realizadas anteriormente, se le fuerza a cambiar aleatoriamente un enlace,
añadiéndole un valor de una distribución Gaussiana $N(0,1)$ (Paso 8.4).

A continuación, se aplica al hijo el operador de mutación en cada una de sus
neuronas de la capa oculta (Paso 8.5). Al comenzar el algoritmo se establece un número
máximo de neuronas como en el algoritmo MPENSGAII (ver capítulo \ref{MOEANNs}). Para el
total del máximo de neuronas, si la neurona $i$ existe se elimina, y si no, se añade,
estableciendo enlaces y pesos de la misma forma que la mutación añadir neurona del
algoritmo MPENSGAII. Todo ello aplicando la mutación con una probabilidad determinada.

Cuando el nuevo hijo se ha creado, aumentamos en 1 el valor de una
variable llamada ``creados'', que nos servirá, en un momento dado de la evolución. Concretamente
cuando se hayan creado muchos hijos y ninguno de ellos por las circunstancias que se explican a
continuación se pueda añadir a la población actual (Paso 8.6).

Evaluamos al hijo en base a las dos funciones objetivo que guían a MPDE (Paso 8.7).

Si el hijo domina al padre principal se le aplica la LS con el algoritmo iRprop+ y se añade a
la población actual (Paso 8a y 8b). En caso contrario, si no hubiera relación de dominancia entre
padre e hijo, también se añade el hijo a la población actual (Paso 8c). En caso contrario, si el
padre principal domina al hijo, se comprueba si $creados=100$. En ese caso se elige el mejor de los
100 hijos almacenados, en base a la función objetivo $A_{1}$, y la variable creados se establece a
$0$ (Paso 8d y 8e). Sino se produce nada de lo anterior, el candidato es descartado (Paso
8f). El paso 8 se repite entero hasta completar el tamaño de la población que se establece en la
variable $M$.

Cuando se complete el tamaño de la población, ésta queda preparada para la siguiente
generación (Paso 9), y el proceso evolutivo continúa hasta que se cumpla la condición de
parada, que en nuestro caso es un número determinado de generaciones (Paso 10).


\subsection{Diferencias con el algoritmo MPANN}\label{diferencias}
\noindent Las diferencias fundamentales con respecto al algoritmo MPANN \cite{Abbass2002a} son las
siguientes:
\begin{itemize}
	\item El operador de cruce que nosotros utilizamos, también calcula aleatoriamente una
probabilidad uniforme en el intervalo $(0,1)$, y si el valor obtenido es menor que
el	valor de $CR$ no se aplica el operador. La diferencia está en que nuestro método no aplica el
valor de probabilidad obtenido aleatoriamente a todos los enlaces y neuronas de la capa oculta, sino
que utilizamos un nuevo valor aleatorio dentro del intervalo uniforme definido para
cada neurona y no para la capa oculta entera, como en el caso de MPANN. En este caso, nuestro
operador de cruce es menos agresivo con los cambios en las ANNs, ya que en alguna ocasión puede
que el valor aleatorio obtenido a partir de una probabilidad uniforme no sea menor que $CR$, con lo
que la neurona que se esté tratando es ese momento no cambia.
	\item La probabilidad de mutación $MR$, también se utiliza de manera independiente, al igual que
en el cruce, para cada neurona, y no para la capa oculta entera, como en el caso del algoritmo
MPANN.
	\item La manera en que  se añaden los individuos a la población en el algoritmo MPANN
	(solo los que dominan al padre principal), hace que el algoritmo pueda quedar estancado durante
	un buen número de generaciones (se ha comprobado experimentalmente) hasta que se pueda añadir
	un nuevo hijo. Nosotros añadimos individuos de una manera más ``relajada'', de
	manera que hijos que no dominen al padre principal tienen la opción de añadirse a la
	población. De esta manera también se reduce el coste computacional sin disminuir la
	calidad de los resultados.
\end{itemize}

\subsection{Diseño experimental}
\noindent Para analizar el rendimiento de MPDE hemos utilizado 6
conjuntos de datos del repositorio de la UCI \cite{UCI2007}.

En la tabla \ref{tabla1MPDE} se muestran las características de cada
conjunto: Número total de patrones por cada
conjunto de datos, número de patrones en entrenamiento y en generalización, número de
variables de entrada, número de clases, número total de patrones por clase y valor de $p^*$.

\begin{table}[htb!]
\scriptsize
\caption{Características de los conjuntos de datos de la UCI.}
\label{tabla1MPDE}
\centering
\tabcolsep 1pt
\begin{tabular}{c c c c c c p{2.5cm} c}
\hline
\rowcolor[rgb]{0.70,0.85,1}\textbf{Conjunto} & \textbf{Patrones} &
\textbf{Patrones} & \textbf{Patrones} &
\textbf{Variables} & \textbf{Clases} &
\textbf{Patrones} & $\mathbf{p^{*}}$ \\
\rowcolor[rgb]{0.70,0.85,1} & & \textbf{entrena.} & \textbf{generaliz.} & \textbf{de
entrada} & & \textbf{por clase} & \\ \hline
% \multicolumn{8}{>{\columncolor[rgb]{0.70,0.85,1}}c}{Dos clases} \\ \hline
\rowcolor[rgb]{0.86,0.94,1}Autos & 205 & 152 & 53 & 72 & 6 & 67-3-22-54-32-27 & 0.0188 \\
\rowcolor[rgb]{0.86,0.94,1}Balance & 625 & 469 & 156 & 4 & 3 & 288-49-288 & 0.0641 \\
\rowcolor[rgb]{0.86,0.94,1}BreastC & 286 & 215 & 71 & 15 & 2 & 201-85 & 0.2957 \\
\rowcolor[rgb]{0.86,0.94,1}HeartStatlog & 270 & 202 & 68 & 13 & 2 & 150-120 & 0.4411 \\
\rowcolor[rgb]{0.86,0.94,1}Newthyroid & 215 & 161 & 54 & 5 & 3 & 150-35-30 & 0.1296 \\
\rowcolor[rgb]{0.86,0.94,1}Pima & 768 & 576 & 192 & 8 & 2 & 500-268 & 0.3489 \\ \hline
\end{tabular}
\end{table}

El diseño experimental consiste en una partición estratificada del conjunto de datos con $3n/4$
patrones para
el conjunto de entrenamiento y $n/4$ patrones para el conjunto de generalización, siendo $n$ el
tamaño del conjunto.

El proceso de obtención de resultados es el mismo que el utilizado con MPENSGAII (ver
sección \ref{disenio} del capítulo \ref{MOEANNs}). Una vez se forma el frente de Pareto
se utilizan dos estrategias de selección
automática de individuos, el mejor modelo en $E$ y el mejor modelo en $MS$
(extremos del frente de Pareto). En cada ejecución del algoritmo (hacemos 30, dado que el proceso
de entrenamiento de la red es estocástico), una vez que tenemos el frente de Pareto de la última
generación del proceso evolutivo, se escogen los
extremos del frente en entrenamiento. Esto es, el mejor individuo en $E$, y el
mejor individuo en $MS$. A estos individuos se les llamamos individuo
$EI$, para el primer caso, e individuo $MSI$, para el segundo. Cuando tenemos los
individuos del paso anterior calculamos su valor de $C$ y de $MS$, sobre el conjunto
de generalización. De esta manera, tenemos para los extremos del frente dos pares de valores,
$\displaystyle EI=(C_{EI},MS_{EI})$ y $\displaystyle MSI=(C_{MSI},MS_{MSI})$ de una ejecución de las
30 realizadas.

Al repetirse el proceso anterior 30 veces obtenemos la media y la desviación
típica de los dos pares de valores para los individuos $EI$ y $MSI$, es decir,
$\displaystyle \overline{EI}=(\overline{C}_{EI},\overline{MS}_{EI})$ y
$\displaystyle \overline{MSI}=(\overline{C}_{MSI},\overline{MS}_{MSI})$, de forma que la
primera expresión muestra el rendimiento medio obtenido teniendo en cuenta solo los
mejores individuos en $E$, mientras que la segunda expresión muestra el rendimiento
medio teniendo en cuenta solo los mejores individuos en $MS$. A la manera
de obtener automáticamente el rendimiento medio teniendo en cuenta los mejores individuos
en $E$ (parte superior del frente) le hemos llamado MPEDEE, a y la forma de obtener el
rendimiento medio teniendo en cuenta los mejores individuos en $MS$ (parte inferior del
frente) la hemos llamado MPDES.

La probabilidad de cruce se estableció a $CR=0.8$ y la de mutación a $MR=0.1$, que es la
adoptada por Abbass en MPANN, y el tamaño de la población se estableció como $M=25$.

Para iRprop+, los parámetros adoptados son $\eta^{-}=0.5$ (tamaño de paso para el factor
de decremento), $\eta^{+}=1.2$ (tamaño de paso para el factor de incremento),
$\bigtriangleup_{0} =0.0125$ (valor inicial de tamaño de paso para los pesos,
$\bigtriangleup_{ij}$),  $\bigtriangleup_{min} =0$ (tamaño
mínimo de paso para los pesos), $\bigtriangleup_{max} =50$ (tamaño máximo de paso
para los pesos), $Epochs=5$ (número de épocas para la optimización local).

\subsection{Resultados}
\noindent Hemos comparado MPDE con nuestro algoritmo MPENSGAII y con la metodología SVM, a
partir del algoritmo SMO que proporciona Weka \footnote{http://www.cs.waikato.ac.nz/ml/weka/}
\cite{Witten2005}.

La tabla \ref{tabla2MPDE} presenta los valores de media y desviación típica para $C$ y
$MS$ obtenidos de los mejores modelos en $E$ en cada ejecución. Observar que en Balance y
Breast Cancer, el algoritmo MPDES obtiene los mejores valores en $MS$, encontrándose muy
cercano al algoritmo MPENSGAII en valores de $C$. En Autos, el mejor resultado en $C$ lo
obtiene MPDEE, pero el mejor resultado en $MS$ lo consigue el algoritmo MPENSGAIIS. En Newthyroid,
MPDE obtiene los mejores valores en $MS$ y $C$, y en Pima y Heart Statlog, MPDES obtiene
los mejores valores en $MS$, y muy similares en $C$ a los que obtiene MPENSGAIIE.

\begin{table}[!htb]
\tiny
\caption{Resultados estadísticos para MPDE, MPENSGAII y SVM en media y desviación típica
sobre el conjunto de generalización para $C$ y $MS$.}
\label{tabla2MPDE}
\centering
\tabcolsep 2pt
\renewcommand{\arraystretch}{1.2}
\begin{tabular}{llccllcc}
\hline
\rowcolor[rgb]{0.70,0.85,1}\textbf{Conjunto} & \textbf{Algoritmo} & \textbf{C(\%)} &
\textbf{MS(\%)} & \textbf{Conjunto} & \textbf{Algoritmo} & \textbf{C(\%)} &
\textbf{MS(\%)} \\ \hline
\rowcolor[rgb]{0.86,0.94,1}Autos & MPDEE & \textbf{68.79$\pm$5.59} & 28.75$\pm$21.40 &
Balance & MPDEE & 91.43$\pm$1.01 & 54.36$\pm$26.25 \\
\rowcolor[rgb]{0.86,0.94,1}& MPDES & 64.15\textit{$\pm$}5.63 &
12.26\textit{$\pm$}20.54\textbf{} &  & MPDES & 91.41$\pm$1.53 & \textbf{87.42$\pm$4.32}
\\
\rowcolor[rgb]{0.86,0.94,1}& MPENSGAIIE & \textit{66.67$\pm$4.07} &
\textit{39.64$\pm$14.92} &  & MPENSGAIIE & \textbf{94.01}$\pm$1.52\textbf{} &
42.66$\pm$17.00 \\
\rowcolor[rgb]{0.86,0.94,1}& MPENSGAIIS & 66.04\textit{$\pm$}4.78\textbf{\textit{}} &
\textbf{42.28\textit{$\pm$}10.98}\textit{\underbar{}} &  & MPENSGAIIS &
\textit{92.47$\pm$2.16} & \textit{83.72$\pm$8.19} \\
\rowcolor[rgb]{0.86,0.94,1}& SVM & 67.92 & 0.00 &  & SVM & 88.46 & 0.00 \\ \hline
\rowcolor[rgb]{0.86,0.94,1}BreastC & MPDEE & \textit{67.27$\pm$2.71} & 38.09$\pm$11.59 &
Newthy & MPDEE & \textit{96.66$\pm$2.02} & \textit{81.42$\pm$10.74} \\
\rowcolor[rgb]{0.86,0.94,1} & MPDES & 65.39$\pm$3.40\textbf{} & \textbf{57.04$\pm$7.01} &
& MPDES & \textbf{96.66$\pm$1.84} & \textbf{81.64$\pm$9.76} \\
\rowcolor[rgb]{0.86,0.94,1}& MPENSGAIIE & \textbf{69.34$\pm$2.30} & 28.88$\pm$9.09 &  &
MPENSGAIIE & 95.12$\pm$2.30\textbf{} & 74.81$\pm$10.07 \\
\rowcolor[rgb]{0.86,0.94,1} & MPENSGAIIS & 63.99$\pm$3.10 & \textit{53.08$\pm$6.57} &  &
MPENSGAIIS & 95.55$\pm$2.15 & 75.07$\pm$10.66\textit{} \\
\rowcolor[rgb]{0.86,0.94,1}& SVM & 64.79 & 23.81 &  & SVM & 88.89 & 55.56 \\ \hline
\rowcolor[rgb]{0.86,0.94,1}Pima & MPDEE & \textit{78.59$\pm$1.59} & 61.94$\pm$4.10 &
HeartStlg & MPDEE & 76.17$\pm$1.41 & 61.11$\pm$2.20 \\
\rowcolor[rgb]{0.86,0.94,1}& MPDES & 77.11$\pm$2.20\textbf{\textit{}} &
\textbf{73.12$\pm$2.98} &  & MPDES & \textit{76.27$\pm$1.57} & \textbf{63.66$\pm$2.37} \\
\rowcolor[rgb]{0.86,0.94,1}& MPENSGAIIE & \textbf{78.99$\pm$1.80} & 60.44$\pm$2.59 &  &
MPENSGAIIE & \textbf{78.28$\pm$1.75}\textit{} & 61.88$\pm$2.08\textit{} \\
\rowcolor[rgb]{0.86,0.94,1}\textbf{} & MPENSGAIIS & 76.96$\pm$2.08 &
\textit{72.68$\pm$3.06} &  & MPENSGAIIS & 77.5$\pm$1.73 & \textit{62.66$\pm$2.38}\textbf{}
\\
\rowcolor[rgb]{0.86,0.94,1} & SVM & 78.13 & 50.75 &  & SVM & 76.47 & 60.00 \\ \hline
\multicolumn{8}{l}{Los mejores resultados se muestran en \textbf{negrita} y los segundos
mejores resultados se muestran en \textit{cursiva}.}\\
\end{tabular}
\end{table}

Como ejemplo gráfico en la figura \ref{figuraBalance}, presentamos los resultados obtenidos por
el algoritmo MPDE en el conjunto de datos Balance, y al igual que en el caso de MPENSAII (ver
sección \ref{resultadosMPENSGAII} del capítulo \ref{MOEANNs}), están divididos en gráficos de
entrenamiento $(A_{1},A_{2})$, y en gráficos de generalización $(A_{1},C)$.

En nuestra opinión, el uso de la DE junto con métodos de LS usando $E$ y $MS$ como
objetivos
a optimizar, puede ser un nuevo punto de vista para tratar problemas multi-clase en clasificación,
con resultados muy prometedores.

En el siguiente capítulo se expone una aplicación realiza con el algoritmo MPENSGAII que
estudiamos y detallamos en el capítulo \ref{MOEANNs}, concretamente una aplicación sobre
microbiología predictiva \cite{Valero2009}.

\begin{figure}[!htb]
\centering
\includegraphics[keepaspectratio, width=8cm]{figuras/ejemploBalanceMPDE.jpg}
\caption{Frente de Pareto en entrenamiento $(A_{1},A_{2})$, y valores asociados a\\
$(A_{1},C)$ en generalización para el conjunto de datos Balance.}
\label{figuraBalance}
\end{figure}
\paginavaciacompleta
% \section{Introducción a los intervalos de confianza}
%
% \subsection{Intervalos de confianza aplicados a la evolución diferencial con redes}
% \noindent Aquí se hablará de los intervalos de confianza, las distintas normas y se
% pondrá HPDE con y sin intervalos de confianza.
% Se podría hacer también una breve comparación con resultados obtenidos sin intervalos de
% confianza para ver si se mejora.
%
% Velazquez tesis, pag 96
% \chapter{El algoritmo HPDCI}
%
% \section{Funciones objetivo}
%
% \section{Operadores}
%
% \section{iRprop+ como búsqueda local}
%
% \section{Etapas y aspectos relevantes de HPDE}
%
% \section{Resultados}
		%\paginavaciacompleta
		% -------------------------------------------------------------------------------

      %\include{anexo}
		%  -------------------------------------------------------------------------------
		% Localizacion de Bibliografía .bib para BibTex DESCOMENTAR LAS 2 de BIBLIOGRAFIA
		%\bibliography{./bibliografia/tesis}
		% Estilo de la Bibliografía conforme aparece referenciada en el documento
		%\bibliographystyle{unsrt}
		%-------------------------------------------------------------------------------

  \paginavaciacompleta
\end{document}